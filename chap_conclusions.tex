\chapter{Conclusions}
\label{chap:conclusions}

A search for supersymmetry is presented based on a data sample of pp collisions collected at $\sqrt{s} = 8$ \TeV, corresponding to an integrated luminosity of 11.7$\pm$0.5 fb$^{-1}$. Final states with two or more jets and significant \met, a typical final state topology of R-parity conserving \ac{SUSY} models have been analysed. The sum of standard model backgrounds per bin are estimated from a simultaneous binned likelihood fit to hadronic, \mupjets, \dimupjets, and \gpjets samples. Systematic errors due to theory, detector effects and analysis choices are quantified through the use of data driven closure tests and accounted for in the final interpretation. 

No excess of events is observed over the expected \ac{SM} background. The analysis is further interpreted in a set of \ac{SMS} models, with a special emphasis on third generation squarks and compressed spectra scenarios. In the considered models with gluino pair production and for small \ac{LSP} masses, exclusion limits of the gluino mass are in the range 950-1125 \GeV. For \ac{SMS} with squark pair production, first or second generation squarks are excluded up to around 775 \GeV and bottom squarks are excluded up to 600 \GeV, again for small \ac{LSP} masses.

A complementary approach using a templated method to estimate the b-tag jet distribution of \ac{SM} processes, is used to search for gluino induced third generation squark \ac{SUSY} production. The \alphat analysis is used to demonstrate conceptually and experimentally this technique in the \mupjets control sample. This method is further applied to the \alphat hadronic search region where good agreement is observed between the data and the background estimation procedure of the \alphat analysis.

The performance of the Level-1 trigger for jets and energy sum quantities is also presented. These studies quantify any change in level-1 performance after the introduction of a 5 \GeV jet seed threshold into the jet algorithm configuration. No significant change in single jet trigger efficiencies is observed and good performance is observed for a range of level-1 quantities.

