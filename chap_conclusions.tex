\chapter{Conclusions}
\label{chap:conclusions}

A search for supersymmetry has been presented based on a data sample of pp collisions collected at $\sqrt{s} = 8$ \TeV, corresponding to an integrated luminosity of 11.7$\pm$0.5 fb$^{-1}$. Final states with two or more jets and significant missing transverse energy, a typical final state topology of R-parity conserving \ac{SUSY} models have been analysed. The \alphat variable is utilised as the main discriminator between balanced multi-jet backgrounds and those with real missing transverse energy. 

Within the search presented, \acf{SM} backgrounds are estimated from a simultaneous binned likelihood fit to a hadronic signal selection as well as three \ac{SM} process enriched control samples. The search is split into total transverse hadronic energy (\theht), jets identified as originating for a b-quark (\nbreco), and jet multiplicity (\njet) categories to improve sensitivity to a range of possible supersymmetric final states. Systematic errors due to theory, detector effects and simulation deficiencies are quantified through the use of data driven closure tests and accounted for in the final interpretation. Observations in data are found to be compatible with a \ac{SM} only hypothesis. 

In the absence of a signal like excess the analysis is further interpreted in a set of \acf{SMS} models, representing a set of model independent decay topologies parameterised only by the production process and the masses of their parent sparticle and \acf{LSP}. In models mediated by gluino pair production and containing a large mass difference between the gluino and \ac{LSP}, exclusion limits of the gluino mass are set in the range 950-1125 \GeV. For \ac{SMS} models describing direct squark pair production, first or second generation squarks are excluded up to 775 \GeV, with direct bottom squarks production excluded up to masses of 600 \GeV. 

In the case of gluino mediated third generation signatures containing many jets originating from b-quarks in the final state, mass limits are set in the range of 975-1125 \GeV for large mass splittings between the gluino and the \ac{LSP}. The experimental sensitivity to these models is attributed to the \nbreco categorisation of the analysis, where the signal-to-background is enhanced within the phase space of the search at high \nbreco.

Furthermore, a measurement of the performance of the Level-1 trigger for jets and jet energy sum quantities has also been presented. These studies quantify any change in Level-1 performance after the introduction of a 5 \GeV jet seed threshold into the jet clustering algorithm.  No significant change in single jet trigger efficiencies is observed and good performance is observed for a range of Level-1 jet energy sum quantities.

This change was introduced to facilitate a reduction in the rate of events triggered by energy deposits due to soft non-collimated jets from secondary interactions, and which are not of interest to physics analyses. This was necessary to ensure, that trigger thresholds can be maintained at low values in the presence of an ever increasing number of bunch crossings per proton interaction. In the context of \ac{SUSY}, this is a necessity to keep \ac{CMS} sensitive to types of compressed spectra signatures characterised by low transverse energy jets and small missing transverse energy signatures.

Finally, an approach that uses a template fit method to the \nbreco distribution of \ac{SM} processes within a supersymmetric search is introduced and then validated in simulation and data. The approach can be used to identify any excess in data arising from gluino mediated third generation supersymmetric signatures. It is utilised within this thesis as a crosscheck to the \alphat background prediction at high b-tagged jet multiplicities. This method is found to give a \ac{SM} background estimation that is in good agreement with the \alphat search within the hadronic signal region.

The continued absence of a supersymmetric signal in the \alphat search or other analyses at \ac{CMS} \cite{Chatrchyan:2014lfa}\cite{Chatrchyan:2013fea}\cite{Chatrchyan:2013iqa}, puts pressure on the parameter space in which \ac{SUSY} can reside. Indeed the smoking gun that many theorists and experimentalists hoped to see at the \ac{lhc} has not materialised. Instead identifying a \ac{SUSY} signal may now only result from many years of data taking and the incorporation of increasingly advanced analysis techniques. An unenviable task considering the difficulties of not knowing where \ac{SUSY} may reside, but perhaps solace can be taken in remembering that nothing worth having ever comes easy.

\phantom{}
