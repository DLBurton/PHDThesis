\chapter{Conclusions}
\label{chap:conclusions}

A search for supersymmetry is presented based on a data sample of pp collisions collected at $\sqrt{s} = 8$ \TeV, corresponding to an integrated luminosity of 11.7$\pm$0.5 fb$^{-1}$. Final states with two or more jets and significant \met, a typical final state topology of R-parity conserving \ac{SUSY} models have been analysed. The \alphat variable is utilised as the main discriminator between balanced multi-jet backgrounds and those with real missing energy. 

Additionally a measurement of the performance of the Level-1 trigger for jets and energy sum quantities have also been presented. These studies quantify any change in Level-1 performance after the introduction of a 5 \GeV jet seed threshold into the jet clustering algorithm. This change is introduced to facilitate a reduction in the rate at which jets are formed at Level-1 from pile-up jets which are not of interest to physics analyses. This change is necessary to ensure that trigger thresholds can be maintained at low values, in the presence of an increasing number of pile-up interactions per bunch crossing over the 2012 run period. This is good for susy�.
 No significant change in single jet trigger efficiencies is observed and good performance is observed for a range of Level-1 energy sum quantities.

Within the \ac{SUSY} search presented in this analysis, Standard Model backgrounds are estimated from a simultaneous binned likelihood fit to a hadronic signal selection, as well as three Standard Model process enriched control samples. The search is split into \theht, \nbreco and \njet categories to increase the sensitivity to a range of possible supersymmetric final states. Systematic errors due to theory, detector effects and simulation deficiencies are quantified through the use of data driven closure tests and accounted for in the final interpretation, where observations in data are found to be compatible with a \ac{SM} only hypothesis. 

In the absence of a signal like excess, the analysis is further interpreted in a set of \ac{SMS} models. In models mediated by gluino pair production and with small \ac{LSP} masses and thus large mass splittings, exclusion limits of the gluino mass are set in the range 950-1125 \GeV. For \ac{SMS} models with direct squark pair production, first or second generation squarks are excluded up to around 775 \GeV and bottom squarks excluded up to masses of 600 \GeV, again for small \ac{LSP} masses. In the context of `natural' \ac{SUSY} models, which contain many reconstructed b-jets in the final state, limits are set in the range of 975-1125 \GeV again for large mass splittings between the parent squarks and the \ac{LSP}. This gain is due to�.

A complementary approach, that uses a template fit method to estimate the b-tagged jet distribution of \ac{SM} processes within a supersymmetric search is introduced and then validated in simulation and data. The approach can be used to identify any excess in data arising from third generation supersymmetric signatures, and is utilised within this thesis as a crosscheck of the \alphat background estimation in a high number of reconstructed b-tagged jets. It is found to give a \ac{SM} background estimation that is in good agreement with both the \alphat search and a \ac{SM} only hypothesis. As light third generation squarks are an important feature of `natural' \ac{SUSY} models if they are to ultimately be the solution to the fine tuning problem \cite{Hardy:2013ywa}, the limits imposed through interpretations of the \texttt{T1bbbb} and \texttt{T1tttt} \ac{SMS} models within the \alphat search, put pressure on such theories, by squeezing the parameter space in which light third generation squarks can reside. 

