\chapter{Conclusions}
\label{chap:conclusions}

A search for supersymmetry is presented based on a data sample of pp collisions collected at $\sqrt{s} = 8$ \TeV, corresponding to an integrated luminosity of 11.7$\pm$0.5 fb$^{-1}$. 
Final states with two or more jets and significant \met, a typical final state topology of R-parity conserving \ac{SUSY} models have been analysed and in which the \alphat variable is utilised as the main discriminator between balanced multi-jet backgrounds and those with real missing energy. An additional complementary approach using a template method to estimate the b-tag jet distribution of \ac{SM} processes, to search for gluino induced third generation squark \ac{SUSY} production is also introduced, in which the \alphat search selection is applied in both simulation and data to validate this technique.

 Additionally a measurement the performance of the Level-1 trigger for jets and energy sum quantities is also presented. These studies quantify any change in level-1 performance after the introduction of a 5 \GeV jet seed threshold into the jet algorithm configuration. This change is introduced to facilitate a reduction in the rate at which jets are formed at level-1 from pile-up jets which are not of interest to physics analyses. This change is necessary to ensure that trigger thresholds can be maintained at lower values, in the presence of an increasing number of pile-up interactions per event over the 2012 run period. No significant change in single jet trigger efficiencies is observed and good performance is observed for a range of level-1 quantities.

Within the \ac{SUSY} search presented in this analysis, the sum of standard model backgrounds binned in \theht, \nbreco and \njet categories are estimated from a simultaneous binned likelihood fit to a hadronic signal selection and \mupjets, \dimupjets, and \gpjets control samples. Systematic errors due to theory, detector effects and analysis choices are quantified through the use of data driven closure tests and accounted for in the final interpretation,  where observations in data are found to be compatible with a \ac{SM} only hypothesis. 

In the absence of a signal like excess the analysis is further interpreted in a set of \ac{SMS} models. In the considered models with gluino pair production and for small \ac{LSP} masses, exclusion limits of the gluino mass are in the range 950-1125 \GeV. For \ac{SMS} models with direct squark pair production, first or second generation squarks are excluded up to around 775 \GeV and bottom squarks are excluded up to 600 \GeV, again for small \ac{LSP} masses. In the context of `natural' \ac{SUSY} models, with many reconstructed b-jets in the final state, limits are set in the range of 975-1125 \GeV again for large mass splittings between the parents squark and the \ac{LSP}. 

The template method, whose purpose is to identify any excess in data arising from third generation signatures, finds results that are compatible with the \alphat search and a \ac{SM} only hypothesis at a high number of reconstructed b-jets. As light third generation squarks are an important feature of `natural' \ac{SUSY} models if they are to solve the fine tuning problem \cite{Hardy:2013ywa}, the limits imposed through interpretations in the \texttt{T1bbbb} and \texttt{T1tttt} \ac{SMS} models within the \alphat search, put pressure on such theories, by squeezing the parameter space in which `natural' \ac{SUSY} can reside. 

