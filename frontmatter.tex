%% Title
\titlepage[
Imperial College London \\
Department of Physics]%
{\Large{A thesis submitted to Imperial College London\\
  for the degree of Doctor of Philosophy}}

%% Abstract
\begin{abstract}%[\smaller \thetitle\\ \vspace*{1cm} \smaller {\theauthor}]
  %\thispagestyle{empty}
A search for supersymmetric particles in events with high transverse momentum jets and a large missing transverse energy signature, is conducted using 11.7 fb$^{-1}$ of data, collected with a center-of-mass collision energy of 8 \TeV by the CMS detector. The dimensionless kinematic variable \alphat is used to select events with genuine missing transverse energy signatures. Standard Model backgrounds are estimated through the use of data driven control samples. 
No excess over Standard Model expectations is found. Exclusion limits on squark and gluino masses are set at the 95\% confidence level in the parameter space of a range of supersymmetric simplified model topologies.  \\ 

Results of benchmarking the Level-1 (the first line of the CMS trigger system) single jet and hadronic transverse energy trigger efficiencies, before and after the implementation of a change to the Level-1 jet clustering algorithm are presented. Similar performance is observed for all L1 quantities. This change was introduced to negate an increase in trigger cross-section, which can be attributed to soft jets from secondary interactions. \\

Furthermore, a templated fit method to estimate the Standard Model background distribution of the number of jets originating from a b-quark within a supersymmetric search, is validated in data and simulation. Applicable to searches sensitive to gluino induced third-generation signatures, this technique is utilised as a crosscheck to the results of the \alphat analysis. Standard Model background predictions from the template fits are compared to those from the \alphat search in the hadronic signal region, where good agreement between the two methods is observed. 
\end{abstract}



%% Declaration
\begin{declaration}
  I, the author of this thesis, declare that the work presented within this document to be my own.  The work presented in Chapters \ref{chap:SUSYsearches}, \ref{chap:SUSYresults},  \ref{chap:templatemethod} and Section \ref{sec:triggersystem}, is a result of the author's own work, or that of which I have been a major contributor unless explicitly stated otherwise, and is carried out within the context of the Imperial College London and \CERN \SUSY groups, itself a subsection of the greater \CMS collaboration.  All figures and studies taken from external sources are referenced appropriately throughout this document.
  
  \vspace*{1cm}
  \begin{flushright}
    Darren Burton
  \end{flushright}
\end{declaration}


%% Acknowledgements
\begin{acknowledgements}
  I would like to thank the many people whom I have had the pleasure of working with during the course of the last three and a half years. The opportunity to work as part of the largest scientific collaboration during one of the most exciting times in particle physics for decades, has been a real privilege to be a part of. I could not have achieved the results presented in this thesis without the help of my colleagues who were part of the RA1 team, Edward Laird, Chris Lucas, Henning Flaecher, Yossof Eshaq, Bryn Mathais, Sam Rogerson, Zhaoxia Meng and Georgia Karapostoli whom I worked with on L1 jets. I also thank my supervisor Oliver Buchmuller for his guidance in getting me to this point. 
 
I also feel it important to single out thanks to the postdocs that I have worked with during my PhD. Jad Marrouche from whom I have learnt a great deal and Robert Bainbridge who has been like a second supervisor to me, helping me during my time at Imperial and CERN, especially during those most stressful of times approaching conference deadlines! 

My fellow PhD students who I live with and have seen on an almost daily basis for the last few years, Andrew Gilbert, Patrick Owen, Indrek Sepp, Matthew Kenzie and my wonderful girlfriend Hannah. Thanks for putting up with the whinging, complaining and clomping. 

Finally my largest thanks go to my Mum and Dad whose patience, encouragement and considerable financial support have allowed me to take the many steps that led me here today. 
 
\end{acknowledgements}

\tableofcontents
\listoffigures
\listoftables
%\newpage
%% Strictly optional!
%\frontquote%
  %{The Universe is about 1,000,000 years old.}%
  %{Matthew Kenzie, 1987-present : Discoverer of the Higgs Boson.}
