%% Title
\titlepage[
Imperial College London \\
Department of Physics]%
{\Large{A thesis submitted to Imperial College London\\
  for the degree of Doctor of Philosophy}}

%% Abstract
\begin{abstract}%[\smaller \thetitle\\ \vspace*{1cm} \smaller {\theauthor}]
  %\thispagestyle{empty}
A search for supersymmetric particles is presented, using the Compact Muon Solenoid detector at the Large Hadron Collider, with a signature of missing energy in events with high \pt jets is presented. The analysis is performed with 11.7 fb$^{-1}$ of data, collected at a  center-of-mass energy of 8 \TeV during the 2012 run period. The dimensionless kinematic variable \alphat is used to select events with genuine missing energy signatures, while Standard Model backgrounds in the signal region estimated using data driven control samples.
A complementary method to search for natural SUSY signatures with a high number of b-flavoured jets, through the use of a simple template fit is presented. The \alphat search is used as a vehicle to demonstrate proof of principle and as a search region for this technique. 
Additionally the efficiency of the hadronic Level-1 single jet triggers are measured throughout the 2012 run period. Results are presented with a view to comparing L1 jet performance, before and after, a change to the jet seed algorithm implemented during data taking.
No excess of events is found over Standard Model expectations in the \alphat search. Exclusion limits are set at the 95\% confidence level in the parameter space of simplified models, with special emphasis on compressed spectra and natural SUSY scenarios.  
\end{abstract}


%% Declaration
\begin{declaration}
  I, the author of this thesis, declare that the work presented within this document to be my own.  The work presented in Chapters \ref{chap:SUSYsearches}, \ref{chap:templatemethod}, \ref{chap:SUSYresults} and Section \ref{subsec:l1trigger}, is a result of the author's own work or that of which I have been a major contributor unless explicitly stated otherwise, and is carried out within the context of the Imperial College London and \CERN \SUSY groups, itself a subsection of the greater \CMS collaboration.  All figures and studies taken from external sources are referenced appropriately throughout this document.
  
  \vspace*{1cm}
  \begin{flushright}
    Darren Burton
  \end{flushright}
\end{declaration}


%% Acknowledgements
\begin{acknowledgements}
  Of the many people who deserve thanks, some are particularly prominent....
  Thank Rob
  Thank Jad
 
\end{acknowledgements}


%% Preface
%% \begin{preface}
%%This thesis will never be read by anyone except maybe Baino when he's forced to proof read it.

 % \noindent
  
%\end{preface}

%% ToC
\tableofcontents
\listoffigures
\listoftables
\newpage
\section*{Acronyms}

\begin{acronym}[AAAAAAA]
\acro{ALICE}[ALICE]{A Large Ion Collider Experiment}
\acro {ATLAS} [ATLAS] {A Toroidal LHC ApparatuS}
\acro{APD}[APD]{Avalanche Photo-Diodes}
\acro{BSM}[BSM]{Beyond Standard Model}
\acro {CERN} [CERN] {European Organization for Nuclear Research}
\acro {CMS} [CMS] {Compact Muon Solenoid}
\acro {CMSSM} [CMSSM] {Compressed Minimal SuperSymmetric Model}
\acro {CSC} [CSC] {Cathode Stripe Chamber}
\acro {CSV} [CSV] {Combined Secondary Vertex}
\acro {CSVM} [CSVM] {Combined Secondary Vertex Medium Working Point}
\acro {DT} [DT] {Drift Tube}
\acro {ECAL} [ECAL] {Electromagnetic CALorimeter}
\acro {EB} [EB] {Electromagnetic CALorimeter Barrel}
\acro {EE} [EE] {Electromagnetic CALorimeter Endcap}
\acro {ES} [ES] {Electromagnetic CALorimeter pre-Shower}
\acro {EMG} [EMG] {Exponentially Modified Gaussian}
\acro {EPJC} [EPJC] {European Physical Journal C}
\acro {EWK} [EWK] {Electroweak Sector}
\acro {GCT} [GCT] {Global Calorimeter Trigger}
\acro {GMT} [GMT] {Global MuonTrigger}
\acro {GT} [GT] {Global Trigger}
\acro {HB} [HB] {Hadron Barrel}
\acro {HCAL} [HCAL] {Hadronic CALorimeter}
\acro {HE} [HE] {Hadron Endcaps}
\acro {HF} [HF] {Hadron Forward}
\acro {HLT}[HLT] {Higher Level Trigger}
\acro {HO} [HO] {Hadron Outer}
\acro {HPD} [HPD] {Hybrid Photo Diode}
\acro {ISR} [ISR] {Initial State Radiation}
\acro {LUT} [LUT] {Look Up Table}
\acro {L1} [L1] {Level 1 Trigger}
\acro{lhc}[LHC]{Large Hadron Collider}
\acro{LHCb}[LHCb]{Large Hadron Collider Beauty}
\acro{LSP}[LSP]{Lightest Supersymmetric Partner}
\acro{NLL}[NLL]{Next to Leading Logorithmic Order}
\acro{NLO}[NLO]{Next to Leading Order}
\acro{NNLO}[NNLO]{Next to Next Leading Order}
\acro{POGs}[POGs]{Physics Object Groups}
\acro{PS}[PS]{Proton Synchrotron}
\acro{QED}[QED]{Quantum Electro-Dynamics}
\acro{QCD}[QCD]{Quantum Chromo-Dynamics}
\acro{QFT}[QFT]{Quantum Field Theory}
\acro {RBXs} [RBXs] {Readout Boxes}
\acro {RPC} [RPC] {Resistive Plate Chamber}
\acro {RCT} [RCT] {Regional Calorimeter Trigger}
\acro {RMT} [RMT] {Regional Muon Trigger}
\acro{SUSY}[SUSY]{SUperSYmmetry}
\acro{SM}[SM]{Standard Model}
\acro{SMS}[SMS]{Simplified Model Spectra}
\acro{SPS}[SPS]{Super Proton Synchrotron}
\acro{TF}[TF]{Transfer Factor}
\acro{TP}[TP]{Trigger Primative}
\acro{VEV}[VEV]{Vacuum Expectation Value}
\acro{VPT}[VPT]{Vacuum Photo-Triodes}
\acro{WIMP}[WIMP]{Weakly Interacting Massive Particle}

\end{acronym}

%% Strictly optional!
\frontquote%
  {The Universe is about 1,000,000 years old.}%
  {Matthew Kenzie, 1987-present : Discoverer of the Higgs Boson.}
