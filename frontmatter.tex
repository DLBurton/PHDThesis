%% Title
\titlepage[
Imperial College London \\
Department of Physics]%
{\Large{A thesis submitted to Imperial College London\\
  for the degree of Doctor of Philosophy}}

%% Abstract
\begin{abstract}%[\smaller \thetitle\\ \vspace*{1cm} \smaller {\theauthor}]
  %\thispagestyle{empty}
A search for supersymmetric particles in events with high \pt jets and a large missing energy signature, is conducted using data recorded by the Compact Muon Solenoid detector based at the Large Hadron Collider. The analysis is performed with 11.7 fb$^{-1}$ of data, collected with a center-of-mass collision energy of 8 \TeV during the 2012 run period. The dimensionless kinematic variable \alphat is used as a tool to select events with genuine missing energy signatures, whilst Standard Model backgrounds in the signal region are estimated using data driven control samples, which have similar kinematics to the signal region.
No excess of over Standard Model expectations is found. Exclusion limits are set at the 95\% confidence level in the parameter space of a range of supersymmetric simplified model topologies, with special emphasis on those with compressed spectra (small mass splittings) and natural SUSY scenarios (large number of final state b flavoured jets).  
A complementary method to search for natural SUSY signatures, through the use of a simple template fit is also presented. The event selections of the \alphat search are used as a vehicle to validate the technique in both data and simulation. Estimated Standard Model backgrounds from the template fits are compared with those determined from the data driven background estimation method of the \alphat search within the signal search region, where good agreement between the individual predictions and that of data are observed.
Additionally the efficiency of the hadronic Level-1 single jet triggers are measured throughout the 2012 run period, where a change to the jet seed algorithm was implemented during the data taking period. This change was introduced to negate an increase in rate which can be attributed to pile-up jets, whilst maintaining similar performance in the triggering of physics events.
\end{abstract}


%% Declaration
\begin{declaration}
  I, the author of this thesis, declare that the work presented within this document to be my own.  The work presented in Chapters \ref{chap:SUSYsearches}, \ref{chap:SUSYresults},  \ref{chap:templatemethod} and Section \ref{sec:triggersystem}, is a result of the author's own work, or that of which I have been a major contributor unless explicitly stated otherwise, and is carried out within the context of the Imperial College London and \CERN \SUSY groups, itself a subsection of the greater \CMS collaboration.  All figures and studies taken from external sources are referenced appropriately throughout this document.
  
  \vspace*{1cm}
  \begin{flushright}
    Darren Burton
  \end{flushright}
\end{declaration}


%% Acknowledgements
\begin{acknowledgements}
  I would like to thank the many people whom I have had the pleasure of working with during the course of the last three and a half years. The opportunity to work as part of the largest scientific collaboration during one of the most exciting times in particle physics for decades, has been a real privilege to be a part of. I could not have achieved the results presented in this thesis without the help of my colleagues who were part of the RA1 team, Edward Laird, Chris Lucas, Henning Flaecher, Yossof Eshaq, Bryn Mathais, Sam Rogerson, Zhaoxia Meng and Georgia Karapostoli whom I worked with on L1 jets. I also thank my supervisor Oliver Buchmuller for his guidance in getting me to this point. 
 
I also feel it important to single out thanks to the postdocs that I have worked with during my PhD. Jad Marrouche from whom I have learnt a great deal and Robert Bainbridge who has been like a second supervisor to me, helping me during my time at Imperial and CERN, especially during those most stressful of times approaching conference deadlines! 

My fellow PhD students who I live with and have seen on an almost daily basis for the last few years, Andrew Gilbert, Patrick Owen, Indrek Sepp, Matthew Kenzie and my wonderful girlfriend Hannah. Thanks for putting up with the whinging, complaining and clomping. 

Finally my largest thanks go to my Mum and Dad whose patience, encouragement and considerable financial support have allowed me to take the many steps that led me here today. 
 
\end{acknowledgements}

\tableofcontents
\listoffigures
\listoftables
%\newpage
%% Strictly optional!
%\frontquote%
  %{The Universe is about 1,000,000 years old.}%
  %{Matthew Kenzie, 1987-present : Discoverer of the Higgs Boson.}
