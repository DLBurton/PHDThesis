\chapter{SUSY searches in Hadronic Final States}
\label{chap:SUSYsearches}

In this chapter a model independent search for \ac{SUSY} in hadronic final states with $\met$ using the $\alphat$ variable and b-quark multiplicity is introduced and described in detail. The results presented are based on a data sample of pp collisions collected in 2012 at $\com =$8 \TeV, corresponding to an integrate luminosity of 11.7$\pm$0.5 fb$^{-1}$.

The kinematic variable $\alphat$ is motivated as a variable to provide strong rejections of QCD backgrounds, whilst maintaining sensitivity to possible a \ac{SUSY} signal within Section (\ref{sec:alphatintroduction}). The search and trigger strategy in addition to the event reconstruction and selection are outlined within Sections (\ref{subsec:searchstrategy}-\ref{subsec:eventselection}). 

The method in which the \ac{SM} background is estimated using an analytical technique to improve statistical precision at higher b-tag multiplicities is detailed within Section (\ref{subsec:backgroundestimation}), with a discussion on the impact of b-tagging and mis-tagging scale factors between data and MC on any background predictions. Finally a description of the formulation of appropriate systematic uncertainties applied to the background predictions to account for theoretical uncertainties and limitations in the simulation modelling of event kinematics and instrumental effects is covered in Section (\ref{subsec:sysuncertainties}).


In addition to the $\alphat$ search, a complimentary technique is discussed as a means to predict the distribution of 3 and 4 reconstructed b-quark jets in an event in Section (\ref{sec:templatemethod}). The recent discovery of the Higgs boson has made third-generation ``Natural \ac{SUSY}'' models attractive, given that light top and bottom squarks are a candidate to stabilise divergent loop corrections to the Higgs boson mass.

Using the $\alphat$ search as a base, a simple templated fit is employed to estimate the \ac{SM} background in higher b-tag multiplicities (3-4) from a region of a low number of reconstructed b-jets (0-2). The predictions using this technique are first tested in simulation before being compared to the \ac{SM} background predictions obtained from the $\alphat$ search.  

The experimental reach of the analysis discussed within this thesis is interpreted in two classes of \ac{SMS} models, the topologies of which are detailed in Section (\ref{subsec:sms}). The \ac{SMS} models considered in this analysis are summaries in Table \ref{tab:sms_model_table}. For each model, the \ac{LSP} is assumed to be the lightest neutralino. 

Within Table \ref{tab:sms_model_table} is also defined reference points, parameterised in terms of parent gluino/squark and \ac{LSP} sparticle masses, m$_{parent}$ and m$_{LSP}$, respectively, which are used within the following two chapters to demonstrate potential yields within the signal region of the search. The masses are chosen to reflect parameter space which is within the expect sensitivity reach of the search.

\begin{table}[h!]
\begin{center}
\begin{tabular*}{0.75\textwidth}{@{\extracolsep{\fill}}llcc}
\cline{1-4}
Model & Production/decay mode &  \multicolumn{2}{c}{Reference model}\\ 
&& m$_{parent}$ & m$_{LSP}$ \\  \cline{1-4}
G1 (T1) & $ pp \rightarrow \widetilde{g}\widetilde{g}^{*} \rightarrow q\bar{q}\widetilde{\chi}^{0}_{1}q\bar{q}\widetilde{\chi}^{0}_{1}$ & 700 & 300 \\
G2 (T1bb) & $ pp \rightarrow \widetilde{g}\widetilde{g}^{*} \rightarrow b\bar{b}\widetilde{\chi}^{0}_{1}b\bar{b}\widetilde{\chi}^{0}_{1}$ & 900 & 500 \\
G3 (T1tt) & $ pp \rightarrow \widetilde{g}\widetilde{g}^{*} \rightarrow t\bar{t}\widetilde{\chi}^{0}_{1}t\bar{t}\widetilde{\chi}^{0}_{1}$ & 850 & 250 \\
D1 (T2) & $ pp \rightarrow \widetilde{q}\widetilde{q}^{*} \rightarrow q\widetilde{\chi}^{0}_{1}\bar{q}\widetilde{\chi}^{0}_{1}$ & 600 & 250 \\
D2 (T2bb) & $ pp \rightarrow \widetilde{b}\widetilde{b}^{*} \rightarrow b\widetilde{\chi}^{0}_{1}\bar{b}\widetilde{\chi}^{0}_{1}$ & 500 & 150 \\
D3 (T2tt) & $ pp \rightarrow \widetilde{t}\widetilde{t}^{*} \rightarrow t\widetilde{\chi}^{0}_{1}\bar{t}\widetilde{\chi}^{0}_{1}$ & 400 & 0 \\
\cline{1-4}
\end{tabular*}
\end{center}
\caption[A summary of the \ac{SMS} models interpreted in this analysis, involving both direct (D) and glunio-induced (G) production of squarks and their decays.]{A summary of the \ac{SMS} models interpreted in this analysis, involving both direct (D) and glunio-induced (G) production of squarks and their decays. Reference models are also defined in terms of parent and \ac{LSP} sparticle mass }
\label{tab:sms_model_table}
\end{table}

\section{An introduction to the \alphat search}
\label{sec:alphatintroduction}

The experimental signature of \ac{SUSY} signal in the hadronic channel would manifest as a final state containing energetic jets and $\met$. The search focuses on topologies where new heavy supersymmetric, R-parity conserving particles are pair-produced in pp collisions. These particles decaying to a \ac{LSP} escape the detector undetected, leading to significant missing energy and missing hadronic transverse energy,

\begin{equation}
\mht =  \lvert \sum_{i=1}^{n} p_{T}^{jet_{i}} \rvert,
\end{equation}

defined as the vector sum of the transverse energies of jets selected in an event. Energetic jets produced in the decay of these supersymmetric particles also 
can produce significant visible transverse energy, 

\begin{equation}
\theht = \sum_{i=1}^{n} E_{T}^{jet_{i}},
\end{equation}

defined as the scalar sum of the transverse energies of jets selected in an event.

A search within this channel is greatly complicated in a hadron collider environment, where the overwhelming background comes from inherently balanced multi-jet (``QCD'') events which are produced with an extremely large cross section as demonstrated within Figure \ref{fig:htqcdbackground}. $\met$ can appear in such events with a substantial mis-measurement of jet energy or missed objects due to detector miscalibration or noise effects. 

\begin{figure}[!h]

\centering
\includegraphics[width=0.60\textwidth]{plots/nocuts_htdistribution.pdf}
\caption[Reconstructed offline \theht for 11.7fb$^{-1}$ of data after a basic pre-selection.]{Reconstructed offline $\theht$ for 11.7fb$^{-1}$ of data after a basic pre-selection. Sample is collected from prescaled \theht triggers. Overlaid are expectations from MC simulation of \ac{EWK} processes as well as a reference signal model (labelled D2 from Table.\ref{tab:sms_model_table}).}  
\label{fig:htqcdbackground}
\end{figure}

Additional \ac{SM} background from \ac{EWK} processes with genuine $\met$ from escaping neutrinos comprise the irreducible background within this search and come mainly from:

\begin{itemize}
\item $Z \rightarrow \nu\bar{\nu} +$ jets,
\item $W \rightarrow l\nu$ + jets in which a lepton falls outside of detector acceptance, or the lepton decays hadronically $\tau \rightarrow$ had ,
\item $t\bar{t}$ with at least one leptonic W decay,
\item small background contributions from DY, single top and Diboson (WW,ZZ,WZ) processes.
\end{itemize}

The search is designed to have a strong separation between events with genuine and ``fake'' $\met$ which is achieved primarily though the dimensionless kinematic variable, $\alphat$ \cite{PhysRevLett.101.221803}\cite{CMS:2008vya}.

\subsection{The $\alphat$ variable}
\label{subsec:alphatvariable}

For a perfectly measured di-jet QCD event, conservation laws dictate that they must be produced back-to-back and of equal magnitude. However in di-jet events with real $\met$, both of these jets are produced independently of one another, depicted in Figure \ref{fig:susytopology}.
\begin{figure}[!h]
\centering
\includegraphics[width=0.90\textwidth]{plots/susy_topology.pdf}
\caption[The event topologies of background QCD diet events (right) and a generic \ac{SUSY} signature with genuine $\met$ (left).]{The event topologies of background QCD diet events (right) and a generic \ac{SUSY} signature with genuine $\met$ (left).}  
\label{fig:susytopology}
\end{figure}

 Exploiting this feature leads to the formulation of $\alphat$ in di-jet systems defined as,

\begin{equation}
\alphat = \frac {E^{j2}_{T}}{M_{T}},
\end{equation} 

where $E^{j2}_{T}$ is the transverse energy of the least energetic of the two jets and $M_{T}$ defined as:

\begin{equation}
\label{eq:transmass}
M_{T} = \sqrt{\left(\sum^{2}_{i=1}E^{j_{i}}_{T}\right)^{2}-\left(\sum^{2}_{i=1}p^{j_{i}}_{x}\right)^{2}-\left(\sum^{2}_{i=1}p^{j_{i}}_{y}\right)^{2}} \equiv \sqrt{H_{T}^{2} - \mht^{2}} .
\end{equation}

A perfectly balanced di-jet event i.e. $E_{T}^{j_{1}} = E_{T}^{j_{2}}$ would give an $\alphat = 0.5$, where as events with jets which are not back-to-back, for example in events in which
a W or Z recoils off a system of jets, $\alphat$ can achieve values in excess of 0.5.

$\alphat$ can be extended to apply to any arbitrary number of jets, undertaken by modelling a system of $n$ jets as a di-jet system, through the formation of two pseudo-jets \cite{CMS-PAS-SUS-09-001}. The two pseudo-jets are built by merging the jets present in the event such that the 2 pseudo-jets are chosen to be as balanced as possible, i.e the $\Delta$ \theht $\equiv \lvert E_{T}^{pj_{1}} - E_{T}^{pj_{2}}\rvert$ is minimised between the two pseudo jets. Using Equation (\ref{eq:transmass}), $\alphat$ can be rewritten as,

\begin{equation}
\alphat = \frac{1}{2} \frac {\theht - \Delta\theht}{\sqrt{\theht^{2}-\mht^{2}}}= \frac{1}{2}\frac{1-\Delta\theht/\theht}{\sqrt{1-(\mht/\theht)^{2}}}.
\end{equation}

The distribution of $\alphat$ for the two jet categories used within this analysis, 2,3 and $\geq 4$ jets, is shown in the Figure.\ref{fig:fullalphatdistribution}, demonstrating the ability of the $\alphat$ variable to discriminate between multi jet events and \ac{EWK} processes with genuine $\met$ in the final state.  

\begin{figure}[ht]
\centering
\begin{minipage}[b]{0.48 \linewidth}
\includegraphics[width = 1.0\linewidth,height = 7.0cm]{plots/alphat_low.pdf}
\end{minipage}
\quad
\begin{minipage}[b]{0.48\linewidth}
\includegraphics[width = 1.0\linewidth, height = 7.0cm]{plots/alphat_high.pdf}
\end{minipage}
\caption[ The $\alphat$ distributions for the low 2-3 (left) and high $\geq 4$ (right) jet multiplicities after a full analysis selection and shown for $\theht > 375$.]{The $\alphat$ distributions for the low 2-3 (left) and high $\geq 4$ (right) jet multiplicities after a full analysis selection and shown for $\theht > 375$ . Data is collected using both prescaled $\theht$ triggers and dedicated $\alphat$ triggers for below and above $\alphat = 0.55$ respectively. . Expected yields as given by simulation are also shown for multijet events (green dash-dotted line), \ac{EWK} backgrounds with genuine $\met$ (blue long-dashed line), the sum of all \ac{SM} processes (cyan solid line) and the reference signal model D2 (left, red dotted line) or G2 (right, red dotted line). }
\label{fig:fullalphatdistribution}
\end{figure}

The $\alphat$ requirement used within the search is chosen to be $\alphat >$ 0.55 to ensure that the QCD multijet background is negligible even in the presence of moderate jet mis-measurement. There still remains other effects which can cause multijet events to artificially have a large $\alphat$ value, which are discussed in detail in Section (\ref{subsec:eventselection}).  


\section{Search Strategy}
\label{subsec:searchstrategy}

The aim of the analysis presented in this thesis is to identify an excess of events in data over the \ac{SM} background expectation in multi-jet final states and significant $\met$. The essential suppression of the dominant QCD background for such a search is addressed by the $\alphat$ variable described in the previous section. For estimation of the remaining \ac{EWK} backgrounds, three independent data control samples are used to predict the different processes that compose the background :

\begin{itemize}
\item \mupjets to determine W + jets, \ttbar and single top backgrounds,
\item \gpjets  to determine the irreducible \zinv + jets background,
\item \dimupjets to determine the irreducible \zinv + jets background.
\end{itemize}

These control samples are chosen to both be rich in specific \ac{EWK} processes, be free of QCD multi-jet events and to also be kinematically similar to the hadronic signal region that they are estimating the backgrounds of, see Section (\ref{subsec:controlsampledefinition}).

To remain inclusive to a large range of possible \ac{SUSY} models, the signal region is binned in the following categories to allow for increased sensitivity in the interpretation of results for different \ac{SUSY} topologies:

\begin{itemize}

\item[] \textbf{Sensitivity to a range of \ac{SUSY} mass splittings}

The hadronic signal region is defined by \theht $>$ 275, divided into eight bins in \theht. 

\begin{itemize}
\item Two bins of width 50 \GeV in the range 275 $<$ \theht $<$ 375 \GeV,
\item five bins of width 100 \GeV in the range 375 $<$ \theht$<$ 875 \GeV,
\item and a final open bin, \theht $>$ 875 \GeV.
\end{itemize}

The choice at low \theht is driven primarily by trigger constraints. The mass difference between the \ac{LSP} and the particle that it decays from is an important factor in the amount of hadronic activity in the event. 

A large mass splitting will lead to hard high \pt jets which contribute to the \theht sum. From Figure \ref{fig:htqcdbackground} it can be seen that the \ac{SM} background falls sharply at high \theht values, therefore a large number of \theht bins will lead to easier of identification of such signals. Conversely smaller mass splittings lead to softer jet \pt's which will subsequently fall into the lower \theht range.

\item[] \textbf{Sensitivity to production method of \ac{SUSY} particles}

The production mechanism of any potential \ac{SUSY} signal can lead to different event topologies. One such way to discriminate between gluino ($g\widetilde{g}$ - ``high multiplicity''), and direct squark ($q\widetilde{q}$ - ``low multiplicity'') induced production of \ac{SUSY} particles is realised through the number of reconstructed jets in the final state.  

The analysis is thus split into two jet categories : 2-3 jets , $\geq$ 4 jets to give sensitivity to both of these mechanisms. 

\item[] \textbf{Sensitivity to  ``Natural \ac{SUSY}'' via tagging jets from b-quarks}

Jets originating from bottom quarks (b-jets) are identified through vertices that are displaced with respect to the primary interaction. The algorithm used to tag b-jets is the \acf{CSVM} tagger, described within Section (\ref{subsec:cmsobjects-btagging}). A cut is placed on the discriminator variable of $> 0.679$, leading to a gluon/light-quark mis-tag rate of 1\% and a jet p$_{\text{T}}$ dependant b-tagging efficiency of 60-70\% \cite{btag8tev}.

Natural \ac{SUSY} models would be characterised through final-state signatures rich in bottom quarks. A search relying on methods to identify jets originating from bottom quarks through b-tagging, will significantly improve the sensitivity to this class of signature. 

This is achieved via the binning of events in the signal region according to the number of b-tagged jets reconstructed in each event, in the following: 0,1,2,3,$\geq$ 4 b-tag categories . In the highest $\geq$ 4 b-tag category due to a limited number of expected signal and background, just three \theht bins are employed: 275-325 \GeV, 325-375 \GeV, $\geq$ 375 \GeV.

This characterisation is identically mirrored in all control samples, with the information from all samples and b-tag categories used simultaneously in the likelihood model (see Chapter \ref{chap:SUSYresults}) in order to interpret the results in a coherent and powerful way.

\end{itemize}
 
 The combination of the \theht, jet multiplicity and b-tag categorisation of the signal region as described above, resultantly leads to 67 different bins in which the analysis is interpreted in, which is depicted in Figure \ref{fig:analysisbinning}. 
 
 \begin{figure}[!h]
 \centering
\includegraphics[width=0.80\textwidth]{plots/analysis_binning.pdf}
\caption[Pictorial depiction of the analysis strategy employed by the $\alphat$ search to increase sensitivity to a wide spectra of \ac{SUSY} models.]{Pictorial depiction of the analysis strategy employed by the $\alphat$ search to increase sensitivity to a wide spectra of \ac{SUSY} models.}  
\label{fig:analysisbinning}
\end{figure}


\subsection{Control Sample Definition}
\label{subsec:controlsampledefinition}

The method used to estimate these background contributions in the hadronic signal region relies on the use of a \acf{TF}. This is determined from MC simulation in both the control and signal region to transform the observed yield measured in data for a control sample into a background prediction. 

The control samples and the \ac{EWK} processes they are specifically tuned to select are defined as: 

\begin{itemize} 

\item[] \textbf{The \mupjets control sample}

Events from W + jets and \ttbar

\item[] \textbf{The \dimupjets control sample}

The �

\item[] \textbf{The \gpjets control sample}

The ...
\end{itemize}

\section{Trigger Strategy}
\label{subsec:triggerstrategy}

\section{Event Selection}
\label{subsec:eventselection}


\section{A method to determine MC yields with higher statistical precision}
\label{subsec:backgroundestimation}


\section{Systematic Uncertainties on Transfer Factors}
\label{subsec:sysuncertainties}

\section{Searches for Natural SUSY with B-tag templates.}
\label{sec:templatemethod}

Btag Templates blah blah

