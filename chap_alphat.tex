\chapter{Searches for SUSY in Hadronic Final States at the LHC}
\label{chap:SUSYsearches}

In this chapter a model independent search for \ac{SUSY} in hadronic final states with $\met$ using the $\alphat$ variable and b-quark multiplicity is introduced and described in detail. The results presented are based on a data sample of pp collisions collected in 2012 at $\com =$8 \TeV, corresponding to an integrate luminosity of 11.7$\pm$0.5 fb$^{-1}$.

The kinematic variable $\alphat$ is motivated as a variable to provide strong rejections of QCD backgrounds, whilst maintaining sensitivity to possible a \ac{SUSY} signal within Section (\ref{sec:alphatintroduction}). The search and trigger strategy in addition to the event reconstruction and selection are outlined within Sections (\ref{subsec:searchstrategy}-\ref{subsec:eventselection}). 

The method in which the \ac{SM} background is estimated using an analytical technique to improve statistical precision at higher b-tag multiplicities is detailed within Section (\ref{subsec:backgroundestimation}), with a discussion on the impact of b-tagging and mis-tagging scale factors between data and MC on any background predictions. Finally a description of the formulation of appropriate systematic uncertainties applied to the background predictions to account for theoretical uncertainties and limitations in the simulation modelling of event kinematics and instrumental effects is covered in Section (\ref{subsec:sysuncertainties}).


In addition to the $\alphat$ search, a complimentary technique is discussed as a means to predict the distribution of 3 and 4 reconstructed b-quark jets in an event in Section (\ref{sec:templatemethod}). The recent discovery of the Higgs boson has made third-generation ``Natural \ac{SUSY}'' models attractive, given that light top and bottom squarks are a candidate to stabilise divergent loop corrections to the Higgs boson mass.

Using the $\alphat$ search as a base, a simple templated fit is employed to estimate the \ac{SM} background in higher b-tag multiplicities (3-4) from a region of a low number of reconstructed b-jets (0-2). The predictions using this technique are first tested in simulation before being compared to the \ac{SM} background predictions obtained from the $\alphat$ search.  

\section{The \alphat search}
\label{sec:alphatintroduction}

The experimental signature of \ac{SUSY} signal in the hadronic channel would manifest as a final state containing energetic jets and $\met$. The search focuses on topologies where new heavy supersymmetric, R-parity conserving particles are pair-produced in pp collisions. These particles decaying to a \ac{LSP} escape the detector undetected, leading to significant missing energy. 

A search within this channel is greatly complicated in a hadron collider environment, where the overwhelming background comes from inherently balanced multi-jet (``QCD'') events which are produced with an extremely large cross section. $\met$ can appear in such events with a substation mis-measurement of jet energy or missed objects due to detector miscalibration or noise effects. 

Additional \ac{SM} background contribution comes from \ac{EWK} processes with genuine $\met$�. 

\subsection{Search Strategy}
\label{subsec:searchstrategy}

\subsection{Trigger Strategy}
\label{subsec:triggerstrategy}

\subsection{Event Selection}
\label{subsec:eventselection}


\subsection{Background Estimation}
\label{subsec:backgroundestimation}


\subsection{Systematic Uncertainties on Transfer Factors}
\label{subsec:sysuncertainties}

\section{Searches for Natural SUSY with B-tag templates.}
\label{sec:templatemethod}

Btag Templates blah blah

