\documentclass{mythesis}
\usepackage{mythesis}


%% You can set the line spacing this way
%\setallspacing{double}
%% or a section at a time like this
%\setfrontmatterspacing{double}

%% PDF metadata
\makeatletter

\@ifpackageloaded{hyperref}{%
\hypersetup{%
pdftitle = {Searches for Supersymmetry using the \alphat variable.},
pdfsubject = {Darren Burton's PhD thesis},
pdfkeywords = {CMS, SUSY, physics, LHC },
pdfauthor = {\textcopyright\ Darren Burton}
}
}{}
\makeatother

%% Define the thesis title and author
\title{\LARGE{Searches for Supersymmetry using the \alphat variable with the CMS detector at the LHC}}
\author{Darren Burton \vspace*{0.5cm}}

%% Start the document
\begin{document}
\runninglinenumbers

%% Define the un-numbered front matter (cover pages, rubrik and table of contents)
\begin{frontmatter}
  %% Title
\titlepage[
Imperial College London \\
Department of Physics]%
{\Large{A thesis submitted to Imperial College London\\
  for the degree of Doctor of Philosophy}}

%% Abstract
\begin{abstract}%[\smaller \thetitle\\ \vspace*{1cm} \smaller {\theauthor}]
  %\thispagestyle{empty}
A search for supersymmetric particles in events with high transverse momentum jets and a large missing transverse energy signature, is conducted using 11.7 fb$^{-1}$ of data, collected with a center-of-mass collision energy of 8 \TeV by the CMS detector. The dimensionless kinematic variable \alphat is used to select events with genuine missing transverse energy signatures. Standard Model backgrounds are estimated through the use of data driven control samples. 
No excess over Standard Model expectations is found. Exclusion limits on squark and gluino masses are set at the 95\% confidence level in the parameter space of a range of supersymmetric simplified model topologies.  \\ 

Results of benchmarking the Level-1 (the first line of the CMS trigger system) single jet and hadronic transverse energy trigger efficiencies, before and after the implementation of a change to the Level-1 jet clustering algorithm are presented. Similar performance is observed for all L1 quantities. This change was introduced to negate an increase in trigger cross-section, which can be attributed to soft jets from secondary interactions. \\

Furthermore, a templated fit method to estimate the Standard Model background distribution of the number of jets originating from a b-quark within a supersymmetric search, is validated in data and simulation. Applicable to searches sensitive to gluino induced third-generation signatures, this technique is utilised as a crosscheck to the results of the \alphat analysis. Standard Model background predictions from the template fits are compared to those from the \alphat search in the hadronic signal region, where good agreement between the two methods is observed. 
\end{abstract}



%% Declaration
\begin{declaration}
  I, the author of this thesis, declare that the work presented within this document to be my own.  The work presented in Chapters \ref{chap:SUSYsearches}, \ref{chap:SUSYresults},  \ref{chap:templatemethod} and Section \ref{sec:triggersystem}, is a result of the author's own work, or that of which I have been a major contributor unless explicitly stated otherwise, and is carried out within the context of the Imperial College London and \CERN \SUSY groups, itself a subsection of the greater \CMS collaboration.  All figures and studies taken from external sources are referenced appropriately throughout this document.
  
  \vspace*{1cm}
  \begin{flushright}
    Darren Burton
  \end{flushright}
\end{declaration}


%% Acknowledgements
\begin{acknowledgements}
  I would like to thank the many people whom I have had the pleasure of working with during the course of the last three and a half years. The opportunity to work as part of the largest scientific collaboration during one of the most exciting times in particle physics for decades, has been a real privilege to be a part of. I could not have achieved the results presented in this thesis without the help of my colleagues who were part of the RA1 team, Edward Laird, Chris Lucas, Henning Flaecher, Yossof Eshaq, Bryn Mathais, Sam Rogerson, Zhaoxia Meng and Georgia Karapostoli whom I worked with on L1 jets. I also thank my supervisor Oliver Buchmuller for his guidance in getting me to this point. 
 
I also feel it important to single out thanks to the postdocs that I have worked with during my PhD. Jad Marrouche from whom I have learnt a great deal and Robert Bainbridge who has been like a second supervisor to me, helping me during my time at Imperial and CERN, especially during those most stressful of times approaching conference deadlines! 

My fellow PhD students who I live with and have seen on an almost daily basis for the last few years, Andrew Gilbert, Patrick Owen, Indrek Sepp, Matthew Kenzie and my wonderful girlfriend Hannah. Thanks for putting up with the whinging, complaining and clomping. 

Finally my largest thanks go to my Mum and Dad whose patience, encouragement and considerable financial support have allowed me to take the many steps that led me here today. 
 
\end{acknowledgements}

\tableofcontents
\listoffigures
\listoftables
%\newpage
%% Strictly optional!
%\frontquote%
  %{The Universe is about 1,000,000 years old.}%
  %{Matthew Kenzie, 1987-present : Discoverer of the Higgs Boson.}

\end{frontmatter}

%% Start the content body of the thesis
\begin{mainmatter}
  %% Actually, more semantic chapter filenames are better, like "chap-bgtheory.tex"
  \chapter{Introduction}
\label{chap:introduction}

Introduce the thesis  ~\cite{Amato:1998xt}

  \chapter{A Theoretical Overview}
\label{chap:theorysection}

Within this chapter, a brief introduction and background to the \ac{SM} is given. Its success as a rigorously tested and widely accepted theory is discussed as well as the deficiencies with this theory that hint there this theory is not a complete description of our universe. The motivations for new physics at the \TeV scale and in particular Supersymmetric theories are outlined within Section \ref{sec:susytheory}, with the chapter concluding with how an experimental signature of such theories can be produced and observed at the \ac{lhc}, Section \ref{sec:susysearches}.

\section{The Standard Model}

\label{sec:thesm}

The \ac{SM} is the name given to the relativistic  \acf{QFT}, where particles are represented as excitations of fields, which describes the interactions and properties of all the known elementary particles \cite{PhysRevLett.19.1264}\cite{Glashow:1961tr}\cite{Salam:1968rm}\cite{Hooft1971167}. It is a renormalisable field theory which contains three symmetries: $SU(3)$ for colour charge, SU(2) for weak isospin and U(1) relating to weak hyper charge, which require its Lagrangian \lsm to be invariant under local gauge transformation. 

Within the \ac{SM} theory, matter is composed of spin \half fermions, which interact with each other via the exchange of spin-1 gauge bosons. A summary of the known fundamental fermions and bosons is given in Table \ref{tab:sm_particles}.

\begin{table}[h!]
\begin{center}
\begin{tabular*}{0.75\textwidth}{@{\extracolsep{\fill}}|c|c|c|c|c|}
\cline{1-5}
Particle                   & Symbol      & Spin & Charge         & Mass (\GeV) \\ \cline {1-5}
\multicolumn{5}{|c|}{\textbf{First Generation Fermions}}  					 \\ \cline{1-5}
Electron Neutrino &$\nu_{e}$   & \half & 0                     &   $< 2.2 \times 10^{-6}$ \\ \cline{1-5}
Electron                  & e                 & \half & -1                    &   $0.51 \times 10^{-3}$ \\ \cline{1-5}
Up Quark                & u                 & \half & $\frac{2}{3}$ &   $2.3 ^{+0.7}_{-0.5} \times 10^{-3}$ \\ \cline{1-5}
Down Quark          & d                 & \half & $-\frac{1}{3}$&   $4.8 ^{+0.7}_{-0.3}\times 10^{-3}$ \\ \cline{1-5}
\multicolumn{5}{|c|}{\textbf{Second Generation Fermions}}  					 \\ \cline{1-5}
Muon Neutrino       &$\nu_{\mu}$& \half & 0                     &   -                           \\ \cline{1-5}
Muon                       & $\mu$       & \half & -1                    &   $1.05 \times 10^{-3}$ \\ \cline{1-5}
Charm Quark        & c                 & \half & $\frac{2}{3}$ &   $1.275 \pm 0.025$ \\ \cline{1-5}
Strange Quark      & s                 & \half & $-\frac{1}{3}$&   $95 \pm 5 \times 10^{-3}$ \\ \cline{1-5}
\multicolumn{5}{|c|}{\textbf{Third Generation Fermions}}  					 \\ \cline{1-5}
Tau Neutrino         &$\nu_{\tau}$& \half & 0                     &   -                           \\ \cline{1-5}
Tau                          & $\tau$       & \half & -1                    &   1.77		 	\\ \cline{1-5}
Top Quark              & t                 & \half & $\frac{2}{3}$ &   $173.5  \pm 0.8$ \\ \cline{1-5}
Bottom Quark        & b                 & \half & $-\frac{1}{3}$&   $4.65  \pm 0.03$ \\ \cline{1-5}
\multicolumn{5}{|c|}{\textbf{Gauge Bosons}}  					 \\ \cline{1-5}
Photon                    &$\gamma$&1   & 0                         &   0                           \\ \cline{1-5}
W Boson                 & $\Wboson$&1& $\pm$1              &   80.385 $\pm 0.015$ \\ \cline{1-5}
Z Boson                 & $\Zboson$& 1 & 0		          &   $91.187  \pm 0.002$ \\ \cline{1-5}
Gluons                   & g                 & 1 & 0			 &   0 \\ \cline{1-5}
Higgs Boson         & H                & 0 & 0			 &   125.3 $\pm 0.5$ \cite{Chatrchyan:2012ufa} \\ \cline{1-5}
\end{tabular*}
\end{center}
\caption[The fundamental particles of the \ac{SM}, with spin, charge and mass displayed.]{The fundamental particles of the \ac{SM}, with spin, charge and mass displayed. Latest mass measurements taken from \cite{pdg2012}. }
\label{tab:sm_particles}

\end{table}

Fermions are separated into quarks and leptons of which only quarks interact with the strong nuclear force. Quarks unlike leptons are not seen as free particles in nature, but rather exist only within baryons, composed of three quarks with an overall integer charge, and quark-anti-quark pairs called mesons. Both leptons and quarks are grouped into three generations which have the same properties, but with ascending mass in each subsequent generation. 

The gauge bosons mediate the interactions between fermions. The field theories of \acf{QED} and \acf{QCD}, yield massless mediator bosons, the photon and eight coloured gluons which are consequences of the gauge invariance of those theories, detailed in Section \ref{subsec:gaugetheories}.  

The unification of the electromagnetic and weak-nuclear forces into the current Electroweak theory yield the weak gauge bosons, \Wboson and \Zboson through the mixing of the associated gauge fields. The force carriers of this theory were experimentally detected by the observation of weak neutral current, discovered in 1973 in the Gargamelle bubble chamber located at \ac{CERN} \cite{Hasert:1973ff}, with the masses of and the weak gauge bosons measured by the UA1 and U2 experiments at the \acf{SPS} collider in 1983 \cite{Arnison:1983mk}\cite{Banner:1983jy}.

\subsection{Gauge Symmetries of the SM}

\label{subsec:gaugetheories}

Symmetries are of fundamental importance in the description of physical phenomena. Noether's theorem states that for a dynamical system, the consequence of any symmetry is an associated conserved quantity \cite{Noether1918}. Invariance under translations, rotations, and Lorentz transformations in physical systems lead to conservation of momentum, energy and angular momentum. 

In the \ac{SM}, a quantum theory described by Lagrangian formalism, the weak,strong and electromagnetic interactions are described in terms of ``gauge theories''.  A gauge theory possesses invariance under a set of ``local transformations", which are transformations whose parameters are space-time dependent. The requirement of gauge invariance within the \ac{SM} necessitates the introduction of force-mediating gauge bosons and interactions between fermions and the bosons themselves. Given the nature of the topics covered by this thesis, the formulation of \ac{EWK}  within the \ac{SM} Lagrangian is reviewed within this section.

The simplest example of the application of the principle of local gauge invariance within the \ac{SM} is in \acf{QED}, the consequences of which require a massless photon field \cite{quarksandleptons}\cite{introtoparticles}. 

Starting from the free Dirac Lagrangian written as 

\begin{equation}
\label{eq:diracequation}
\mathcal{L} = \bar{\psi}(i\gamma^{\mu}\partial_{\mu} - m)\psi,
\end{equation}

where \fermfield represents a free non interacting fermionic field, with the matrices $\gamma^{\mu}$,$\mu \in  0,1,2,3$ defined by the anti commutator relationship $\gamma^{\mu}\gamma^{\nu}  + \gamma^{\mu}\gamma^{\nu} = 2\eta^{\mu\nu}I_{4}$ where $\eta^{\mu\nu}$ is the flat space-time metric $(+,-,-,-)$ and $I_{4}$ is the 4 $\times$ 4 identity matrix. 

Under a local U(1) abelian gauge transformation in which \fermfield transforms as:

\begin{equation}
\fermfield(x) \rightarrow \fermfield^{'}(x) = e^{i\theta(x)}\fermfield(x) \qquad      \bar{\fermfield}(x) \rightarrow \bar{\fermfield}^{'}(x) = e^{i\theta(x)}\bar{\fermfield}(x)
\end{equation}

the kinetic term of the Lagrangian does not remain invariant, due to the partial derivative interposed between the $\bar{\fermfield}$ and \fermfield yielding,

\begin{equation}
\label{eq:remainderterm}
\partial_{\mu}\fermfield \rightarrow  e^{i\theta(x)}\partial_{\mu}\fermfield + ie^{i\theta(x)}\fermfield\partial_{\mu}\theta.
\end{equation} 

To ensure that $\mathcal{L}$ remains invariant, a modified derivative, $D_{\mu}$, that transforms covariantly under phase transformations is introduced. In doing this a vector field $A_{\mu}$ with transformation properties that cancel out the unwanted term in (\ref{eq:remainderterm}) must also be included, 

\begin{equation}
\label{eq:afieldtrans}
D_{\mu} \equiv \partial_{\mu} - ieA_{\mu},   \qquad A_{\mu} \rightarrow A_{\mu} + \frac{1}{e}\partial_{\mu}\theta .
\end{equation}

Invariance of the Lagrangian is then achieved by replacing $\partial_{\mu}$ by $D_{\mu}$:

\begin{align}
\label{eq:covlag}
\mathcal{L} &= i\bar{\fermfield}\gamma^{\mu}D_{\mu}\fermfield - m\bar{\fermfield}\fermfield \nonumber    \\
 &= \bar{\fermfield}(i\gamma^{\mu}\partial_{\mu} - m)\fermfield +  e\bar{\fermfield}\gamma^{\mu}\fermfield A_{\mu}
\end{align}


An additional interaction term is now present in the Lagrangian, coupling the Dirac particle to this vector field, which is interpreted as the photon in \ac{QED}. To regard this new field as the physical photon field, a term corresponding to its kinetic energy must be added to the Lagrangian in (\ref{eq:covlag}). Since this term must also be invariant under (\ref{eq:afieldtrans}), it is defined in the form $F_{\mu\nu} = \partial^{\mu}A^{\nu} - \partial_{\nu}A{\mu}$. 

This then leads to the Lagrangian of \ac{QED}: 

\begin{equation}
\label{qedlagrangian}
\mathcal{L}_{QED} =.\overbrace{i\bar{\fermfield}\gamma^{\mu}\partial_{\mu}\fermfield - \frac{1}{4}F_{\mu\nu}F^{\mu\nu} }^\text{kinetic term} + \overbrace{m\bar{\fermfield}\fermfield}^\text{mass term} + \overbrace{e\bar{\fermfield}\gamma^{\mu}\fermfield A_{\mu} }^\text{interaction term}
\end{equation}

Within the Lagrangian there remains no mass term of the form $m^{2}A_{\mu}A^{\mu}$, which is prohibited by gauge invariance. This implies that the gauge particle, the photon, must be massless.


\subsection{The Electroweak Sector and Electroweak Symmetry Breaking}

\label{subsec:ewsb}
The same application of gauge symmetry and the requirement of local gauge invariance can be used to unify \ac{QED} and the Weak force in the \acf{EWK}. The nature of \ac{EWK} interactions is encompassed within a Lagrangian invariant under transformations of the group $SU(2)_{L} \times U(1)_{Y}$. 

The weak interactions from experimental observation \cite{wu-parity}, are known to violate parity and are therefore not symmetric under interchange of left and right helicity fermions. Thus within the \ac{SM} the left and right handed parts of these fermion fields are treated separately. A fermion field is then split into two left and right handed chiral components, $\fermfield = \fermfield_{L} + \fermfield_{R}$, where $\fermfield_{L\slash R} = (1 \pm \gamma^{5})\fermfield$. 

The $SU(2)_{L}$ group is the special unitary group of $2 \times 2$ matrices $U$ satisfying $UU^{\dagger} = I$ and $det(U) = 1$. It may be written in the form $U = e^{-i\omega_{i}T_{i}}$, with the generators of the group $T_{i} = \frac{1}{2}\tau_{i}$ where $\tau_{i}$, $i \in$ 1,2,3 being the $2 \times 2$ Pauli matrices

\begin{equation}
\tau_{1}= \left( \begin{array}{ccc}
0 & 1\\
1 & 0\end{array} \right)\qquad
\tau_{2}= \left( \begin{array}{ccc}
0 & -i\\
i & 0\end{array} \right)\qquad
\tau_{3}= \left( \begin{array}{ccc}
1 & 0\\
0 & -1\end{array} \right),
\end{equation}


which form a non Abelian group obeying the commutation relation $[T^{a},T^{b}] \equiv  if^{abc}T^{c} \neq 0$. The gauge fields that accompany this group are represented by $\hat{W}_{\mu}  = (\hat{W}^{1}_{\mu},\hat{W}^{2}_{\mu},\hat{W}^{3}_{\mu}$  ) and act only on the left handed component of the fermion field $\fermfield_{L}$.

One additional generator $Y$ which represents the hypercharge of the particle under consideration is introduced through the $U(1)_{Y}$ group acting on both components of the fermion field, with an associated vector boson field $\hat{B}_{\mu}$.  

The $SU(2)_{L} \times U(1)_{Y}$ transformations of the left and right handed components of $\fermfield$ are summarised by,

\begin{align}
\label{eq:su2xu1transform}
 & \chi_{L} \rightarrow \chi^{'}_{L} = e^{i\omega(x) \cdot T + i\theta(x)Y}\chi_{L}, \nonumber \\
 & \fermfield_{R} \rightarrow \fermfield^{'}_{R} = e^{i\theta(x)Y}\fermfield_{R}, 
\end{align}

where the left handed fermions form isospin doubles $\chi_{L}$ and the right handed fermions are isosinglets $\fermfield_{R}$. For the first generation of leptons and quarks this represents

\begin{align}
\label{firstdoublet}
 \chi_{L} &= \left( \begin{array}{c} 
\nu_{e} \\
e \end{array} \right)_{L}, \qquad
 \left( \begin{array}{c} 
u  \\
d \end{array} \right)_{L} \nonumber \\
 \fermfield &= e_{R}, \qquad \qquad u_{R}, d_{R}
\end{align}

Imposing local gauge invariance within $\mathcal{L}_{EWK}$ is once again achieved by modifying the covariant derivative

\begin{equation}
D_{\mu} = \partial_{\mu} - \frac{ig}{2}\tau^{i}W^{i}_{\mu} - \frac{ig^{'}}{2}Y B_{\mu},
\end{equation}

where $g$ and $g^{'}$ are the coupling constant of the $SU(2)_{L}$ and $U(1)_{Y}$ groups respectively. Taking the example of the first generation of fermions defined in Eq.\ref{firstdoublet}, this would lead to a lagrangian $\mathcal{L}_{1}$ of the form,

\begin{equation}
\mathcal{L}_{1} =  . 
\end{equation}

As in \ac{QED}, these additional gauge fields introduce field strength tensors $B_{\mu\nu}$ and $W^{\theta}_{\mu\nu}$,

\begin{align}
\hat{B}_{\mu\nu} &=  \partial_{\mu}\hat{B}_{\nu} - \partial_{\nu}\hat{B}_{\mu} \\
\hat{W}^{\theta}_{\mu\nu} &=  \partial+{\mu}\hat{W}+{\nu} - \partial_{\nu}\hat{W}_{\mu} - g\hat{W}_{\mu}\times\hat{W}_{\mu}
\end{align}

corresponding to the kinetic energy and self coupling of the $W_{\mu}$ fields and the kinetic energy term of the $B_{\mu}$ field.

None of these gauge bosons are physical particles, and instead linear combinations of these gauge bosons make up $\gamma$ and the W and Z bosons, defined as 

\begin{equation}
W^{\pm} = \frac{1}{\sqrt{2}} \left(W^{1}_{\mu} \mp iW^{2}_{\mu}\right) \qquad   
\left( \begin{array}{c} 
Z_{\mu} \\
A_{\mu} \end{array} \right)   = 
\left( \begin{array}{cc} 
cos\theta_{W} & -sin\theta_{W} \\
sin\theta_{W} & cos\theta_{W} \end{array} \right) 
\left( \begin{array}{c} 
W^{3}_{\mu} \\
B_{\mu} \end{array} \right)
\end{equation}

where the mixing angle, $\theta{w} = \tan ^{-1} \frac{g^{'}}{g}$ , relates the coupling of the neutral weak and electromagnetic interactions. There remains 

As in the case of the formulation of the \ac{QED} Lagrangian there remains no mass term for the photon. However this is also the case for the W, Z and fermions in the, contrary to experimental measurement. Any explicit introduction of mass terms would break the symmetry of the Lagrangian and instead mass terms can be introduced through spontaneous breaking of the \ac{EWK} symmetry via the Higgs mechanism.

Talk about higgs sector


\section{Motivation for Beyond the Standard Model Physics}

\label{sec:bsmmotivation}

As previously described, the \ac{SM} is a successful theory, predicting the existence of the \Wboson and \Zboson bosons and the top quark long before they were experimentally observed. However the theory does not fully explain�.


\section{Supersymmmetry}

\label{sec:susytheory}

What is this theory that doesn't exist all about?

\subsection{R-Parity}

\label{subsec:rparity}

R-Parity stuff here innit.

\subsection{ Supersymmetry Breaking}

\label{subsec:susybreaking}

Why are supersymmetric particles not the same mass.


\section{Searching for SUSY at the LHC}
\label{sec:susysearches}

\subsection{Simplified Models}

\label{subsec:sms}

With such a variety of different way for a \ac{SUSY} signal to manifest itself, it is necessary to be able to interpret experimental reach through the masses of gluinos and squarks which can excluded by experimental searches rather than on a model specific basis. This is accomplished through \acf{SMS} models, which are \ac{SUSY} decays which contain only one process. For example the production of a pair of gluinos which then decay via \ac{SM} processes, to a set decay topology with a 100$\%$ branching ratio, shown in �.

Searching and interpreting \ac{SUSY} searches in this way�.

The convention for the naming of these \ac{SMS} models is via the prefix...

  \chapter{The LHC and the CMS Detector}
\label{chap:cmsoverview}

Probing the \SM for signs of new physics would not be possible without the immensely complex electronics and machinery that makes the TeV energy scale accessible for the first time. This chapter will cover \CERN 's  Large Hadron Collider (\LHC) and the CMS detector, being the experiment the author is a member of. Section \ref{sec:cmsdetector} serves to introduce an overview of the different components of  the CMS detector, with more detail spent on those that are relevant in the search for Supersymmetric particles. Section \ref{sec:cmsobjects} will focus on event and object reconstruction again with more emphasis on jet level quantities which are most relevant to the author's analysis research. Finally Section \ref{sec:l1trigger} will cover work performed by the author, as service to the CMS Collaboration, in measuring the performance of the GCT component of the L1 trigger during the 2012-2013 run period.  


\section{The LHC}
\label{sec:thelhc} 

The \LHC is a storage ring, accelerator, and collider of circulating beams of protons or ions. Housed in the tunnel dug for the Large Electron-Positron collider (LEP), it is approximately 27 km in circumference, 100 m underground, and straddles the border between France and Switzerland outside of Geneva. It is currently the only collider in operation that is able to study physics at the TeV scale.  A double-ring circular synchrotron, it was
designed to collide both proton-proton (pp) and heavy ion (PbPb) with a centre of mass energy $\sqrt{s} = $ 14 \TeV at a final design luminosity of $10^{34}$cm$^{-2}$s$^{-1}$. \\

These counter-circulating beams of protons/Pb ions are merged in four sections around the ring to enable collisions of the beams, with each interaction point being home to one of the four major experiments; ALICE \cite{alicetdr} , ATLAS \cite{atlastdr}, CMS \cite{cmstdr} and LHCb \cite{lhcbtdr} which record the resultant collisions. The layout of the \LHC ring is shown in Figure \ref{fig:lhc-ring}. The remaining four sections contain acceleration,collimation and beam dump systems. In the eight arc sections, the beams are steered by magnetic fields of up to 8 \T provided by super conduction dipole magnets, which are maintained at temperatures of 2 \K using superfluid helium. Additional magnets for focusing and corrections are also present in straight sections within the arcs and near the interaction regions where the detectors are situated. \\


\begin{figure}[!h]

\centering
\includegraphics[width=0.65\textwidth]{plots/lhc-ring-photo.png}
\caption[A top down layout of the \LHC.]{A top down layout of the \LHC. \cite{Jean-Luc:841573}}  
\label{fig:lhc-ring}
\end{figure}


Proton beams are formed inside the Proton Synchrotron (PS) from bunches of protons 50 \ns apart with an energy of 26 \GeV. The protons are then accelerated in the Super Proton Synchrotron(SPS) to 450 \GeV  before being injected into the \LHC. These \LHC proton beams consists of many "bunches" i.e. approximately $1.1 \times 10^{11}$  protons localized into less than 1 \ns in the direction of motion. Before collision the beams are ramped to 4 \TeV (2012) per beam in a process involving increasing the current passing through the dipole magnets. Once the desired \com energy is reached then the beams are allowed to collide at the interaction points. The luminosity falls regularly as the run progresses as protons are lost in collisions, and eventually the beam is dumped before repeating the process again. \\

In the early phase of prolonged operation after the initial shutdown the machine operated in 2010-2011 at 3.5 \TeV per beam, \com $=$ 7 \TeV, delivering 6.13 \fb of data \cite{LHClumo}. During the 2012-2013 run period, data was collected at an increased \com $=$ 8 \TeV improving the sensitivity of searches for new physics. Over the whole run period 23.3 \fb of data was delivered of which 21.8 \fb was recorded by the \CMS detector \cite{LHClumo}. A total of 12 \fb of 8 \TeV certified data was collected by October 2012, and it is this data which forms the basis of the results discussed within this thesis.

\begin{figure}[!h]

\centering
\includegraphics[width=0.65\textwidth]{plots/lhc-lumo-8tev.png}
\caption[The total integrated luminosity delivered to and collected by \CMS during the 2012 8 \TeV \pp runs]{The total integrated luminosity delivered to and collected by \CMS during the 2012 8 \TeV \pp runs.}  
\label{fig:lhc-ring}
\end{figure}


\section{The CMS detector}
\label{sec:cmsdetector}

The Compact Muon Solenoid (\CMS) detector is one of two general purpose detectors at the \LHC designed to search for new physics. The detector is designed to provide efficient identification and measurement of many physics objects including photons, electrons, muons, taus, and hadronic showers over wide ranges of transverse momentum and direction. Its nearly 4$\pi$ coverage in solid angle allows for accurate measurement of global transverse momentum imbalance. These design factors give \CMS the ability to search for direct production of \SUSY particles at the \TeV scale, making the search for Supersymmetric particles one of the highest priorities among the wide range of physics programmes at \CMS. \\

\CMS uses a right-handed Cartesian coordinate system with the origin at the interaction point and the z-axis pointing along the beam axis, the x-axis points radially inwards to the centre of the collider ring, with the y-axis points vertically upward. The azimuthal angle, $\phi$ ranging between [$-\pi$,$\pi$] is defined in the x-y plane starting from the x-axis. The polar angle $\theta$ is measured from the z axis. The common convention in particle physics is to express an out going particle in terms of $\phi$ and its pseudorapidity defined as

\begin{equation}
\eta = -\log\tan\left(\frac{\theta}{2}\right).
\end{equation}

The variable $\Delta R = \sqrt{\Delta\phi^{2} + \Delta\eta^{2} } $ is commonly used to define angular distance between objects within the detector and additionally energy and momentum is typically measured in the transverse plane perpendicular to the beam line. These values are calculated from the x and y components of the object and are denoted as $\et = E\sin\theta$ and $\pt = \sqrt{p^{2}_{x}+p^{2}_{y}}$. 

\subsection{Detector Subsytems}
\label{subsec:detectorsubsystems}

As the range of particles produced in \pp collisions interact in different ways with matter, \CMS is divided into subdetector systems, which perform complementary roles to identify the identity, mass and momentum of the different physics objects present in each event. These detector sub-systems contained inside \CMS are wrapped in layers around a central 13 m long 4 \T super conducting solenoid as shown in Fig \ref{fig:cms-detector}. With the endcaps closed , \CMS is a cylinder of length 22 m, diameter 15 m, and mass 12.5 kilotons. A more detailed complete description of the detector can be found elsewhere \cite{cmstdr}. \\

\begin{figure}[!h]

\centering
\includegraphics[width=0.65\textwidth]{plots/cms-detector.png}
\caption[A pictorial depiction of the \CMS detector.]{A pictorial depiction of the \CMS detector with the main detector subsystems labelled.   \cite{cms-public-detector}}  
\label{fig:cms-detector}
\end{figure}

\subsection{Tracker}
\label{subsec:tracker}

 The inner-most subdetector of the barrel is the multi-layer silicon tracker, formed of a pixel detector component encased by layers of silicon strip detectors. The pixel detector consists of three layers of silicon pixel sensors providing measurements of the momentum, position coordinates of the charged particles as they pass, and the location of primary and secondary vertices between 4cm and 10cm transverse to the beam. Outside the pixel detector, ten cylindrical layers of silicon strip detectors extend the tracking system out to a radius of 1.20m from the beam line. The tracking system provides efficient and precise determination of the charges, momenta, and impact parameters of charged particles with the geometry of the tracker extending to cover a rapidity range up to $\lvert\eta\rvert \textless$ 2.5.  \\
 
 The tracking system also plays a crucial part in the identification of jets originating from b-quarks through measurement of displaced secondary vertices, which is covered in more detail in Section \ref{subsec:cmsobjects-btagging}. The identification of b-jets is important in many searches for natural \SUSY models and forms an important part of the inclusive search strategy described within Section \ref{subsec:searchstrategy}.
 
\subsection{Electromagnetic Calorimeter}
\label{subsec:ecal}

 Immediately outside of the tracker, but still within the magnet core, sits the Electromagnetic Calorimeter (\ECAL). Covering a pseudorapididity up to $\lvert\eta\rvert < 3$ and compromising of over 75,000 PbWO$_{4}$ (lead tungstate) crystals that scintillate as particles deposit energy, the \ECAL provides high resolution measurements of the electromagnetic showers from photons, electrons in the detector. \\ 
 
 Lead tungstate is used because of its short radiation length ($X_{0} \sim 0.9$cm) and small Molier\'{e} radius ($\sim 2.1$cm) leading to high granularity and resolution. It's fast scintillation time ($\sim 25$ns) reduces the effects of pileup due to energy from previous collisions still being read out, and its radiation hardness gives it longevity. The crystals are arranged in modules which surround the beam line in a non-projective geometry,  angled at 3$^{\circ}$ with respect to the interaction point to minimise the risk of particles escaping down the cracks between the crystals.\\
 
 The  \ECAL is primarily composed of two sections, the Electromagnetic Calorimeter Barrel (\EB) which extends in pseudo-rapidity to $\lvert\eta\rvert < 1.479$ with a crystal front cross section of 22 $\times$ 22 mm$^{2}$ and a length of 230 mm corresponding to 25.8 radiation lengths, and the Electronmagnetic Calorimeter Endcaps (\EE) covering a rapidity range of $1.479 < \lvert\eta\rvert < 3.0 $, which consists of two identical detectors
on either side of the EB.  A lead-silicon sampling 'pre-shower' detector (\ES) is placed before the endcaps to aid in the identification of neutral pions. Their arrangement are shown in Figure \ref{fig:cms-ecal}. \\

 
 \begin{figure}[!h]

\centering
\includegraphics[width=0.85\textwidth]{plots/cms-ecal.png}
\caption[Illustration of the \CMS \ECAL showing the arrangement of the lead tungstate crystals in the \EB and \EE. The \ES is also shown and is located infront of the \EE.]{Illustration of the \CMS \ECAL showing the arrangement of the lead tungstate crystals in the \EB and \EE. The \ES is also shown and is located infront of the \EE \cite{CMS_ECAL_TDR}.}  
\label{fig:cms-ecal}
\end{figure}


Scintillation photons from the lead tungstate crystals are instrumented with avalanche photo-diodes (\APD) and vacuum photo-triodes (\VPT) located in the \EB and \EE respectively, converting the scintillating light into an electric signal which is consequently used to determine the amount of energy deposited within the crystal . These instruments are chosen for their resistance under operation to the strong magnetic field of \CMS. The scintillation of the \ECAL crystals as well as the response of the \APD s varies as a function of temperature and so cooling systems continually maintain an overall constant \ECAL temperature $\pm 0.05 ^{\circ}C$.
 

\subsection{Hadronic Calorimeter}
\label{subsec:hcal} 
Beyond the \ECAL lies the Hadronic Calorimeter (\HCAL), which is responsible for the accurate measurement of hadronic showers, crucial for analyses involving jets or missing energy signatures. The \HCAL is a sampling calorimeter which consists of alternating layers of brass absorber and plastic scintillator, except in the hadron forward ($3.0 < \lvert\eta\rvert < 5.0 $) region in which steel absorbers and quartz �bre scintillators are used because of their increased radiation tolerance. The \HCAL's size is constrained to a compact size by the presence of the solenoid, requiring the placement of an additional outer calorimeter on the outside of the solenoid to increase the sampling depth of the \HCAL .\\
 
 
The HCAL covers the range $\lvert\eta\rvert < 5$ and consists of four subdetectors: the Hadron Barrel (\HB), the Hadron Outer (\HO), the Hadron Endcap (\HE) and the Hadron Forward (\HF). \\

Hadron showers initiated in the absorber layers induce scintillation in the tiles which is converted by wavelength shifting fibres for read-out by hybrid photodiodes
 
\subsection{Muon Systems}
\label{subsec:muonsystems} 
 Muon \\ 


\section{Event Reconstruction and Object Definition}
\label{sec:cmsobjects}

The goal of event reconstruction is to take the raw information recorded by the detector and to compute from it higher-level quantities which can be used at an analysis level. These typically correspond to an individual particle's energy and momenta, or groups of particles which shower in a narrow cone and the overall global energy and momentum balance of the event. The reconstruction of these objects are described in great detail in \cite{CMS_TDR_PHYS_vol1}, however covered below are brief descriptions of those which are most relevant to the analysis detailed in Section \ref{chap:SUSYsearches}.

\subsection{Jets}
\label{subsec:cmsobjects-jets}

Quarks and gluons are produced copiously at the LHC in the hard scattering of partons. \\

\subsection{B-tagging}
\label{subsec:cmsobjects-btagging}

The decays of b quarks are suppressed by small \CKM matrix elements. As a result, the lifetimes of b-flavoured hadrons, produced in the fragmentation of b quarks, are relatively long; $\mathcal{O}$ 1ps. Testing

\section{L1 Trigger}
\label{sec:l1trigger}


L1 Work

  \chapter{SUSY searches in Hadronic Final States}
\label{chap:SUSYsearches}

In this chapter a model independent search for \ac{SUSY} in hadronic final states with $\met$ using the $\alphat$ variable and b-quark multiplicity is introduced and described in detail. The results presented are based on a data sample of pp collisions collected in 2012 at $\com =$8 \TeV, corresponding to an integrate luminosity of 11.7$\pm$0.5 fb$^{-1}$.

The kinematic variable $\alphat$ is motivated as a variable to provide strong rejections of QCD backgrounds, whilst maintaining sensitivity to possible a \ac{SUSY} signal within Section (\ref{sec:alphatintroduction}). The search and trigger strategy in addition to the event reconstruction and selection are outlined within Sections (\ref{subsec:searchstrategy}-\ref{subsec:triggerstrategy}). 

The method in which the \ac{SM} background is estimated using an analytical technique to improve statistical precision at higher b-tag multiplicities is detailed within Section (\ref{subsec:backgroundestimation}), with a discussion on the impact of b-tagging and mis-tagging scale factors between data and MC on any background predictions. Finally a description of the formulation of appropriate systematic uncertainties applied to the background predictions to account for theoretical uncertainties and limitations in the simulation modelling of event kinematics and instrumental effects is covered in Section (\ref{subsec:sysuncertainties}).

 
The experimental reach of the analysis discussed within this thesis is interpreted in two classes of \ac{SMS} models, the topologies of which are detailed in Section (\ref{subsec:sms}). The \ac{SMS} models considered in this analysis are summaries in Table \ref{tab:sms_model_table}. For each model, the \ac{LSP} is assumed to be the lightest neutralino. 

Within Table \ref{tab:sms_model_table} is also defined reference points, parameterised in terms of parent gluino/squark and \ac{LSP} sparticle masses, m$_{parent}$ and m$_{LSP}$, respectively, which are used within the following two chapters to demonstrate potential yields within the signal region of the search. The masses are chosen to reflect parameter space which is within the expect sensitivity reach of the search.

\begin{table}[h!]
\begin{center}
\begin{tabular*}{0.75\textwidth}{@{\extracolsep{\fill}}llcc}
\cline{1-4}
Model & Production/decay mode &  \multicolumn{2}{c}{Reference model}\\ 
&& m$_{parent}$ & m$_{LSP}$ \\  \cline{1-4}
G1 (T1) & $ pp \rightarrow \widetilde{g}\widetilde{g}^{*} \rightarrow q\bar{q}\widetilde{\chi}^{0}_{1}q\bar{q}\widetilde{\chi}^{0}_{1}$ & 700 & 300 \\
G2 (T1bb) & $ pp \rightarrow \widetilde{g}\widetilde{g}^{*} \rightarrow b\bar{b}\widetilde{\chi}^{0}_{1}b\bar{b}\widetilde{\chi}^{0}_{1}$ & 900 & 500 \\
G3 (T1tt) & $ pp \rightarrow \widetilde{g}\widetilde{g}^{*} \rightarrow t\bar{t}\widetilde{\chi}^{0}_{1}t\bar{t}\widetilde{\chi}^{0}_{1}$ & 850 & 250 \\
D1 (T2) & $ pp \rightarrow \widetilde{q}\widetilde{q}^{*} \rightarrow q\widetilde{\chi}^{0}_{1}\bar{q}\widetilde{\chi}^{0}_{1}$ & 600 & 250 \\
D2 (T2bb) & $ pp \rightarrow \widetilde{b}\widetilde{b}^{*} \rightarrow b\widetilde{\chi}^{0}_{1}\bar{b}\widetilde{\chi}^{0}_{1}$ & 500 & 150 \\
D3 (T2tt) & $ pp \rightarrow \widetilde{t}\widetilde{t}^{*} \rightarrow t\widetilde{\chi}^{0}_{1}\bar{t}\widetilde{\chi}^{0}_{1}$ & 400 & 0 \\
\cline{1-4}
\end{tabular*}
\end{center}
\caption[A summary of the \ac{SMS} models interpreted in this analysis, involving both direct (D) and glunio-induced (G) production of squarks and their decays.]{A summary of the \ac{SMS} models interpreted in this analysis, involving both direct (D) and glunio-induced (G) production of squarks and their decays. Reference models are also defined in terms of parent and \ac{LSP} sparticle mass }
\label{tab:sms_model_table}
\end{table}

\section{An introduction to the \alphat search}
\label{sec:alphatintroduction}

The experimental signature of \ac{SUSY} signal in the hadronic channel would manifest as a final state containing energetic jets and $\met$. The search focuses on topologies where new heavy supersymmetric, R-parity conserving particles are pair-produced in pp collisions. These particles decaying to a \ac{LSP} escape the detector undetected, leading to significant missing energy and missing hadronic transverse energy,

\begin{equation}
\mht =  \lvert \sum_{i=1}^{n} p_{T}^{jet_{i}} \rvert,
\end{equation}

defined as the vector sum of the transverse energies of jets selected in an event. Energetic jets produced in the decay of these supersymmetric particles also 
can produce significant visible transverse energy, 

\begin{equation}
\theht = \sum_{i=1}^{n} E_{T}^{jet_{i}},
\end{equation}

defined as the scalar sum of the transverse energies of jets selected in an event.

A search within this channel is greatly complicated in a hadron collider environment, where the overwhelming background comes from inherently balanced multi-jet (``QCD'') events which are produced with an extremely large cross section as demonstrated within Figure \ref{fig:htqcdbackground}. $\met$ can appear in such events with a substantial mis-measurement of jet energy or missed objects due to detector miscalibration or noise effects. 

\begin{figure}[!h]

\centering
\includegraphics[width=0.60\textwidth]{plots/nocuts_htdistribution.pdf}
\caption[Reconstructed offline \theht for 11.7fb$^{-1}$ of data after a basic pre-selection.]{Reconstructed offline $\theht$ for 11.7fb$^{-1}$ of data after a basic pre-selection. Sample is collected from prescaled \theht triggers. Overlaid are expectations from MC simulation of \ac{EWK} processes as well as a reference signal model (labelled D2 from Table.\ref{tab:sms_model_table}).}  
\label{fig:htqcdbackground}
\end{figure}

Additional \ac{SM} background from \ac{EWK} processes with genuine $\met$ from escaping neutrinos comprise the irreducible background within this search and come mainly from:

\begin{itemize}
\item $Z \rightarrow \nu\bar{\nu} +$ jets,
\item $W \rightarrow l\nu$ + jets in which a lepton falls outside of detector acceptance, or the lepton decays hadronically $\tau \rightarrow$ had ,
\item $t\bar{t}$ with at least one leptonic W decay,
\item small background contributions from DY, single top and Diboson (WW,ZZ,WZ) processes.
\end{itemize}

The search is designed to have a strong separation between events with genuine and ``fake'' $\met$ which is achieved primarily though the dimensionless kinematic variable, $\alphat$ \cite{PhysRevLett.101.221803}\cite{CMS:2008vya}.

\subsection{The $\alphat$ variable}
\label{subsec:alphatvariable}

For a perfectly measured di-jet QCD event, conservation laws dictate that they must be produced back-to-back and of equal magnitude. However in di-jet events with real $\met$, both of these jets are produced independently of one another, depicted in Figure \ref{fig:susytopology}.
\begin{figure}[!h]
\centering
\includegraphics[width=0.90\textwidth]{plots/susy_topology.pdf}
\caption[The event topologies of background QCD diet events (right) and a generic \ac{SUSY} signature with genuine $\met$ (left).]{The event topologies of background QCD diet events (right) and a generic \ac{SUSY} signature with genuine $\met$ (left).}  
\label{fig:susytopology}
\end{figure}

 Exploiting this feature leads to the formulation of $\alphat$ in di-jet systems defined as,

\begin{equation}
\alphat = \frac {E^{j2}_{T}}{M_{T}},
\end{equation} 

where $E^{j2}_{T}$ is the transverse energy of the least energetic of the two jets and $M_{T}$ defined as:

\begin{equation}
\label{eq:transmass}
M_{T} = \sqrt{\left(\sum^{2}_{i=1}E^{j_{i}}_{T}\right)^{2}-\left(\sum^{2}_{i=1}p^{j_{i}}_{x}\right)^{2}-\left(\sum^{2}_{i=1}p^{j_{i}}_{y}\right)^{2}} \equiv \sqrt{H_{T}^{2} - \mht^{2}} .
\end{equation}

A perfectly balanced di-jet event i.e. $E_{T}^{j_{1}} = E_{T}^{j_{2}}$ would give an $\alphat = 0.5$, where as events with jets which are not back-to-back, for example in events in which
a W or Z recoils off a system of jets, $\alphat$ can achieve values in excess of 0.5.

$\alphat$ can be extended to apply to any arbitrary number of jets, undertaken by modelling a system of $n$ jets as a di-jet system, through the formation of two pseudo-jets \cite{CMS-PAS-SUS-09-001}. The two pseudo-jets are built by merging the jets present in the event such that the 2 pseudo-jets are chosen to be as balanced as possible, i.e the $\Delta$ \theht $\equiv \lvert E_{T}^{pj_{1}} - E_{T}^{pj_{2}}\rvert$ is minimised between the two pseudo jets. Using Equation (\ref{eq:transmass}), $\alphat$ can be rewritten as,

\begin{equation}
\label{eq:alphatmht}
\alphat = \frac{1}{2} \frac {\theht - \Delta\theht}{\sqrt{\theht^{2}-\mht^{2}}}= \frac{1}{2}\frac{1-\Delta\theht/\theht}{\sqrt{1-(\mht/\theht)^{2}}}.
\end{equation}

The distribution of $\alphat$ for the two jet categories used within this analysis, 2,3 and $\geq 4$ jets, is shown in the Figure.\ref{fig:fullalphatdistribution}, demonstrating the ability of the $\alphat$ variable to discriminate between multi jet events and \ac{EWK} processes with genuine $\met$ in the final state.  

\begin{figure}[ht]
\centering
\begin{minipage}[b]{0.48 \linewidth}
\includegraphics[width = 1.0\linewidth,height = 7.0cm]{plots/alphat_low.pdf}
\end{minipage}
\quad
\begin{minipage}[b]{0.48\linewidth}
\includegraphics[width = 1.0\linewidth, height = 7.0cm]{plots/alphat_high.pdf}
\end{minipage}
\caption[ The $\alphat$ distributions for the low 2-3 (left) and high $\geq 4$ (right) jet multiplicities after a full analysis selection and shown for $\theht > 375$.]{The $\alphat$ distributions for the low 2-3 (left) and high $\geq 4$ (right) jet multiplicities after a full analysis selection and shown for $\theht > 375$ . Data is collected using both prescaled $\theht$ triggers and dedicated $\alphat$ triggers for below and above $\alphat = 0.55$ respectively. . Expected yields as given by simulation are also shown for multijet events (green dash-dotted line), \ac{EWK} backgrounds with genuine $\met$ (blue long-dashed line), the sum of all \ac{SM} processes (cyan solid line) and the reference signal model D2 (left, red dotted line) or G2 (right, red dotted line). }
\label{fig:fullalphatdistribution}
\end{figure}

The $\alphat$ requirement used within the search is chosen to be $\alphat >$ 0.55 to ensure that the QCD multijet background is negligible even in the presence of moderate jet mis-measurement. There still remains other effects which can cause multijet events to artificially have a large $\alphat$ value, which are discussed in detail in Section (\ref{subsec:eventselection}).  


\section{Search Strategy}
\label{subsec:searchstrategy}

The aim of the analysis presented in this thesis is to identify an excess of events in data over the \ac{SM} background expectation in multi-jet final states and significant $\met$. The essential suppression of the dominant QCD background for such a search is addressed by the $\alphat$ variable described in the previous section. For estimation of the remaining \ac{EWK} backgrounds, three independent data control samples are used to predict the different processes that compose the background :

\begin{itemize}
\item \mupjets to determine W + jets, \ttbar and single top backgrounds,
\item \gpjets  to determine the irreducible \zinv + jets background,
\item \dimupjets to determine the irreducible \zinv + jets background.
\end{itemize}

These control samples are chosen to both be rich in specific \ac{EWK} processes, be free of QCD multi-jet events and to also be kinematically similar to the hadronic signal region that they are estimating the backgrounds of, see Section (\ref{subsec:controlsampledefinition}).

To remain inclusive to a large range of possible \ac{SUSY} models, the signal region is binned in the following categories to allow for increased sensitivity in the interpretation of results for different \ac{SUSY} topologies:

\begin{itemize}

\item[] \textbf{Sensitivity to a range of \ac{SUSY} mass splittings}

The hadronic signal region is defined by \theht $>$ 275, divided into eight bins in \theht. 

\begin{itemize}
\item Two bins of width 50 \GeV in the range 275 $<$ \theht $<$ 375 \GeV,
\item five bins of width 100 \GeV in the range 375 $<$ \theht$<$ 875 \GeV,
\item and a final open bin, \theht $>$ 875 \GeV.
\end{itemize}

The choice at low \theht is driven primarily by trigger constraints. The mass difference between the \ac{LSP} and the particle that it decays from is an important factor in the amount of hadronic activity in the event. 

A large mass splitting will lead to hard high \pt jets which contribute to the \theht sum. From Figure \ref{fig:htqcdbackground} it can be seen that the \ac{SM} background falls sharply at high \theht values, therefore a large number of \theht bins will lead to easier of identification of such signals. Conversely smaller mass splittings lead to softer jet \pt's which will subsequently fall into the lower \theht range.

\item[] \textbf{Sensitivity to production method of \ac{SUSY} particles}

The production mechanism of any potential \ac{SUSY} signal can lead to different event topologies. One such way to discriminate between gluino ($g\widetilde{g}$ - ``high multiplicity''), and direct squark ($q\widetilde{q}$ - ``low multiplicity'') induced production of \ac{SUSY} particles is realised through the number of reconstructed jets in the final state.  

The analysis is thus split into two jet categories : 2-3 jets , $\geq$ 4 jets to give sensitivity to both of these mechanisms. 

\item[] \textbf{Sensitivity to  ``Natural \ac{SUSY}'' via tagging jets from b-quarks}

Jets originating from bottom quarks (b-jets) are identified through vertices that are displaced with respect to the primary interaction. The algorithm used to tag b-jets is the \acf{CSVM} tagger, described within Section (\ref{subsec:cmsobjects-btagging}). A cut is placed on the discriminator variable of $> 0.679$, leading to a gluon/light-quark mis-tag rate of 1\% and a jet p$_{\text{T}}$ dependant b-tagging efficiency of 60-70\% \cite{btag8tev}.

Natural \ac{SUSY} models would be characterised through final-state signatures rich in bottom quarks. A search relying on methods to identify jets originating from bottom quarks through b-tagging, will significantly improve the sensitivity to this class of signature. 

This is achieved via the binning of events in the signal region according to the number of b-tagged jets reconstructed in each event, in the following: 0,1,2,3,$\geq$ 4 b-tag categories . In the highest $\geq$ 4 b-tag category due to a limited number of expected signal and background, just three \theht bins are employed: 275-325 \GeV, 325-375 \GeV, $\geq$ 375 \GeV.

This characterisation is identically mirrored in all control samples, with the information from all samples and b-tag categories used simultaneously in the likelihood model (see Chapter \ref{chap:SUSYresults}) in order to interpret the results in a coherent and powerful way.

\end{itemize}
 
 The combination of the \theht, jet multiplicity and b-tag categorisation of the signal region as described above, resultantly leads to 67 different bins in which the analysis is interpreted in, which is depicted in Figure \ref{fig:analysisbinning}. 
 
 \begin{figure}[!h]
 \centering
\includegraphics[width=0.70\textwidth]{plots/analysis_binning.pdf}
\caption[Pictorial depiction of the analysis strategy employed by the $\alphat$ search to increase sensitivity to a wide spectra of \ac{SUSY} models.]{Pictorial depiction of the analysis strategy employed by the $\alphat$ search to increase sensitivity to a wide spectra of \ac{SUSY} models.}  
\label{fig:analysisbinning}
\end{figure}



\subsection{Physics Objects}
\label{subsec:physicsobjects}

The physics objects used in the analysis defined below, follow the recommendation of the various \ac{CMS} \acf{POGs}. 

\begin{itemize}

\item \textbf{Jets}

The jets used in this analysis are CaloJets, reconstructed as described in Section (\ref{subsec:cmsobjects-jets}) using the anti-k$_{T}$ jet clustering algorithm. 

To ensure the jet object falls within the calorimeter systems a pseudo-rapidity requirement of \abeta $<$ 3 is applied. Each jet must pass a ``loose'' identification criteria to reject jets resulting from unphysical energy, the criteria of which are detailed in Table \ref{tabapp:calojetid} of Appendix (\ref{app:noise})  \cite{CMS-PAS-JME-09-008}.

\item \textbf{Muons}

Muons are selected in the \mupjets and \dimupjets control samples, and vetoed in the signal region. The same cut based identification criteria is applied to muons in both search regions and is summarised in Table \ref{tab:muonidtable} \cite{1748-0221-7-10-P10002}.

\begin{table}[h!]
\begin{center}
\begin{tabular*}{0.5\textwidth}{@{\extracolsep{\fill}}ll}
\cline{1-2}
Categories & Criteria \\ 
\cline{1-2}
Global Muon & True \\
PFMuon & True \\
$\chi^{2}$ & $<$ 10 \\
Muon chamber hits & $>$ 0 \\
Muon station hits & $>$ 1 \\
Transvere impact d$_{xy}$ & $<$ 0.2mm \\
Longitudinal distance d$_{z}$ & $<$ 0.5mm \\
Pixel hits & $>$ 0\\
Track layer hits & $>$ 5 \\
PF Isolation (DeltaB corrected) & $<$0.12 \\
\cline{1-2}
\end{tabular*}
\end{center}
\caption[Muon Identification criteria used within the analysis for selection/veto purposes in the muon control/signal selections.]{Muon Identification criteria used within the analysis for selection/veto purposes in the muon control/signal selections.}
\label{tab:muonidtable}
\end{table}

Additionally muons are required to be within the acceptance of the muon tracking systems. For the muon control samples, trigger requirements necessitate a \abeta $<$ 2.1 for the selection of muons. In the signal region where muons are vetoed these conditions are relaxed to  \abeta $<$ 2.5 and a minimum threshold of \pt $> 10 $ \GeV is required of muon objects. 

\item \textbf{Photons} 

Photons are selected within the \gpjets control sample and vetoed in all other selections. Photons are identified in both cases according to the cut based criteria listed in Table \ref{tab:photonidtable} \cite{CMS-PAS-SUS-12-018}.

\begin{table}[h!]
\begin{center}
\begin{tabulary}{0.80\textwidth}{LL}
\cline{1-2}
Variable & Definition \\ 
\cline{1-2}
H/E $< $ 0.05  \qquad\qquad\qquad\qquad\qquad\qquad & The ratio of hadronic energy in the \ac{HCAL} tower directly behind the \ac{ECAL} super-cluster and the \ac{ECAL} super-cluster itself. \\
$\sigma_{i\eta i\eta}< 0.011$ \qquad\qquad\qquad\qquad\qquad\qquad\qquad\qquad  & The log energy weighted width ($\sigma$), of the extent of the shower in the $\eta$ dimension.\\
R9 $<$ 1.0 & The ratio of the energy of the 3$\times$3 crystal core of the super-cluster compared to the total energy stored in the 5$\times$5 super-cluster. \\
Combined Isolation $<$ 6 \GeV &  The photons are required to be isolated with no electromagnetic or hadronic activity within a radius $\Delta$R = 0.3 of the photon object. A combination of the pileup subtracted \cite{Cacciari:2007fd}, \ac{ECAL}, \ac{HCAL} and tracking isolation sums are used to determine the combined total isolation value.  \\
\cline{1-2}
\end{tabulary}
\end{center}
\caption[Photon Identification criteria used within the analysis for selection/veto purposes in the \gpjets control/signal selections. ]{Photon Identification criteria used within the analysis for selection/veto purposes in the \gpjets control/signal selections.}
\label{tab:photonidtable}
\end{table}

Photon objects are also required to have a minimum momentum of \pt $>$ 25 \GeV.

\item \textbf{Electrons}

Electron identification is defined for veto purposes. They are selected according to the following cut-based criteria listed in Table \ref{tab:electronidtable}, utilising PF-based isolation.

\begin{table}[h!]
\begin{center}
\begin{tabular*}{0.5\textwidth}{@{\extracolsep{\fill}}lcc}
\cline{1-3}
Categories & Barrel &  EndCap\\ 
\cline{1-3}
$\Delta \eta_{In}$ & 0.007 & 0.009 \\
$\Delta \phi_{In}$ & 0.15 & 0.10 \\
$\sigma_{i\eta i\eta}$ & 0.01 & 0.03 \\
H/E & 0.12 & 0.10 \\
d0 (vtx) & 0.02 & 0.02 \\
dZ (vtx) & 0.20 & 0.20 \\
$\lvert$(1/E$_{ECAL}$ - 1/p$_{track}$)$\rvert$ & 0.05 & 0.05 \\
PF Combined isolation/\pt & 0.15 & 0.15 \\
Vertex fit probability & 10$^{-6}$ & 10$^{-6}$ \\
\cline{1-3}
\end{tabular*}
\end{center}
\caption[Electron Identification criteria used within the analysis for veto purposes.]{Electron Identification criteria used within the analysis for veto purposes.}
\label{tab:electronidtable}
\end{table}

Electrons are required to be identified at \abeta $<$ 2.5, with a minimum \pt $>$ 10 \GeV threshold to ensure that the electron falls within the tracking system of the detector.

\item \textbf{Noise and \met Filters}

A series of Noise filters are applied to veto events which contain spurious non-physical jets that are not picked up by the jet id, and events which give large unphysical \met values. These filters are listed within Table \ref{apptab:noiseid} of Appendix (\ref{app:noise}).

\end{itemize}


\subsection{Event Selection}
\label{subsec:eventselection}

The selection criteria for events within the analysis are detailed below. A set of common cuts are applied to both signal  (maximise acceptance to a range of \ac{SUSY} signatures),  and control samples (retain similar jet kinematics for background predictions), with additional selection cuts applied to each control sample to enrich the sample in a particular \ac{EWK} processes, see Section (\ref{subsec:controlsampledefinition}).

The jets considered in the analysis are required to have a transverse momentum \pt $>$ 50 \GeV, with a minimum of two jets required in the event. The highest \et jet is required to lie within the central tracker acceptance \abeta $<$ 2.5, and the two leading \pt jets must each have \pt $>$ 100\GeV.  Any event which has a jet with \pt $>$ 50 \GeV that either fails the ``loose'' identification criteria described in Section(\ref{subsec:physicsobjects}) or has \abeta $>$ 3.0, is rejected. Similarly events in which an electron,muon or photon fails object identification but pass \eta and \pt restrictions are identified as an ``odd'' lepton/photon and the event is vetoed.

At low \theht, the jet threshold requirements applied to be considered as part of the analysis and enter the \theht sum are scaled downwards. These are scaled down in order to not restrict phase space, preserving jet multiplicities and background admixture in the lower \theht bins, as listed in Table \ref{tab:jetthresholdtable}.

\begin{table}[h!]
\begin{center}
\begin{tabular*}{0.6\textwidth}{@{\extracolsep{\fill}}|l|c|c|}
\cline{1-3}
\theht bin & minimum jet \pt &  second leading jet \pt \\ 
\cline{1-3}
275 $<$ \theht$<$ 325 & 36.7 & 73.3 \\
325 $<$ \theht$<$ 375 & 43.3 & 86.6 \\
375 $<$ \theht & 50.0 & 100.0 \\

\cline{1-3}
\end{tabular*}
\end{center}
\caption[Jet thresholds used in the three \theht regions of the analysis.]{Jet thresholds used in the three \theht regions of the analysis.}
\label{tab:jetthresholdtable}
\end{table}

Within the signal region to suppress \ac{SM} processes with genuine \met from neutrinos, events containing isolated electrons or muons are vetoed. Furthermore to ensure a pure multi-jet topology, events are vetoed if an isolated photon is found with \pt $>$ 25 \GeV. 

An \alphat requirement of $>$ 0.55 is required to reduce the QCD multi-jet background to a negligible amount. Finally additional cleaning cuts are applied to protect against pathological deficiencies such as reconstruction failures or severe energy mis-measurements due to detector inefficiencies:

\begin{itemize}
\item Significant \mht can arise in events with no real \met due to multiple jets falling below the \pt threshold used for selecting jets. This in turn leads to events which can then incorrectly pass the \alphat requirements of the analysis. This effect can be negated by requiring that the missing transverse momentum reconstructed from jets alone does not greatly exceed the missing transverse momentum reconstructed from all of the detector's calorimeter towers,
\begin{equation}
R_{miss} = \mht / \met < 1.25. \nonumber
\end{equation}  

\item Fake \met and \mht can arise due to significant jet mis-measurements cause by a small number of non-functioning \ac{ECAL} regions. These regions absorb electromagnetic showers which are subsequently not added to the jet energy sum. To circumvent this problem the following procedure is employed : For each jet in the event, the angular separation

\begin{equation}
\Delta\phi_{j}^{*}\equiv \Delta\phi(p_{j}^{\rightarrow}-\sum_{i\neq j}p_{i}^{\rightarrow}),
\end{equation}

is calculated where that jet is itself removed from the event. Here $\Delta\phi^{*}$ is a measure of how aligned the \mht of an event is with a jet, a small value is compatible with the hypothesis of an inherently balanced event in
which a jet has been mis-measured. For every jet in a event with $\Delta\phi^{*} <$ 0.5, if the $\Delta R$ distance between the selected jet and the closest dead \ac{ECAL} region is also $<$ 0.3, then the event is rejected. Similarly events are rejected if the jet points within $\Delta R <$ 0.3 of the \ac{ECAL} barrel-endcap gap at \abeta $=$ 1.5.

\end{itemize}

Some of the key distributions of the data used in this analysis compared to MC simulation are shown in Figure \ref{fig:hadmcplots}. The MC samples are normalised to a luminosity of 11.7 \fb,  with no requirement placed upon the number of b-tagged jets or number of jets in the events. 

The distributions shown are presented for purely illustrative purposes, with the MC simulation itself not used in absolute term to estimate the yields from background processes, see Sections (\ref{subsec:controlsampledefinition},\ref{subsec:backgroundestimation}). However it is nevertheless important to demonstrate that good agreement exists between simulation and observation in data.

\begin{minipage}{\linewidth}
\centering
\begin{minipage}{.48\textwidth}
\centering
\includegraphics[width = 3.3in]{plots/had_njet_datamc.pdf}
(a) Jet Multiplicity
\end{minipage}
\begin{minipage}{.48\textwidth}
\centering
\includegraphics[width = 3.3in]{plots/had_ht_datamc.pdf}
(b) \theht
\end{minipage}
\begin{minipage}{.48\textwidth}
\centering
\includegraphics[width = 3.3in]{plots/had_nbtag_datamc.pdf}
$\text{(c}$) Btag Multiplicity
\end{minipage}
\begin{minipage}{.48\textwidth}
\centering
\includegraphics[width = 3.3in]{plots/had_mht_datamc.pdf}
(d) \mht
\end{minipage}
\captionof{figure}[Data/MC comparisons of key variables for the hadronic signal region.]{Data/MC comparisons of key variables for the hadronic signal region,following the application of the hadronic selection criteria and the requirements of \theht $>$ 275 \GeV and \alphat $>$ 0.55. Bands represent the uncertainties due to the statistical size of the MC samples. No requirement is made upon the number of b-tagged jets or jet multiplicity in these distributions.}\label{fig:hadmcplots}
\end{minipage}

\subsection{Control Sample Definition and Background Estimation}
\label{subsec:controlsampledefinition}

The method used to estimate the background contributions in the hadronic signal region relies on the use of a \acf{TF}. This is determined from MC simulation in both the control, $\text{N}_{\text{MC}}^{\text{control}}$, and signal, $\text{N}_{\text{MC}}^{\text{signal}}$, region to transform the observed yield measured in data for a control sample,  $\text{N}_{\text{obs}}^{\text{control}}$, into a background prediction, $\text{N}_{\text{pred}}^{\text{signal}}$, via Equation (\ref{eq:transfactor}),

\begin{equation}
\label{eq:transfactor}
\text{N}_{\text{pred}}^{\text{signal}} = \frac{\text{N}_{\text{MC}}^{\text{signal}}}{ \text{N}_{\text{MC}}^{\text{control}}} \times  \text{N}_{\text{obs}}^{\text{control}}.
\end{equation}

All MC samples are normalised to the luminosity of the data samples, 11.7 \fb. Through this method, ``vanilla'' predictions for the \ac{SM} background in the signal region can be made by considering separately the sum of the prediction from either the \mupjets and \gpjets or \mupjets and \dimupjets samples. However the final background estimation from which results are interpreted, is calculated via a fitting procedure defined formally by the likelihood model described in Chapter \ref{chap:SUSYresults}. 

The sum of the expected yields from all MC processes, in each control sample enter the denominator, $\text{N}_{\text{MC}}^{\text{control}}$  , of the \ac{TF} defined in Eq (\ref{eq:transfactor}). However for the numerator , $\text{N}_{\text{MC}}^{\text{signal}}$, only the relevant processes that the control sample is used in estimating a background for, enter into the \ac{TF}.

For the \mupjets sample the simulated MC processes which enter the numerator of the \ac{TF} are,

\begin{equation} 
\text{N}_{\text{MC}}^{\text{signal}}(\theht,n_{\text{jet}}) = N_{W} + N_{\ttbar} + N_{DY} + N_{t} + N_{di-boson},
\end{equation}

whilst for both the \dimupjets and \gpjets samples the only MC process used in the numerator is,

\begin{equation} 
\text{N}_{\text{MC}}^{\text{signal}}(\theht,n_{\text{jet}}) = N_{\zinv}.
\end{equation}

The control samples and the \ac{EWK} processes they are specifically tuned to select are defined below, with distributions of key variables for each of the control samples shown for illustrative purposes in Figures \ref{fig:muonmcplots}, \ref{fig:dimuonmcplots} and \ref{fig:photonmcplots}. No requirement is placed upon the number of b-tagged jets or jet multiplicity in the distributions shown. The MC distributions highlight the background compositions of each control sample, where in general, good agreement is observed between data and simulation, giving confidence that the samples are well understood. The contribution from QCD multi-jet events is expected to be negligible : 

\begin{itemize} 

\item[] \textbf{The \mupjets control sample}

Events from W + jets and \ttbar processes enter into the hadronic signal sample due to unidentified leptons from acceptance or threshold effects and hadronic tau decays. These leptons originate from the decay of high \pt W bosons. 

The control samples specifically identifies $W \rightarrow \mu\bar{\nu}$ decays within the same phase-space of the signal region, where the muon is subsequently ignored in the calculation of event level variables, i.e. \theht, \mht, \alphat. All kinematic jet-based cuts are identical to those applied in the hadronic search region detailed in Section (\ref{subsec:eventselection}), with the same \theht, jet multiplicity and b-jet multiplicity binning described above.

\begin{itemize}
\item Muons originating from W boson decays are selected by requiring one tightly isolated muon defined in Table \ref{tab:muonidtable}, with a \pt $>$ 30 \GeV and \abeta $<$ 2.1. Both of these threshold arise from trigger restrictions.  
\item The transverse mass of the W candidate must satisfy \mt$(\mu,\met) <$ 30\GeV ( to suppress QCD multi-jet events). 
\item Events which contain a jet overlapping with a muon $\Delta \text{R}(\mu,\text{jet}) <$ 0.5 are vetoed to remove events from muons produced as part of a jet's hadronisation process. 
\item Events containing a second muon candidate which has failed id, but passed \pt and \abeta requirements, are checked to have an invariant mass that satisfies m$_{Z}$ - 25 $<$ M$_{\mu_{1}\mu_{2}} >$ m$_{Z}$ + 25, thus removing $Z \rightarrow \mu\mu$ contamination.
\end{itemize}


\begin{minipage}{\linewidth}
\centering
\begin{minipage}{.48\textwidth}
\centering
\includegraphics[width = 3.2in]{plots/muon_leadmu_datamc.pdf}
(a) Lead Muon \pt
\end{minipage}
\begin{minipage}{.48\textwidth}
\centering
\includegraphics[width = 3.2in]{plots/muon_ht_datamc.pdf}
(b) \theht
\end{minipage}
\end{minipage}
\xspace
\begin{minipage}{\linewidth}
\centering
\begin{minipage}{.48\textwidth}
\centering
\includegraphics[width = 3.2in]{plots/muon_alphat_datamc.pdf}
$\text{(c}$) \alphat
\end{minipage}
\begin{minipage}{.48\textwidth}
\centering
\includegraphics[width = 3.2in]{plots/muon_mt_datamc.pdf}
(d) Transverse mass $M_{T}$
\end{minipage}
\captionof{figure}[Data/MC comparisons of key variables for the \mupjets selection.]{Data/MC comparisons of key variables for the \mupjets selection,following the application of selection criteria and the requirements that \theht $>$ 275 \GeV. Bands represent the uncertainties due to the statistical size of the MC samples. No requirement is made upon the number of b-tagged jets or jet multiplicity in these distributions.}\label{fig:muonmcplots}
\end{minipage}

\item[] \textbf{The \dimupjets control sample}

The  \zinv + jets background enters into the signal region from genuine \met from the escaping neutrinos. This background is estimated using two control samples, the first of which is the \zmumu + jets process, which posses identical kinematic properties, but with different acceptance and branching ratio \cite{pdg2012}.

The same acceptance requirements as the \mupjets selection for muons is applied, as defined in Table  \ref{tab:muonidtable}. Muons  in the event are ignored for the purpose of the calculation of event level variables. Kinematic jet-based cuts and phase space binning identical to the hadronic search region are also applied.

\begin{itemize}
\item Muons origination from a Z boson decay are selected requiring exactly two tightly isolated muons. Due to trigger requirements the leading muon is required to have  \pt $>$ 30 \GeV and \abeta $<$ 2.1. The requirement of the \pt on the second muon is relaxed to 10 \GeV.
\item Events are vetoed if containing a jet overlapping with a muon $\Delta \text{R}(\mu,\text{jet}) <$ 0.5. 
\item In order to specifically select two muons both originating from a single Z boson decay, the invariant mass of the two muons must satisfy m$_{Z}$ - 25 $>$ M$_{\mu_{1}\mu_{2}} <$ m$_{Z}$ + 25. 
\end{itemize}

The \dimupjets sample is used to make predictions in the signal region in the two lowest \theht bins, providing coverage where the \gpjets sample is unable to, due to trigger requirements. In higher \theht bins, the higher statistics of the \gpjets sample is instead used to determine the \zinv estimation.

\begin{minipage}{\linewidth}
\centering
\begin{minipage}{.48\textwidth}
\centering
\includegraphics[width = 3.2in]{plots/dimuon_leadmu_datamc.pdf}
(a) Lead Muon \pt
\end{minipage}
\begin{minipage}{.48\textwidth}
\centering
\includegraphics[width = 3.2in]{plots/dimuon_ht_datamc.pdf}
(b) \theht
\end{minipage}
\end{minipage}

\xspace

\begin{minipage}{\linewidth}
\centering
\begin{minipage}{.48\textwidth}
\centering
\includegraphics[width = 3.2in]{plots/dimuon_alphat_datamc.pdf}
$\text{(c}$) \alphat
\end{minipage}
\begin{minipage}{.48\textwidth}
\centering
\includegraphics[width = 3.2in]{plots/dimuon_zmass_datamc.pdf}
(d) $\mu\mu$ invariant mass
\end{minipage}
\captionof{figure}[Data/MC comparisons of key variables for the \dimupjets selection.]{Data/MC comparisons of key variables for the \dimupjets selection,following the application of selection criteria and the requirements that \theht $>$ 275 \GeV. Bands represent the uncertainties due to the statistical size of the MC samples. No requirement is made upon the number of b-tagged jets or jet multiplicity in these distributions.}\label{fig:dimuonmcplots}
\end{minipage}


\item[] \textbf{The \gpjets control sample}

The \zinv + jets background is also estimated from a \gpjets control sample, which possesses a larger cross section and kinematic properties similar to those of \zmumu events where the photon is ignored \cite{PhysRevD.84.114002}\cite{CMS-PAS-SUS-08-002}. The photon is ignored for the purpose of the calculation of event level variables, and identical selection cuts to the hadronic signal region are applied. 

\begin{itemize}
\item Exactly one photon is selected, satisfying identification criteria as detailed in Table \ref{tab:photonidtable}, with a minimum \pt $> $165 \GeV to satisfy trigger thresholds and \abeta $<$ 1.45 to ensure the photon remains in the barrel of the detector.
\item A selection criteria of $\Delta R(\gamma,jet) <$ 1.0, between the photon and all jets is applied to ensure the acceptance of only well isolated \gpjets events. 
\item Given that the photon is ignored, this control sample can only be applied in the \theht region $>$ 375 \GeV, due to the trigger thresholds on the minimum \pt of the photon, and the \mht requirement of an \alphat $>$ 0.55 cut from Equation (\ref{eq:alphatmht}). 
\end{itemize}


\begin{minipage}{\linewidth}
\centering
\begin{minipage}{.48\textwidth}
\centering
\includegraphics[width = 3.2in]{plots/photon_leadphoton_datamc.pdf}
(a) Lead Photon \pt
\end{minipage}
\begin{minipage}{.48\textwidth}
\centering
\includegraphics[width = 3.2in]{plots/photon_ht_datamc.pdf}
(b) \theht
\end{minipage}
\begin{minipage}{.48\textwidth}
\centering
\includegraphics[width = 3.2in]{plots/photon_alphat_datamc.pdf}
$\text{(c}$) \alphat
\end{minipage}
\begin{minipage}{.48\textwidth}
\centering
\includegraphics[width = 3.2in]{plots/photon_njet_datamc.pdf}
(d) Jet multiplicity
\end{minipage}
\captionof{figure}[Data/MC comparisons of key variables for the \gpjets selection.]{Data/MC comparisons of key variables for the \gpjets selection,following the application of selection criteria and the requirements that \theht $>$ 375 \GeV and \alphat $>$ 0.55. Bands represent the uncertainties due to the statistical size of the MC samples. No requirement is made upon the number of b-tagged jets or jet multiplicity in these distributions.}\label{fig:photonmcplots}
\end{minipage}


\end{itemize}


The selection criteria of the three control samples are defined to ensure background composition and event kinematics mirror closely the signal region. This is done in order to minimise the reliance on MC simulation to model correctly the backgrounds and event kinematics in the control and signal samples. 

However in the case of the \mupjets and \dimupjets samples, the \alphat requirement is relaxed in the selection criteria of these samples. This is made possible as contamination from QCD multi-jet events is suppressed to a negligible level by the other kinematic selection criteria within the two control samples, to select pure \ac{EWK} processes. Thus in this way, the acceptance of the two muon control samples can be significantly increased, which simultaneously improves their predictive power and further reduces the effect of any potential signal contamination. 

The modelling of the \alphat variable is probed through a dedicated set of closure tests, described in Section (\ref{subsec:sysuncertainties}), which demonstrate that the different \alphat acceptances for the control and signal samples have no significant systematic bias on the prediction.


\subsection{Estimating the QCD Background Multi-jet Background}
\label{subsec:qcdbackground}

A negligible background from QCD multi-jet events within the hadronic signal region is expected due to the selection requirement, and additional cleaning filters applied. However a conservative approach is still adopted and the likelihood model (see Section (\ref{sec:resultsintro})), is given the freedom to estimate any potential QCD multi-jet contamination. 

Any potential contamination can be identified through the variable $R_{\alphat}$, defined as the ratio of events above and below the \alphat threshold value used in the analysis. This is modelled by a \theht dependant falling exponential function which takes the form,

\begin{equation}
R_{\alphat}(\theht) =  A \exp^{-k_{QCD}\theht},
\end{equation}

where the parameters A and $k_{QCD}$ are the normalisation and exponential decay constants respectively. 

For QCD event topologies this exponential behaviour is expected as a function of \theht for several reasons. The improvement of jet energy resolution at higher \theht due to higher \pt jets leads to a narrower peaked distribution, causing $R_{\alphat}$ to fall. Similarly at higher \theht values $>$ 375 \GeV, the jet multiplicity rises slowly with \theht. As shown in Figure \ref{fig:fullalphatdistribution}, at higher jet multiplicities, the result of the combinatorics used in the determination of \alphat, also lead to a narrower \alphat distribution. 

The value of the decay constant $k_{QCD}$ is constrained via measurements within data sidebands to the signal region. This is also done to validate the falling exponential assumption for QCD multi-jet topologies. The sidebands are enriched in QCD multi-jet background and defined as regions where \alphat is relaxed or that the $R_{miss}$ cut is inverted. Figure \ref{fig:qcdcartoon} depicts the definition of these data sidebands used to constrain the value of $k_{QCD}$.

\begin{minipage}{\linewidth}
\centering
\includegraphics[width = 3.5in]{plots/qcd_cartoon.pdf}
\captionof{figure}[QCD sideband regions, used for determination of $k_{QCD}$.]{QCD sideband regions, used for determination of $k_{QCD}$.}
\label{fig:qcdcartoon}
\end{minipage}

The fits to determine the value of $k_{QCD}$ are shown in Appendix (\ref{app:kqcd}), for which the best fit value obtained from sideband region B is determined to be $k_{QCD} = 2.96 \pm 0.64 \times 10^{-2}$ \GeV$^{-1}$. 

The best fit values of the remaining three C sideband regions are used to estimate the systematic uncertainty on the central value obtained from sideband region B. The variation of these measured values is used to determine the error on the determined central value, and is calculated to be $1.31 \pm 0.26 \times 10^{-2} \GeV^{-1}$. This relative error of $\sim$ 20\% gives an estimate of the systematic uncertainty of the measurement to be applied to $k_{QCD}$.

Finally the same procedure is performed for sideband region D to establish that the value of $k_{QCD}$ extracted from a lower \alphat slice can be applied to the signal region \alphat $>$ 0.55. The likelihood fit is performed across all \theht bins within the QCD enriched region with no constraint applied to $k_{QCD}$. The resulting best fit value for $k_{QCD}$ shows good agreement between that and the weighted mean determined from the three C sidebands regions. This demonstrates that the assumption of using the central value determined from sideband region B, to provide an unbiased estimator for $k_{QCD}$ in the signal region (\alphat $>$ 0.55) is valid.

Table \ref{tab:kqcdresults}, summarises the best fit $k_{QCD}$ values determined for each of the sideband regions to the signal region.

\begin{table}[h!]
\begin{center}
\begin{tabular*}{0.5\textwidth}{@{\extracolsep{\fill}}ccc}
\cline{1-3}
Sideband region & $k_{QCD}$($\times 10^{-2} \GeV^{-1})$ & $p-$value\\ 
\cline{1-3}
B & 2.96 $\pm$ 0.64 & 0.24 \\
C$_{1}$ & 1.19 $\pm$ 0.45 & 0.93 \\
C$_{2}$ & 1.47 $\pm$ 0.37 & 0.42 \\
C$_{3}$ & 1.17 $\pm$ 0.55 & 0.98 \\
\cline{1-3}
C(weighted mean) & 1.31 $\pm$ 0.26 & - \\
D(likelihood fit) & 1.31 $\pm$ 0.09 & 0.57 \\
\cline{1-3}
\end{tabular*}
\end{center}
\caption[Best fit values for the parameters $k_{QCD}$ obtained from sideband regions B,C$_{1}$,C$_{2}$,C$_{3}$. ]{Best fit values for the parameters $k_{QCD}$ obtained from sideband regions B,C$_{1}$,C$_{2}$,C$_{3}$. The weighted mean is determined from the three measurements made within sideband region C. The maximum likelihood value of $k_{QCD}$ given by the simultaneous fit using sideband region D. Quotes errors are statistical only. }
\label{tab:kqcdresults}
\end{table}


\section{Trigger Strategy}
\label{subsec:triggerstrategy}

A cross trigger based on the quantities \theht and \alphat, labelled is used with varying thresholds across \theht bins to record the events used in the hadronic signal region. The \alphat legs of the \htalphat triggers used in the analysis are chosen to fully suppress QCD multi-jet events, whilst maintaining a sustainable trigger rate. To further maintain an acceptable rate for these analysis specific triggers, only calorimeter information is used in the reconstruction of the \theht sum, leading to the necessity for Calo jets to be used within the analysis. 

A single object prescaled \theht trigger is used to collect events for the hadronic control region described above in Section (\ref{subsec:qcdbackground}).

The performance of the \alphat and \theht triggers used to collect data for the signal and hadronic control region is measured with respect to a reference sample collected using the muon system. This allows measurement of both the Level 1 seed and higher level triggers simultaneously, as the reference sample is collected independent of any jet requirements. 

The selection for the trigger efficiency measurement is identical to that described in Section (\ref{subsec:eventselection}), with the requirement of exactly one well identified muon with \pt $>$ 30 \GeV which is subsequently ignored.  

The efficiencies measure for the \htalphat triggers in bins indiviual \theht and \alphat legs, is summarised in Table \ref{tab:trigeffs}.

\begin{table}[h!]
\begin{center}
\begin{tabular*}{0.5\textwidth}{@{\extracolsep{\fill}}ccc}
\cline{1-3}
\theht range (\GeV) & $\epsilon$ on \theht leg (\%) & $\epsilon$ on \alphat leg (\%) \\ 
\cline{1-3}
275-325 & $87.7^{+1.9}_{-1.9}$ & $82.8^{+1.0}_{-1.1}$ \\
325-375 & 90.6$^{+2.9}_{-2.9}$ & 95.9$^{+0.7}_{-0.9}$ \\
375-475 & 95.7$^{+0.1}_{-0.1}$ & 98.5$^{+0.5}_{-0.9}$ \\
475-$\infty$ & 100.0$^{+0.0}_{-0.0}$ & 100.0$^{+0.0}_{-4.8}$ \\
\cline{1-3}
\end{tabular*}
\end{center}
\caption[Measured efficiencies of the \theht and \alphat legs of the HT and \htalphat triggers in independent analysis bins.]{Measured efficiencies of the \theht and \alphat legs of the HT and \htalphat triggers in independent analysis bins. The product of the two legs gives the total efficiency of the trigger in a given offline \theht bin.}
\label{tab:trigeffs}
\end{table}

Data for the control samples of the analysis, detailed in Section (\ref{subsec:controlsampledefinition}), are collected using single object photon trigger for the \gpjets sample, and a single object muon trigger for both the \mupjets and \dimupjets control samples. The photon trigger is measured to be full efficient for the threshold $\pt^{photon} > 150 \GeV$, whilst the single muon efficiency satisfying $\pt^{muon} > 30 \GeV$ is measured to have an efficiency of (88$\pm$2)\% that is independent of \theht. In the case of the \dimupjets control sample, the efficiency is measured to be (95$\pm$2)\% for the lowest \theht bin, rising to (98$\pm$2)\% for the highest \theht bin.

\section{Measuring MC normalisation factors via \theht sidebands}
\label{subsec:mckfactors}

The theoretical cross sections of different \ac{SM} processes at \acf{NNLO} and the number of MC simulated events generated for that particular process, is typically used to determine the appropriate normalisation for a MC sample. However within the particular high-\theht and high-\met corners of kinematic phase space probed within this search, the theoretical cross sections for various processes are far less well understood. 

To mitigate the problem of theoretical uncertainties and arbitrary choices of cross sections, the normalisation of MC samples used in the analysis are determined through the use data sidebands. The sidebands are used to calculated sample specific correct factors (k-factors) that are appropriate for the \theht-\met phase space coverd by this analysis. 

They are defined within the \mupjets and \dimupjets control sample, by the region 200$<$ \theht$<$275, using the same jet \pt thresholds as the adjacent first analysis bin. Individual \ac{EWK} processes are isolated within each of these control samples via requirements on jet multiplicity and the requirement on b-tags, summarised in Table \ref{tab:mckfactors}. The purity of the samples are typically $>$ 90\% with any residual contamination corrected for. The resultant k-factor for each process is determined by then taking ratio of the data yield over the MC expectation in the sideband. Subsequently these k-factors are then applied to the processes within the phase space of the analysis.

 \begin{table}[h!]
\begin{center}
\begin{tabular*}{0.95\textwidth}{@{\extracolsep{\fill}}llccc}
\cline{1-5}
Process & Selection & Observation & MC expectation & k-factor \\
\cline{1-5}
W + jets & \mupjets, n$_{b}$=0, n$_{jet}$ = 2,3 &26950 & 29993.2 $\pm$ 650.1 & 0.90 $\pm$ 0.02 \\
$Z \rightarrow \mu\mu$ + jets & \dimupjets, n$_{b}$=0, n$_{jet}$ = 2,3 & 3141 & 3402.0 $\pm$ 43.9 & 0.92 $\pm$ 0.02 \\
\ttbar & \mupjets, n$_{b}$=2, n$_{jet}$ = $\geq$4 & 2190 & 1967.8 $\pm$ 25.1 & 1.11 $\pm$ 0.02 \\
\cline{1-5}
\end{tabular*}
\end{center}
\caption[k-factors calculated for different \ac{EWK} processes.]{k-factors calculated for different \ac{EWK} processes. All k-factors are derived relative to theoretical cross sections calculated in \ac{NNLO}. The k-factors measured for the Z$\rightarrow \mu\mu$ + jets processes, are also applied to the \zinv + jets and \gpjets MC samples.}\label{tab:mckfactors}
\end{table}


\section{Determining MC Yields With Higher Statistical Precision}
\label{subsec:backgroundestimation}

Reconstructing events from \ac{EWK} processes with many b-tagged jets ($\geq$3),\nbreco  ,is largely driven by the mis-tagging of light jets within the event. This is clear when considering the main \ac{EWK} backgrounds in the analysis, such as \ttbar + jets events, which typically contain two b-flavoured jets from the decay of the top quarks, whilst W + jets and Z$\rightarrow \mu\mu$ + jets events will typically contain no b-flavoured jets.

When the expectation for the number of \nbreco is taken directly from simulation, the statistical uncertainty at large b-tag multiplicities becomes relatively large. In order to reduce this uncertainty one approach is to use the information encoded throughout all events in the simulation sample, to measure each of the four ingredients:

\begin{enumerate}
\item the b-tagging efficiency in the event selection,
\item the charm-tagging efficiency in the event selection
\item the mis-tagging rate in the event selection,
\item the underlying flavour distribution of the jets in the events,
\end{enumerate}

 that determine the \nbreco distribution of the process being measured. This method allows the determination of higher b-tag multiplicities to a higher degree of accuracy reducing the statical uncertainties of the MC which enter into the \ac{TF}'s. For the discussion that follows, these predictions are determined on average (i.e not on an event-by-event basis), and is known as the formula method.

\subsection{The formula method}
\label{subsec:formulamethod}

The assigning of jet flavours to reconstruction level jets in simulation is achieved via an algorithmic method defined as:

\begin{itemize}
\item Try to find the parton that most likely determines the properties of the jet and assign that flavour as true flavour,
\item Here, the ``final state'' partons (after showering, radiation) are analysed (also within $\Delta R <$ 0.3 of reconstructed jet cone),
\item Jets from radiation are matched with full efficiency,
\item If there is a b/c flavoured parton within the jet cone: label as b/c flavoured jet,
\item Otherwise: assign flavour of the hardest parton.
\end{itemize}

Within each individual MC process and each \theht-$n_{jet}$ bin in the analysis, the \nbreco distribution is constructed in the following way:

 Let \nbcq represent the yield in simulation of events with \textit{b} underlying b-quarks, \textit{c} underlying c-quarks and \textit{q} underlying light quarks which are matched to reconstructed jets. Light quarks are defined as those which originate from a \textit{u},\textit{d},\textit{s},\textit{g} and $\tau$ jets which are grouped together having similar mis-tagging rates.  Similarly defining \eff, \ceff and \textit{m}, which represent the measured b-tagging,c-tagging and mis-tagging efficiency averaged over all the jets within that particular analysis bin. 
 
 Using this information the expected number of jets which have been b-tagged can be analytically calculated using the formula :

\begin{align}
\label{eq:btagformula}
N(n_{b}) =& \sum_{n_{b}^{gen}+n_{c}^{gen}+n_{q}^{gen} = n_{jet}} \quad \sum_{n_{b}^{tag}+n_{c}^{tag}+n_{q}^{tag} = n_{b}} \nbcq \times \probb \times \nonumber \\
& \probc \times \probl,
\end{align}

with N(n$_{b}$) representing the event yield where $n_{b}$ jets have been b-tagged, $n_{b}^{tag}$, $n_{c}^{tag}$ and $n_{q}^{tag}$ represent the number of times that a particular jet flavour results in a b-tagged jet, and \probb,\probc and \probl represent the binomial probabilities for that to happen. 

This approach ultimately results in a more precise \nbreco distribution prediction as information from throughout the entire MC sample is used to estimate the high $n_{b}^{reco}$ bins.

\subsection{Establishing proof of principle}
\label{subsec:formulamethodsanity}

In order to validate the procedure, the predictions obtained from the formula method summarised in Eq (\ref{eq:btagformula}), are compared directly to those obtained directly from simulation. These results for the \mupjets control sample are summarised in Table \ref{tab:sanitycheck}, for the 0,1,2 and 3 $n_{b}^{reco}$ bins.  

 \begin{table}[h!]
\begin{center}
\begin{tabular*}{0.95\textwidth}{@{\extracolsep{\fill}}llccc}
\cline{1-5}
Process & Selection & Observation & MC expectation & k-factor \\
\cline{1-5}
\end{tabular*}
\end{center}
\caption[place holder]{place holder}\label{tab:sanitycheck}
\end{table}

\subsection{Correcting Measured Efficiencies In Simulation To Data}
\label{subsec:formulamethodsf}

As detailed in Section (\ref{subsec:cmsobjects-btagging}), it is necessary for certain \pt and $\eta$ dependant corrections, to be applied to both the b-tagging efficiency and mis-tagging rates in order correct the efficiencies from simulation to the distributions seen in data. These corrections are factored in�.

Show plot of before and after correction to btag/mistag rate.


These corrections come with uncertainties�..

show plot of effect of scaling correction factor up and down.
2
\section{Systematic Uncertainties On Transfer Factors}
\label{subsec:sysuncertainties}

Since the \ac{TF}'s used to establish the background prediction are obtained from simulation, an appropriate systematic uncertainty is assigned to each factor to account for theoretical uncertainties \cite{Bern:2011pa} and limitations in the simulation modelling of event kinematics and instrumental effects. 

The magnitudes of these systematic uncertainties are established through a set of data driven method, in which the three independent control samples of the analysis (\mupjets, \dimupjets, \gpjets) are used to in a series of closure tests. The yields from one of these control samples, along with the corresponding \ac{TF} obtained from simulation, are used to predict the yields in another control sample, using the same method of establishing a background prediction for the signal region as described in Section (\ref{subsec:controlsampledefinition}).

The level of agreement between the predicted and observed yields is expressed as the ratio 

\begin{equation}
\label{eq:closuretests}
\frac{(N_{obs}-N_{pred})}{N_{pred}},
\end{equation}

while considering only the statistical uncertainties on $N_{pred}$, the prediction, and $N_{obs}$, the observation. No systematic uncertainty is assigned to the prediction, and resultantly the level of closure is defined by the statistical significance of a deviation from the ratio from zero.

This ratio is measured for each \theht bin in the analysis, allowing these closure tests to be sensitive to both the presence of any significant biases or any possible \theht dependence on the level of closure.

Eight sets of closure tests are defined between the three data control samples, conducted independently between the two jet multiplicity (2 $\leq n_{jets} \leq 3$, $n_{jet} \geq 4$ ) bins. Each of these tests are specifically chosen to probe each of the different key ingredients of the simulation modelling that can affect the background prediction.

Each of the different modelling components and the relevant closure tests are described below :

\begin{itemize}

\item[] \textbf{\alphat modelling}

The modelling of the \alphat distribution in genuine \met events is probed with the \mupjets control sample. This test is important to verify the approach of remove the \alphat $>$ 0.55 requirement from the \mupjets and \dimupjets samples to increase the precision of the background prediction. The test uses the \mupjets sample without an \alphat cut to make a prediction into the \mupjets sample defined with the requirement  \alphat $>$ 0.55.

\item[] \textbf{Background admixture}

The sensitivity of the translation factors to the relative admixture of events from $W +$ jets and \ttbar processes is probed by two closure tests. These tests represent an extremely conservative approach as the admixture of the background remains similar between the \mupjets sample and the signal region, contrary to the defined closure tests which make predictions between two very different admixtures of $W +$ jets and \ttbar events.  

Within the \mupjets sample, a W boson enriched sub-sample ($n_{b} =$ 0) is used to predict yields in a \ttbar enriched sub-sample ($n_{b} =$ 1). Similarly the \\tbar enriched sub-sample ($n_{b} =$1) is also used to predict yields for a further enriched \ttbar sub-sample ($n_{b} =$ 2). 

Similarly a further closure test probes the relative contribution of $Z +$ jets to $W +$jets and \ttbar events, through the use of the \mupjets sample to predict yields for the \dimupjets control sample. This closure test, also at some level probes the muon trigger and reconstruction efficiencies, given that exactly one and two muons are required by the different selections.
 
\item[] \textbf{Consistency between control samples}

An important consistency check between the \dimupjets jets and \gpjets, which are both used in the prediction of the \zinv in the signal region, is measured by using the \gpjets sample to predict yields for the \dimupjets control sample.

\item[]\textbf{Modelling of jet multiplicity}

The simulation modelling of the jet multiplicity within each control sample is important due to the exclusive jet multiplicity binning within the analysis. This is probed via the use of each of the three control samples to independently predict from the lower jet multiplicity category $2 \leq n_{jet} \leq 3$, to the high jet category $\geq 4$. 

For the case of the \mupjets and \dimupjets control samples this test this is also a further probe of the admixture between $W +$ jets/$Z +$ jets and \ttbar. 
\end{itemize}

To test for the assumption that no \theht dependences exist within the background predictions of the analysis, the first five closure tests defined above are taken, with zeroeth and first order polynomial fits are applied to each. This is summarised in Table \ref{tab:closuretestfitslow} and Table \ref{tab:closuretestfitshigh} which show the results for both the 2 $\leq n_{jet} \leq 3$ and $\geq 4$ jet multiplicity bins respectively.

 \begin{table}[h!]
\begin{center}
\begin{tabular*}{0.95\textwidth}{@{\extracolsep{\fill}}ll|cc|cc}
\cline{1-6}
& & \multicolumn{2}{c|}{\footnotesize{Constant fit}} & \multicolumn{2}{c}{\footnotesize{Linear fit}} \\ 
\footnotesize{Closure test} & \footnotesize{Symbol} & \footnotesize{Best fit value} & \footnotesize{p-value} & \footnotesize{Slope (10$^{-4}$)} & \footnotesize{p-value} \\
\cline{1-6}
\footnotesize{\alphat $< 0.55 \rightarrow \alphat > 0.55$ (\mupjets)} & \footnotesize{Circle} & $-0.06 \pm 0.02$ & 0.93 & $-1.3 \pm 2.2$ & 0.91 \\ 
\footnotesize{0 b-jets $\rightarrow$ 1 b-jet (\mupjets)} & \footnotesize{Square} & $ \footnotesize{\quad}0.07 \pm 0.02$ & 0.98 & $-1.6 \pm 1.6$ & 1.00 \\ 
\footnotesize{1 b-jets $\rightarrow$ 2 b-jet (\mupjets)} & \footnotesize{Triangle} & $ -0.07 \pm 0.03$ & 0.76 & $-2.7 \pm 3.0$ & 0.76 \\ 
\footnotesize{\mupjets $\rightarrow$ \dimupjets} & \footnotesize{Cross} & $ \footnotesize{\quad}0.10 \pm 0.03$ & 0.58 & $-1.1 \pm 2.3$ & 0.49 \\ 
\footnotesize{\dimupjets $\rightarrow$ \gpjets} & \footnotesize{Star} & $ -0.06 \pm 0.04$ & 0.31 & $\footnotesize{\quad}4.2 \pm 4.3$ & 0.29 \\ \cline{1-6}
\end{tabular*}
\end{center}
\caption[A summary of the results obtained from fits of zeroeth order polynomials (i.e. a constant) to five sets of closure tests performed in the 2 $\leq n_{jet} \leq$ 3 bin]{A summary of the results obtained from fits of zeroeth order polynomials (i.e. a constant) to five sets of closure tests performed in the 2 $\leq n_{jet} \leq$ 3 bin. The final two columns show the best fit value for the slope obtained when performing a linear fit and the p-value for the linear fit.}\label{tab:closuretestfitslow}
\end{table}

 \begin{table}[h!]
\begin{center}
\begin{tabular*}{0.95\textwidth}{@{\extracolsep{\fill}}ll|cc|cc}
\cline{1-6}
& & \multicolumn{2}{c|}{\footnotesize{Constant fit}} & \multicolumn{2}{c}{\footnotesize{Linear fit}} \\ 
\footnotesize{Closure test} & \footnotesize{Symbol} & \footnotesize{Best fit value} & \footnotesize{p-value} & \footnotesize{Slope (10$^{-4}$)} & \footnotesize{p-value} \\
\cline{1-6}
\footnotesize{\alphat $< 0.55 \rightarrow \alphat > 0.55$ (\mupjets)} & \footnotesize{Circle} & $-0.05 \pm 0.03$ & 0.21 &  $\footnotesize{\quad}3.0 \pm 2.9$ & 0.21 \\ 
\footnotesize{0 b-jets $\rightarrow$ 1 b-jet (\mupjets)} & \footnotesize{Square} & $ -0.03 \pm 0.03$ & 0.55 & $-1.0 \pm 1.9$ & 0.47 \\ 
\footnotesize{1 b-jets $\rightarrow$ 2 b-jet (\mupjets)} & \footnotesize{Triangle} & $ -0.02 \pm 0.03$ & 0.39 & $ \footnotesize{\quad}1.1 \pm 2.2$ & 0.31 \\ 
\footnotesize{\mupjets $\rightarrow$ \dimupjets} & \footnotesize{Cross} & $  \footnotesize{\quad}0.08 \pm 0.07$ & 0.08 &  $\footnotesize{\quad}4.8 \pm 4.3$ & 0.07 \\ 
\footnotesize{\dimupjets $\rightarrow$ \gpjets} & \footnotesize{Star} & $ -0.03 \pm 0.10$ & 0.72 & $-4.0 \pm 7.0$ & 0.64 \\ \cline{1-6}
\end{tabular*}
\end{center}
\caption[A summary of the results obtained from fits of zeroeth order polynomials (i.e. a constant) to five sets of closure tests performed in the $n_{jet} \geq$ 4 bin]{A summary of the results obtained from fits of zeroeth order polynomials (i.e. a constant) to five sets of closure tests performed in the $ n_{jet} \geq$ q bin. The final two columns show the best fit value for the slope obtained when performing a linear fit and the p-value for the linear fit.}\label{tab:closuretestfitshigh}
\end{table}

Table \ref{tab:closuretestfitsall} shows the same fits applied to the three closure tests that probe the modelling between the different $n_{jet}$ bins. The best fit value and its uncertainty is listed for each set of closure tests in all three tables, along with the p-value of the constant and linear fits applied. 

 \begin{table}[h!]
\begin{center}
\begin{tabular*}{0.95\textwidth}{@{\extracolsep{\fill}}ll|cc|cc}
\cline{1-6}
& & \multicolumn{2}{c|}{\footnotesize{Constant fit}} & \multicolumn{2}{c}{\footnotesize{Linear fit}} \\ 
\footnotesize{Closure test} & \footnotesize{Symbol} & \footnotesize{Best fit value} & \footnotesize{p-value} & \footnotesize{Slope (10$^{-4}$)} & \footnotesize{p-value} \\
\cline{1-6}
\footnotesize{\mupjets} & \footnotesize{Inverted triangle} & $-0.03 \pm 0.02$ & 0.02 & $\footnotesize{\quad}0.0 \pm 1.0$ & 0.01 \\ 
\footnotesize{\mupjets (outlier removed)} & \footnotesize{Inverted triangle} & $-0.04 \pm 0.01$ & 0.42 & $-1.4 \pm 1.1$ & 0.49 \\ 
\footnotesize{\gpjets} & \footnotesize{Diamond} & $  \footnotesize{\quad}0.12 \pm 0.05$ & 0.79 & $\footnotesize{\quad}6.0 \pm 4.7$ & 0.94 \\ 
\footnotesize{\dimupjets} & \footnotesize{Asterisk} & $ -0.04 \pm 0.07$ & 0.20 &  $\footnotesize{\quad}4.9 \pm 4.4$ & 0.20 \\ 
\cline{1-6}
\end{tabular*}
\end{center}
\caption[A summary of the results obtained from fits of zeroeth order polynomials (i.e. a constant) to five sets of closure tests performed in the 2 $\leq$ njet $\leq$ 3 bin]{A summary of the results obtained from fits of zeroeth order polynomials (i.e. a constant) to five sets of closure tests performed in the 2 $\leq$ njet $\leq$ 3 bin. The final two columns show the best fit value for the slope obtained when performing a linear fit and the p-value for the linear fit.}\label{tab:closuretestfitsall}
\end{table}

The best fit value for the constant parameter is indicative of the level of closure, averaged across the full range of \theht bins in the analysis, and the p-value an indicator of any significant dependence on \theht within the closure tests. The best fit values of all the tests are either statistically compatible with zero bias (i.e, less than $2\sigma$ from zero) or at the level of 10\% or less, with the exception of one closure test discussed below. 

Within Table \ref{tab:closuretestfitsall}, there exists one test that does not satisfy the above statement, which is the $2 \leq n_{jet} \leq 3 \rightarrow n_{jet} \geq 4$ test using the \mupjets control sample. The low p-value can be largely attributed to an outlier in the 675 $<$ \theht $<$ 775 \GeV bin, rather than any significant trend in \theht. Removing this single outlier from the constant fit performed, gives a best fit value of $-0.04 \pm 0.01$, $\chi^{2} /$ d.o.f = 6.07/6. and a p-value of 0.42. These modified fit results are included within Table \ref{tab:closuretestfitsall} .

In addition the best fit values for the slope terms of the linear fits in all three tables are of the order $10^{-4}$, which corresponds to a percent level change per 100 \GeV. However in all cases, the best fit values are fully compatible with zero (within 1$\sigma$) once again with the exception detailed above, indicating that the level of closure is \theht independent.

\subsection{Determining systematic uncertainties from closure tests}
\label{subsec:determinesystematics}

Once it has been established that no significant bias or trend has been exist within the closure tests, systematic uncertainties are determined. The statistical precision of the closure tests is considered a suitable benchmark for determining the systematic uncertainties that are assigned to the \ac{TF}'s, which are propagated through to the likelihood fit.

The systematic uncertainty band is split into five separate regions of \theht :

\begin{enumerate}
\singlespacing
\item 275 $<$ \theht $<$ 325\GeV
\item 325 $<$ \theht $<$ 375\GeV
\item 375 $<$ \theht $<$ 575\GeV
\item 575 $<$ \theht $<$ 775\GeV
\item \theht $>$ 775 \GeV
\end{enumerate}
 \onehalfspacing

Within each region the square root of the sample variance, $\sigma^{2}$, is taken over the eight closure tests to determine the systematic uncertainties to be applied within that region.

Using this procedure the systematic uncertainties for each region are calculated and are shown in Table \ref{tab:sysuncert}, with the systematic uncertainty to be used in the likelihood model conservatively rounded up to the nearest decile, shown in brackets.

 \begin{table}[h!]
\begin{center}
\begin{tabular*}{0.95\textwidth}{@{\extracolsep{\fill}}lcc}
\cline{1-3}
\theht band (\GeV)& $2 \leq n_{jet} \leq 3$ & $n_{jet} \geq 4$ \\
\cline{1-3}
275 $<$ \theht $<$ 325 & 6 (10)\%  & 3 (10)\% \\
325 $<$ \theht $<$ 375& 6 (10)\%  & 6 (10)\% \\
375 $<$ \theht $<$ 575& 7 (10)\%  & 9 (10)\% \\
575 $<$ \theht $<$ 775& 13 (20)\%  & 15 (20)\% \\
\theht $>$ 775& 19 (20)\%  & 21 (30)\% \\

\cline{1-3}
\end{tabular*}
\end{center}
\caption[Calculated systematic uncertainties for the five \theht regions, determined from the closure tests. ]{Calculated systematic uncertainties for the five \theht regions, determined from the closure tests. Uncertainties shown for both jet multiplicity categories. Values used within the likelihood model are conservatively rounded up to the nearest decile and shown in brackets.}\label{tab:sysuncert}
\end{table}

Figure \ref{fig:uncertaintyplots} shows the sets of closure tests overlaid on top of grey bands that represent the \theht dependent systematic uncertainties. These systematic uncertainties are assumed to fully uncorrelated between the different $n_{b}$ multiplicity categories and across the five \theht regions. This can be considered a more conservative approach given that some correlations between adjacent \theht bins could be expected due to comparable kinematics.


\begin{figure}[ht]
\centering
\begin{minipage}[b]{0.85 \linewidth}
\includegraphics[width = 1.0\linewidth]{plots/syst-le3j.pdf}
(a)  
\end{minipage}
\quad
\begin{minipage}[b]{0.85\linewidth}
\includegraphics[width = 1.0\linewidth]{plots/syst-ge4j.pdf}
(b) 
\end{minipage}
\caption[Sets of closure tests overlaid on top of the systematic uncertainty used for each of the five \theht regions.]{Sets of closure tests (open symbols) overlaid on top of the systematic uncertainty used for each of the five \theht regions (shaded bands) and for the two different jet multiplicity bins:(a) $2 \leq n_{jet} \leq 3$ and (b) $n_{jet} \geq 4$.}
\label{fig:uncertaintyplots}
\end{figure}

As already referenced. These closure tests represent a conservative estimate of the systematic uncertainty in making a background perdition for the signal region. This is due to significant differences in the background composition and event kinematics between the two sub-samples used in the closure tests. This is contrary to the signal region prediction where the two sub-samples are both have a comparable background admixture and similar kinematics owing to the fact that the predictions are always made using the same ($n_{jet}$,$n_{b}$,\theht) bin.

This point is emphasised when we examine the sensitivity of the \ac{TF}'s to a change in the admixture of W + jets and \ttbar with the control and signal samples. This is accomplished by varying the cross sections of the W +jets and \ttbar by +20\% and -20\%, respectively. Figures \ref{fig:xsecvariedle3j} and \ref{fig:xsecvariedge4j} within Appendix \ref{app:backgroundestimation}, show the effect upon the closure tests for both jet multiplicity categories. Given these variations in cross sections, the level of closure is found to be significantly worse, with biases as large as $\sim$ 30\%, most apparent in the lowest \theht bins. However the \ac{TF}'s used to extrapolate from control to signal are seen to change only at the percent level by this large change in cross section, shown in Table \ref{tab:xsecvaried}.

Given the robust behaviour of the translation factors with respect to large (and opposite) variations in the W + jets and \ttbar cross sections, one can assume with confidence that any bias in the translation factors is adequately (and conservatively) covered by the systematic uncertainties used in the analysis.

\chapter{Searches For Natural SUSY With B-tag Templates.}
\label{chap:templatemethod}


Within this chapter a complimentary technique is discussed as a means to predict the distribution of three and four reconstructed b-quark jets in an event. The recent discovery of the Higgs boson has made third-generation ``Natural \ac{SUSY}'' models attractive, given that light top and bottom squarks are a candidate to stabilise divergent loop corrections to the Higgs boson mass.

Using the $\alphat$ search as a base, a simple templated fit is employed to estimate the \ac{SM} background in higher b-tag multiplicities (3-4) from a region of a low number of reconstructed b-jets (0-2). As a proof- of-concept, the procedure after being shown to close in simulation, is applied to the SM enriched \mupjets control sample of the \alphat all-hadronic search detailed in Chapter \ref{chap:SUSYsearches}. Results are presented using the \ac{CSV} tagger (introduced in Section (\ref{subsec:cmsobjects-btagging})) for the ``Loose'', ``Medium'' and ``Tight'' working points.


\section{Concept}
\label{sec:templateconcept}

The dominant \ac{SM} backgrounds most \ac{SUSY} searches are typically \ttbar + jets, W + jets and \zinv + jets. These process are characterised by typically having zero or two underlying b-quarks per event. The first step in this approach is to categorise two templates to be fitted to the low $n_{b}^{reco}$ multiplicity in terms of these underlying b-quark event topologies :

\begin{itemize}
\item[Z0 -] W + jets, \zinv + jets, DY + jets 
\item[Z2 -] \ttbar, single top
\end{itemize}

where Z0 and Z2 represent processes which have an underlying b-quark content of zero or two respectively. 

Both these templates can be generated through the application of the relevant event selection and taking the underlying $n_{b}^{reco}$ distribution directly from simulation. However as discussed within Section (\ref{subsec:backgroundestimation}), there are large uncertainties for high $n_{b}^{reco}$ multiplicities due to limited MC statistics. This is particularly prominent for the Z0 templates, where the number of reconstructed b-tags is driven primarily by the light-quark mis-tagging rate.  Therefore to improve the statistical precision of the predictions the formula method, introduced in Section (\ref{subsec:formulamethod}) is used. 

The generation of these templates is then dependant upon the jet-flavour content and b-tagging rate within the phase space of interest, with the tagging probabilities of a jet being a function of the jet \pt, the pseudo-rapidity $\rvert\eta\lvert$, and the jet-flavour. This can be observed in Figure \ref{fig:templatetaggingefficiencies}, where the b-tagging / c-quark mis-tagging / light mis-tagging efficiency for the three working points of the \ac{CSV} tagger is shown as a function of jet \pt. 

\begin{figure}[ht]
\centering
\begin{minipage}[b]{0.48 \linewidth}
\includegraphics[width = 1.0\linewidth]{plots/template_btagrate.pdf}
(a)  
\end{minipage}
\quad
\begin{minipage}[b]{0.48\linewidth}
\includegraphics[width = 1.0\linewidth]{plots/template_ctagrate.pdf}
(b) 
\end{minipage}
\quad
\begin{minipage}[b]{0.48\linewidth}
\centering
\includegraphics[width = 1.0\linewidth]{plots/template_mistagrate.pdf}
(c) 
\end{minipage}
\caption[The b-tagging (a), c-quark mis-tagging (b), and light-quark mis-tagging rate (c$)$ as measured in simulation after the \alphat analysis, \mupjets control sample selection in the region \theht $>$ 375.]{The b-tagging (a), c-quark mis-tagging (b), and light-quark mis-tagging rate (c$)$ as measured in simulation after the \alphat analysis, \mupjets control sample selection in the region \theht $>$ 375.}
\label{fig:templatetaggingefficiencies}
\end{figure}

Before the templates are generated, the relevant jet \pt and \eta corrections are applied to correct simulation to data, as specified in Section (\ref{subsec:formulamethodsf}), to then determine the average tagging rates per analysis bin.   

These two templates are then fit to data in the low $n_{b}^{reco}$ region (0-2). The fit result is used, along with the knowledge of the template shapes, to extrapolate an estimate to the high $n_{b}^{reco}$ region (3,4), which is then compared to what is observed in data.

This method can, in principle, be applied to any analysis where the signal hypothesis has a larger underlying b-quark spectra than the \ac{SM} backgrounds, as it solely relies on fitting to the shape of the $n_{b}^{reco}$ distribution.

\section{ Application to the \alphat search}
\label{sec:templateapplication}

As detailed in the previous chapter, the \alphat analysis is a search for \ac{SUSY} particles in all-hadronic final states, utilising the kinematic variable \alphat to suppress QCD to a negligible level. \ac{SM} enriched control samples are used to estimate the background within an all-hadronic signal region. 

The selection for the \mupjets control samples defined in Section (\ref{subsec:controlsampledefinition}) is used to demonstrate the template fitting procedure both conceptually in simulation, and also when applied in data. This is chosen, as such a selection is dominated by events stemming from the \ac{SM} processes with little or no signal contamination from potential new physics.. Neither are contributions from rate \ac{SM} processes with a higher underlying b-quark content (e.g. $t\bar{t}b\bar{b}$) expected. For these reasons, there is a degree of confidence that the procedure should close when applied to this phase space.

The analysis presented here is binning in source jet multiplicity bins, of 3,4 and $\geq$ 5 reconstructed jets per event (di-jet events are not included as there is no contribution to the high $n_{b}^{reco}$ region (3,4)) , in order to reduce the kinematic jet \pt dependence. Furthermore the analysis is split into three \theht regions, 

\begin{itemize}
\item 275-325 \GeV
\item 325-375 \GeV
\item $>$ 375 \GeV
\end{itemize}

contrary to the eight used within the \alphat analysis. Templates for both underlying b-quark content hypotheses are then generated for the nine defined analysis bins. 

\subsection{Proof of principle in simulation}
\label{subsec:templateclosuretest}

The to highlight the relative insensitivity of the performance of the b-tagging algorithm in the effectiveness of the procedure.


  \chapter{Searches For Natural SUSY With B-tag Templates.}
\label{chap:templatemethod}


Within this chapter a complimentary technique is discussed as a means to predict the distribution of three and four reconstructed b-quark jets in an event. The recent discovery of the Higgs boson has made third-generation ``Natural \ac{SUSY}'' models attractive, given that light top and bottom squarks are a candidate to stabilise divergent loop corrections to the Higgs boson mass.

Using the $\alphat$ search as a base, a simple templated fit is employed to estimate the \ac{SM} background in higher b-tag multiplicities (3-4) from a region of a low number of reconstructed b-jets (0-2). As a proof- of-concept, the procedure after being shown to close in simulation, is applied to the SM enriched \mupjets control sample of the \alphat all-hadronic search detailed in Chapter \ref{chap:SUSYsearches}. . To highlight the relative insensitivity of the choice of the b-tagging algorithm working points in the effectiveness of the procedure, results are presented using the \ac{CSV} tagger (introduced in Section (\ref{subsec:cmsobjects-btagging})) for the ``Loose'', ``Medium'' and ``Tight'' working points.


\section{Concept}
\label{sec:templateconcept}

The dominant \ac{SM} backgrounds most \ac{SUSY} searches are typically \ttbar + jets, W + jets and \zinv + jets. These process are characterised by typically having zero or two underlying b-quarks per event. The first step in this approach is to categorise two templates to be fitted to the low $n_{b}^{reco}$ multiplicity in terms of these underlying b-quark event topologies :

\begin{itemize}
\item[Z0 -] W + jets, \zinv + jets, DY + jets 
\item[Z2 -] \ttbar, single top
\end{itemize}

where Z0 and Z2 represent processes which have an underlying b-quark content of zero or two respectively. 

Both these templates can be generated through the application of the relevant event selection and taking the underlying $n_{b}^{reco}$ distribution directly from simulation. However as discussed within Section (\ref{subsec:backgroundestimation}), there are large uncertainties for high $n_{b}^{reco}$ multiplicities due to limited MC statistics. This is particularly prominent for the Z0 templates, where the number of reconstructed b-tags is driven primarily by the light-quark mis-tagging rate.  Therefore to improve the statistical precision of the predictions the formula method, introduced in Section (\ref{subsec:formulamethod}) is used. 

The generation of these templates is then dependant upon the jet-flavour content and b-tagging rate within the phase space of interest, with the tagging probabilities of a jet being a function of the jet \pt, the pseudo-rapidity $\rvert\eta\lvert$, and the jet-flavour. This can be observed in Figure \ref{fig:templatetaggingefficiencies}, where the b-tagging / c-quark mis-tagging / light mis-tagging efficiency for the three working points of the \ac{CSV} tagger is shown as a function of jet \pt. 

\begin{figure}[ht]
\centering
\begin{minipage}[b]{0.48 \linewidth}
\includegraphics[width = 1.0\linewidth]{plots/template_btagrate.pdf}
\centering (a)  
\end{minipage}
\quad
\begin{minipage}[b]{0.48\linewidth}
\includegraphics[width = 1.0\linewidth]{plots/template_ctagrate.pdf}
\centering (b) 
\end{minipage}
\quad
\begin{minipage}[b]{0.48\linewidth}
\centering
\includegraphics[width = 1.0\linewidth]{plots/template_mistagrate.pdf}
\centering (c) 
\end{minipage}
\caption[The b-tagging (a), c-quark mis-tagging (b), and light-quark mis-tagging rate (c$)$ as measured in simulation after the \alphat analysis, \mupjets control sample selection in the region \theht $>$ 375.]{The b-tagging (a), c-quark mis-tagging (b), and light-quark mis-tagging rate (c$)$ as measured in simulation after the \alphat analysis, \mupjets control sample selection in the region \theht $>$ 375.}
\label{fig:templatetaggingefficiencies}
\end{figure}

Before the templates are generated, the relevant jet \pt and \eta corrections are applied to correct simulation to data, as specified in Section (\ref{subsec:formulamethodsf}), to then determine the average tagging rates per analysis bin.   

These two templates are then fit to data in the low $n_{b}^{reco}$ region (0-2). The fit result is used, along with the knowledge of the template shapes, to extrapolate an estimate to the high $n_{b}^{reco}$ signal region (3,4), which is then compared to what is observed in data.

This method can, in principle, be applied to any analysis where the signal hypothesis has a larger underlying b-quark spectra than the \ac{SM} backgrounds, as it solely relies on fitting to the shape of the $n_{b}^{reco}$ distribution.

\section{ Application to the \alphat search}
\label{sec:templateapplication}

As detailed in the previous chapter, the \alphat analysis is a search for \ac{SUSY} particles in all-hadronic final states, utilising the kinematic variable \alphat to suppress QCD to a negligible level. \ac{SM} enriched control samples are used to estimate the background within an all-hadronic signal region. 

The selection for the \mupjets control samples defined in Section (\ref{subsec:controlsampledefinition}) is used to demonstrate the template fitting procedure both conceptually in simulation, and also when applied in data. This is chosen, as such a selection is dominated by events stemming from the \ac{SM} processes with little or no signal contamination from potential new physics.. Neither are contributions from rate \ac{SM} processes with a higher underlying b-quark content (e.g. $t\bar{t}b\bar{b}$) expected. For these reasons, there is a degree of confidence that the procedure should close when applied to this phase space.

The analysis presented here is binning in source jet multiplicity bins, of 3,4 and $\geq$ 5 reconstructed jets per event (di-jet events are not included as there is no contribution to the high $n_{b}^{reco}$ region (3,4)) , in order to reduce the kinematic jet \pt dependence. Furthermore the analysis is split into three \theht regions, 

\begin{itemize}
\item 275-325 \GeV
\item 325-375 \GeV
\item $>$ 375 \GeV
\end{itemize}

contrary to the eight used within the \alphat analysis. Templates for both underlying b-quark content hypotheses are then generated for the nine defined analysis bins.

\subsection{Proof of principle in simulation}
\label{subsec:templateclosuretest}

In order to demonstrate that the template procedure produces accurate predictions within simulation, the simulation samples in the analysis are firstly split into two to allow for statistically independent fits to be performed. 

By combing the relevant ingredients necessary to employ the formula method, $n_{b}^{reco}$ templates for Z = 0 and Z= 2 are generated individually for each $n_{jet}$ and \theht bin using one half of each simulation sample. A fit of these two templates is then performed in the low $n_{b}^{reco}$ (0-2) region, back to the sum of the other halves of each simulation sample in order to check that the relevant information can be recovered in the $n_{b}^{reco}$ signal region (3-4).

The fits are performed independently within each of the defined analysis bins to reduce the dependence of the shapes of these distributions on simulation. The half of the simulation sample for which the templates are fitted too, are taken directly from simulation, extending this procedure to also be a validation of the formula method to accurately estimate the $n_{b}^{reco}$ distribution. Additionally as this test is performed in simulation, the relevant corrections of the b-tagging rates between data and simulation are \emph{not} applied.  

Within Figure \ref{fig:template_closure_njet5}, the results of this fitting procedure is shown for each \ac{CSV} working point. Results are presented for the $n_{jet} \geq 5$ category, using the \mupjets control sample selection in the inclusive \theht$>$ 375 \GeV analysis bin. The grey bands represent the statistical uncertainty on the template shapes. Additional fits are shown for other $n_{jet}$ category within Appendix \ref{app:templatemc}. 

Furthermore the extrapolated fit predictions within the high $n_{b}^{reco}$ signal region, are summarised for all \theht bins and working points in Table \ref{tab:template_mctable}. 

 \begin{table}[h!]
\begin{center}
\footnotesize
\begin{tabular*}{0.95\textwidth}{@{\extracolsep{\fill}}|lccc|}
\cline{1-4}
\multicolumn{1}{|c}{\theht} & 275-325 & 325-375 & $>$375 \\
\cline{1-4}
\multicolumn{4}{c}{Loose working point} \\
\cline{1-4}
Simulation $n_{b} = 3$ & $344.0 \pm 6.8$ & $158.8 \pm 4.5$ & $324.9 \pm 6.5$ \\ 
Template $n_{b} = 3$ & $347.5 \pm 11.6$ & $162.6 \pm 4.7$ & $322.9 \pm 6.9$ \\ 
\cline{1-4}
Simulation $n_{b} = 4$ & $29.8 \pm 1.9$ & $11.1 \pm 1.1$ & $40.2 \pm 2.4$ \\ 
Template $n_{b} = 4$ & $32.6 \pm 2.0$ & $13.0 \pm 1.0$ & $37.0 \pm 1.8$ \\ 
\cline{1-4}
\multicolumn{4}{c}{Medium working point} \\
\cline{1-4}
Simulation $n_{b} = 3$ & $58.2 \pm 2.87$ & $33.3 \pm 2.1$ & $72.1 \pm 3.1$ \\ 
Template $n_{b} = 3$ & $60.1 \pm 1.9$ & $32.1 \pm 1.5$ & $70.8 \pm 2.3$ \\ 
\cline{1-4}
Simulation $n_{b} = 4$ & $1.0 \pm 0.4$ & $0.3 \pm 0.2$ & $1.5 \pm 0.4$ \\ 
Template $n_{b} = 4$ & $1.2 \pm 0.1$ & $0.4 \pm 0.1$ & $2.2 \pm 0.2$ \\ 
\cline{1-4}
\multicolumn{4}{c}{Tight working point} \\
\cline{1-4}
Simulation $n_{b} = 3$ & $58.2 \pm 2.87$ & $33.3 \pm 2.1$ & $72.1 \pm 3.1$ \\ 
Template $n_{b} = 3$ & $60.1 \pm 1.9$ & $32.1 \pm 1.5$ & $70.8 \pm 2.3$ \\ 
\cline{1-4}
Simulation $n_{b} = 4$ & $1.0 \pm 0.4$ & $0.3 \pm 0.2$ & $1.5 \pm 0.4$ \\ 
Template $n_{b} = 4$ & $1.2 \pm 0.1$ & $0.4 \pm 0.1$ & $2.2 \pm 0.2$ \\ 
\cline{1-4}
\end{tabular*}
\end{center}
\caption[Summary of the fit predictions in the $n_{b}^{reco}$ signal region for $n_{jet} =3, =4, \geq 5$. The fit region is $n_{b}^{reco}$ = 0, 1, 2 and simulation yields are normalised to an integrated luminosity of 10 fb$^{-1}$. ]{Summary of the fit predictions in the $n_{b}^{reco}$ signal region for $n_{jet} = 3, =4, \geq 5$. The fit region is $n_{b}^{reco}$ = 0, 1, 2 and simulation yields are normalised to an integrated luminosity of 10 fb$^{-1}$.. The uncertainties quoted on the template yields are purely statistical.}\label{tab:template_mctable}
\end{table}


The pull distributions for all the fits performed are compatible with a mean of zero and standard distributions, see Appendix \ref{app:templatepulldistributions}.

The good overall agreement summarised in the table validates both the formula method used to generate the templates as well as the fitting method itself. The application of this method to the same selection in data is used to demonstrate necessary control over the efficiency and mis-tagging rates.

\begin{figure}[ht]
\centering
\begin{minipage}[b]{0.55 \linewidth}
\includegraphics[width = 1.0\linewidth]{plots/template_mc_loose_njet5.pdf}
\centering (a) Loose working point : $n_{jet} \geq$  5 
\end{minipage}
\quad
\begin{minipage}[b]{0.55\linewidth}
\includegraphics[width = 1.0\linewidth]{plots/template_mc_medium_njet5.pdf}
\centering (b) Medium working point : $n_{jet} \geq$ = 5 
\end{minipage}
\quad
\begin{minipage}[b]{0.55\linewidth}
\centering
\includegraphics[width = 1.0\linewidth]{plots/template_mc_high_njet5.pdf}
\centering (c) Tight working point : $n_{jet} \geq$ 5 
\end{minipage}
\caption[The results of fitting the Z = 0 and Z = 2 templates to the $n_{b}^{reco}$ = 0, 1, 2 bins taken directly from simulation in the region \theht $>$ 375 \GeV, for the $n_{jet} \geq 5$ category.]{The results of fitting the Z = 0 and Z = 2 templates to the $n_{b}^{reco}$ = 0, 1, 2 bins taken directly from simulation in the region \theht $>$ 375 \GeV, for the $n_{jet} \geq 5$ category. The red template represents Z = 0, while the blue template represents Z = 2. Grey bands represent the statistical uncertainty of the fit. The $\chi^{2}$ parameter displayed represents the goodness of fit to the low$ n_{b}^{reco}$ (0-2) control region.}
\label{fig:template_closure_njet5}
\end{figure}

\FloatBarrier
\subsection{Results in a data control sample}
\label{subsec:templatedataresults}

The method above is now applied to the 2012 8 \TeV dataset in the \mupjets control sample, to establish the validity of this method in data. The relevant data to simulation scale factors are applied to get corrected values of the efficiency and mis-tagging rates measured in data \cite{btag8tev} \cite{btagscalefactor}. 

Figure \ref{fig:template_data_med_njet5} show the  the results of the templates derived from simulation to each of the three defined \theht bins, in the $n_{jet} \geq 5$ category for the medium working point \ac{CSV} tagger (the same working point used within the \alphat analysis).  Grey bands represent the statistical uncertainty of the fit combined in quadrature with the systematic uncertainties of varying the data to simulation scale factors up and down by their measured systematic uncertainties.  Additional fit results for the other working points are found in Appendix \ref{app:templatedata}  

\begin{figure}[ht]
\centering
\begin{minipage}[b]{0.55 \linewidth}
\includegraphics[width = 1.0\linewidth]{plots/template_data_medium_njet5_lowht.pdf}
\centering (a) $n_{jet} \geq$  5 , 275 $<$ \theht $<$ 325
\end{minipage}
\quad
\begin{minipage}[b]{0.55\linewidth}
\includegraphics[width = 1.0\linewidth]{plots/template_data_medium_njet5_midht.pdf}
\centering (b) $n_{jet} \geq$ = 5 , 325 $<$ \theht $<$ 375 
\end{minipage}
\quad
\begin{minipage}[b]{0.55\linewidth}
\centering
\includegraphics[width = 1.0\linewidth]{plots/template_data_medium_njet5_highht.pdf}
\centering (c) $n_{jet} \geq$ 5 , \theht $\geq$ 375 
\end{minipage}
\caption[The results of fitting the Z = 0 and Z = 2 templates to the $n_{b}^{reco}$ = 0, 1, 2 bins taken from data, for the $n_{jet} \geq 5$ category and medium \ac{CSV} working point.]{The results of fitting the Z = 0 and Z = 2 templates to the $n_{b}^{reco}$ = 0, 1, 2 bins taken directly from data, for the $n_{jet} \geq 5$ category and medium \ac{CSV} working point. The red template represents Z = 0, while the blue template represents Z = 2. The $\chi^{2}$ parameter displayed represents the goodness of fit to the low$ n_{b}^{reco}$ (0-2) control region.}
\label{fig:template_data_med_njet5}
\end{figure}

The numerical results and extrapolation to the $n_{b}^{reco} =$3,4 bins for all \theht and working points is shown in Table \ref{tab:template_datatable}.

 \begin{table}[h!]
\begin{center}
\footnotesize
\begin{tabular*}{0.95\textwidth}{@{\extracolsep{\fill}}|lccc|}
\cline{1-4}
\multicolumn{1}{|c}{\theht} & 275-325 & 325-375 & $>$375 \\
\cline{1-4}
\multicolumn{4}{c}{Loose working point} \\
\cline{1-4}
Data $n_{b} = 3$ & 717 & 338 & 618\\ 
Template $n_{b} = 3$ & $782.6 \pm 16.8$ & $340.6 \pm 10.2$ & $601.9 \pm 14.2$ \\ 
\cline{1-4}
Data $n_{b} = 4$ & 68 & 39 & 68 \\ 
Template $n_{b} = 4$ & $75.0 \pm 2.7$ & $27.6 \pm 1.3$ & $71.6 \pm 2.6$ \\ 
\cline{1-4}
\multicolumn{4}{c}{Medium working point} \\
\cline{1-4}
Data $n_{b} = 3$ & 124 & 73 & 137 \\ 
Template $n_{b} = 3$ & $124.3 \pm 2.3$ & $62.0 \pm 1.7$ & $121.9 \pm 2.5$ \\ 
\cline{1-4}
Data $n_{b} = 4$ & 1 & 1 & 3 \\ 
Template $n_{b} = 4$ & $2.6 \pm 0.1$ & $1.3 \pm 0.1$ & $4.0 \pm 0.1$ \\ 
\cline{1-4}
\multicolumn{4}{c}{Tight working point} \\
\cline{1-4}
Data $n_{b} = 3$ & 21 & 13 & 23 \\ 
Template $n_{b} = 3$ & $26.7 \pm 0.5$ & $11.7 \pm 0.3$ & $21.9 \pm 0.5$ \\ 
\cline{1-4}
Data $n_{b} = 4$ & 0 & 0 & 0 \\ 
Template $n_{b} = 4$ & $0.23 \pm 0.07$ & $0.09 \pm 0.04$ & $0.29 \pm 0.09$ \\ 
\cline{1-4}
\end{tabular*}
\end{center}
\caption[Summary of the fit predictions in the $n_{b}^{reco}$ signal region of the \mupjets control sample, for $n_{jet} = 3, =4, \geq 5$. The fit region is $n_{b}^{reco}$ = 0, 1, 2 using 11.5 fb$^{-1}$ of data at $\sqrt{s} = 8$\TeV.]{Summary of the fit predictions in the $n_{b}^{reco}$ signal region of the \mupjets control sample, for $n_{jet} = 3, =4, \geq 5$. The fit region is $n_{b}^{reco}$ = 0, 1, 2 using 11.5 fb$^{-1}$ of data at $\sqrt{s} = 8$\TeV. The uncertainties quoted on the template yields are purely statistical.}\label{tab:template_datatable}
\end{table}

\FloatBarrier

The agreement for all working points demonstrates a good control of the b-tagging efficiency and mis-tagging rates and gives confidence in the method outlined. 

\subsection{Application to the \alphat hadronic search region}
\label{subsec:templatedataresults}

As an accompaniment to the background estimation methods outlined by the \alphat search. The b-tag template method offers a complimentary way of estimated the background within the hadronic signal region of the search�.


\section{Conclusions}
\label{subsec:templateconclusions}

A \ac{SUSY} signature such as one from gluino-induced third-generation squark production, would result in a final state with an underlying b-quark content greater than two. In order to be able to discriminate such signatures from the \ac{SM} background, templates are generated based on a parameterisation of the number of the \ac{SM} processes, where the underlying b-quarks per event is typically zero or two. These templates are then fit to data in a low $n_{b}^{reco}$ (0-2) control region in order to extrapolate a prediction in a high $n_{b}^{reco}$ (3-4) signal region. 

The method was demonstrated both in simulation and also in data, using the \ac{SM} enriched \mupjets selection from the \alphat search, to prove conceptually and experimentally that the method works and there is adequate control over the efficiency and mis-tagging rates in data for all working points of the \ac{CSV} tagger. Additionally this method was also applied to the \alphat analysis signal region where good agreement is observed between data and the background estimation method of the \alphat analysis.


  \chapter{Results And Interpretation}
\label{chap:SUSYresults}

Using the statistical framework outlined in the previous chapter, results are compared to a \ac{SM}-only hypothesis (Section (\ref{sec:smhypothesis})) and interpreted within various \ac{SMS} models (Section (\ref{sec:resultsms})). 

\section{Compatibility with the Standard Model Hypothesis}
\label{sec:smhypothesis}

The \ac{SM} background only hypothesis is tested by removing any signal contributions within the signal and control samples, and the likelihood function is maximised over all parameters using Rootfit \cite{2010acat.confE..57M} and MINUIT \cite{James:1975dr}. The results of the search consist of the observed yields in the hadronic signal sample, and the \mupjets, \dimupjets and \gpjets control samples. 

These observed yields along with the expectations and uncertainties given by the simultaneous fit for the hadronic signal region are given in Table \ref{tab:fitsdata}. The results obtained from the simultaneous fits, including that of the three control samples, are shown in Figure \ref{fig:result0blow}-\ref{fig:result4bhigh}, as summarised in Table \ref{tab:fitresults}. 

 \begin{table}[h!]
 \footnotesize
\begin{center}
\begin{tabular*}{0.55\textwidth}{@{\extracolsep{\fill}}cclc}
\hline
$n_{jet}$ & $n_{b}^{reco}$ & Control samples fitted & Figure  \\
\hline\hline
2-3 & 0 & \mupjets,\dimupjets,\gpjets & \ref{fig:result0blow} \\
2-3 & 1 & \mupjets,\dimupjets,\gpjets & \ref{fig:result1blow} \\
2-3 & 1 & \mupjets & \ref{fig:result2blow} \\
$\geq$4 & 0 & \mupjets,\dimupjets,\gpjets & \ref{fig:result0bhigh} \\
$\geq$4 & 1 & \mupjets,\dimupjets,\gpjets & \ref{fig:result1bhigh} \\
$\geq$4 & 2 & \mupjets & \ref{fig:result2bhigh} \\
$\geq$4 & 3 & \mupjets & \ref{fig:result3bhigh} \\
$\geq$4 & 4 & \mupjets & \ref{fig:result4bhigh} \\
\hline
\end{tabular*}
\end{center}
\caption[Summary of control samples used by each fit results, and the Figures in which they are displayed.]{Summary of control samples used by each fit results, and the Figures in which they are displayed.}\label{tab:fitresults}
\end{table}

 \begin{table}[h!]
 \footnotesize
\begin{center}
\begin{tabular*}{1.0\textwidth}{@{\extracolsep{\fill}}ccccccccccc}
\hline
& &&\multicolumn{8}{c}{\theht bin (\GeV)} \\
Cat & $n_{b}^{reco}$ & $n_{jet}$ &  275-325 & 325-375 & 375-475 & 474-575 & 575-675 & 675-775 & 775-875 & 875-$\infty$ \\
\hline\hline
SM & \multicolumn{1}{c}{\multirow{2}{*}{0}} & \multicolumn{1}{c}{\multirow{2}{*}{$\leq$ 3}} & 6235$^{+100}_{-67}$ & 2900$^{+60}_{-54}$ & $1955^{+34}_{-39}$& $558^{+14}_{-15}$ & $186^{+11}_{-10}$ & $51.3^{+3.4}_{-3.8}$ & $21.2^{+2.3}_{-2.2}$ & $16.1^{+1.7}_{-1.7}$ \\
Data &  &  & 6232 & 2904 & 1965 & 552 & 177 & 58 & 16 & 25 \\
\hline
SM & \multicolumn{1}{c}{\multirow{2}{*}{0}} & \multicolumn{1}{c}{\multirow{2}{*}{$\geq$ 4}} & 1010$^{+34}_{-24}$ & 447$^{+19}_{-16}$ & $390^{+19}_{-15}$& $250^{+12}_{-11}$ & $111^{+9}_{-7}$ & $53.3^{+4.3}_{-4.3}$ & $18.5^{+2.4}_{-2.4}$ & $19.4^{+2.5}_{-2.7}$ \\
Data &  &  & 1009 & 452 & 375 & 274 & 113 & 56 & 16 & 27 \\
\hline
SM & \multicolumn{1}{c}{\multirow{2}{*}{1}} & \multicolumn{1}{c}{\multirow{2}{*}{$\leq$ 3}} & 1162$^{+37}_{-29}$ & 481$^{+18}_{-19}$ & $341^{+15}_{-16}$& $86.7^{+4.2}_{-5.6}$ & $24.8^{+2.8}_{-2.7}$ & $7.2^{+1.1}_{-1.0}$ & $3.3^{+0.7}_{-0.7}$ & $2.1^{+0.5}_{-0.5}$ \\
Data &  &  & 1164 & 473 & 329 & 95 & 23 & 8 & 4 & 1 \\
\hline
SM & \multicolumn{1}{c}{\multirow{2}{*}{1}} & \multicolumn{1}{c}{\multirow{2}{*}{$\geq$ 4}} & 521$^{+25}_{-17}$ & 232$^{+15}_{-12}$ & $188^{+12}_{-11}$& $106^{+6}_{-6}$ & $42.1^{+4.1}_{-4.4}$ & $17.9^{+2.2}_{-2.0}$ & $9.8^{+1.5}_{-1.4}$ & $6.8^{+1.2}_{-1.1}$ \\
Data &  &  & 515 & 236 & 204 & 92 & 51 & 13 & 13 & 6 \\
\hline
SM & \multicolumn{1}{c}{\multirow{2}{*}{2}} & \multicolumn{1}{c}{\multirow{2}{*}{$\leq$ 3}} & 224$^{+15}_{-14}$ & 98.2$^{+8.4}_{-6.4}$ & $59.0^{+5.2}_{-6.0}$& $12.8^{+1.6}_{-1.6}$ & $3.0^{+0.9}_{-0.7}$ & $0.5^{+0.2}_{-0.2}$ & $0.1^{+0.1}_{-0.1}$ & $0.1^{+0.1}_{-0.1}$ \\
Data &  &  & 222 & 107 & 58 & 12 & 5 & 1 & 0 & 0 \\
\hline
SM & \multicolumn{1}{c}{\multirow{2}{*}{2}} & \multicolumn{1}{c}{\multirow{2}{*}{$\geq$ 4}} & 208$^{+17}_{-9}$ & 103$^{+9}_{-7}$ & $85.9^{+7.2}_{-6.9}$& $51.7^{+4.6}_{-4.7}$ & $19.9^{+3.4}_{-3.0}$ & $6.8^{+1.2}_{-1.3}$ & $1.7^{+0.7}_{-0.4}$ & $1.3^{+0.4}_{-0.3}$ \\
Data &  &  & 204 & 107 & 84 & 59 & 24 & 5 & 1 & 2 \\
\hline
SM & \multicolumn{1}{c}{\multirow{2}{*}{3}} & \multicolumn{1}{c}{\multirow{2}{*}{$\geq$ 4}} & 25.3$^{+5.0}_{-4.2}$ & 11.7$^{+1.7}_{-1.8}$ & $6.7^{+1.4}_{-1.2}$& $3.9^{+0.8}_{-0.8}$ & $2.3^{+0.6}_{-0.6}$ & $1.2^{+0.3}_{-0.4}$ & $0.3^{+0.2}_{-0.1}$ & $0.1^{+0.1}_{-0.1}$ \\
Data &  &  & 25 & 13 & 4 & 2 & 2 & 3 & 0 & 0 \\
\hline
SM & \multicolumn{1}{c}{\multirow{2}{*}{4}} & \multicolumn{1}{c}{\multirow{2}{*}{$\geq$ 4}} & 0.9$^{+0.4}_{-0.7}$ & 0.3$^{+0.2}_{-0.2}$ & \multicolumn{6}{c}{$0.6^{+0.3}_{-0.3}$} \\
Data &  &  & 1 & 0 & \multicolumn{6}{c}{2} \\
\hline
\end{tabular*}
\end{center}
\caption[Comparison of the measured yields in the each \theht, $n_{jet}$ and $n_{b}^{reco}$ jet multiplicity bins for the hadronic sample with the \ac{SM} expectations and combined statistical and systematic uncertainties given by the simultaneous fit.]{Comparison of the measured yields in the each \theht, $n_{jet}$ and $n_{b}^{reco}$ jet multiplicity bins for the hadronic sample with the \ac{SM} expectations and combined statistical and systematic uncertainties given by the simultaneous fit.}\label{tab:fitsdata}
\end{table}

The figures show a comparison between the observed yields and the \ac{SM} expectations across all \theht bins, and in all $n_{jet}$ and $n_{b}^{reco}$ multiplicity categories. In all categories the samples are well described by the \ac{SM} only hypothesis. In particular no significant excess is observed above \ac{SM} expectation within the hadronic signal region. 

Given the lack of an excess in data hinting at a possible supersymmetric signature within the data, interpretations are made on the production masses and cross section of  a range of \ac{SUSY} decay topologies within the following section.

\begin{figure}[ht]
\footnotesize
\centering
\begin{minipage}[b]{0.48 \linewidth}
\includegraphics[width = 1.0\linewidth]{plots/hadronic_0b_le3j_logy.pdf}
\centering (a)  Hadronic sample, $2 \leq n_{jet} \leq 3$ and $n_{b}^{reco} = 0$ 
\end{minipage}
\quad
\begin{minipage}[b]{0.48\linewidth}
\includegraphics[width = 1.0\linewidth]{plots/muon_0b_le3j_logy.pdf}
\centering (b)  \mupjets sample, $2 \leq n_{jet} \leq 3$ and $n_{b}^{reco} = 0$  
\end{minipage} \\
\vspace{0.4cm}
\begin{minipage}[b]{0.48 \linewidth}
\includegraphics[width = 1.0\linewidth]{plots/mumu_0b_le3j_logy.pdf}
\centering (c$)$ \dimupjets sample, $2 \leq n_{jet} \leq 3$ and $n_{b}^{reco} = 0$ 
\end{minipage}
\quad
\begin{minipage}[b]{0.48\linewidth}
\includegraphics[width = 1.0\linewidth]{plots/photon_0b_le3j_logy.pdf}
\centering (d)  \gpjets sample, $2 \leq n_{jet} \leq 3$ and $n_{b}^{reco} = 0$ 
\end{minipage}
\caption[Comparison of the observed yields and \ac{SM} expectations given by the simultaneous fit in bins of \theht for the (a) hadronic, (b) \mupjets, (c$)$ \dimupjets and (d) \gpjets samples when requiring $n_{b}^{reco}$ = 0 and $n_{jet} \leq 3$.]{Comparison of the observed yields and \ac{SM} expectations given by the simultaneous fit in bins of \theht for the (a) hadronic, (b) \mupjets, (c$)$ \dimupjets and (d) \gpjets samples when requiring $n_{b}^{reco}$ = 0 and $n_{jet} \leq 3$. The observed event yields in data (black dots) and the expectations and their uncertainties for all SM processes (blue line with light blue bands) are shown. An example signal expectation (red solid line) for the $D1$ \ac{SMS} signal point from Table \ref{tab:sms_model_table} is superimposed on the \ac{SM} background expectation.}
\label{fig:result0blow}
\end{figure}

\begin{figure}[ht]
\footnotesize
\centering
\begin{minipage}[b]{0.48 \linewidth}
\includegraphics[width = 1.0\linewidth]{plots/hadronic_1b_le3j_logy.pdf}
\centering (a)  Hadronic sample, $2 \leq n_{jet} \leq 3$ and $n_{b}^{reco} = 1$ 
\end{minipage}
\quad
\begin{minipage}[b]{0.48\linewidth}
\includegraphics[width = 1.0\linewidth]{plots/muon_1b_le3j_logy.pdf}
\centering (b)  \mupjets sample, $2 \leq n_{jet} \leq 3$ and $n_{b}^{reco} = 1$  
\end{minipage} \\
\vspace{0.4cm}
\begin{minipage}[b]{0.48 \linewidth}
\includegraphics[width = 1.0\linewidth]{plots/mumu_1b_le3j_logy.pdf}
\centering (c$)$ \dimupjets sample, $2 \leq n_{jet} \leq 3$ and $n_{b}^{reco} = 1$ 
\end{minipage}
\quad
\begin{minipage}[b]{0.48\linewidth}
\includegraphics[width = 1.0\linewidth]{plots/photon_1b_le3j_logy.pdf}
\centering (d)  \gpjets sample, $2 \leq n_{jet} \leq 3$ and $n_{b}^{reco} = 1$ 
\end{minipage}
\caption[Comparison of the observed yields and \ac{SM} expectations given by the simultaneous fit in bins of \theht for the (a) hadronic, (b) \mupjets, (c$)$ \dimupjets and (d) \gpjets samples when requiring $n_{b}^{reco}$ = 1 and $n_{jet} \leq 3$.]{Comparison of the observed yields and \ac{SM} expectations given by the simultaneous fit in bins of \theht for the (a) hadronic, (b) \mupjets, (c$)$ \dimupjets and (d) \gpjets samples when requiring $n_{b}^{reco}$ = 1 and $n_{jet} \leq 3$. The observed event yields in data (black dots) and the expectations and their uncertainties for all SM processes (blue line with light blue bands) are shown. An example signal expectation (red solid line) for the $D2$ \ac{SMS} signal point from Table \ref{tab:sms_model_table} is superimposed on the \ac{SM} background expectation.}
\label{fig:result1blow}
\end{figure}

\begin{figure}[ht]
\footnotesize
\centering
\begin{minipage}[b]{0.48 \linewidth}
\includegraphics[width = 1.0\linewidth]{plots/hadronic_2b_le3j_logy.pdf}
\centering (a)  Hadronic sample, $2 \leq n_{jet} \leq 3$ and $n_{b}^{reco} = 2$ 
\end{minipage}
\quad
\begin{minipage}[b]{0.48\linewidth}
\includegraphics[width = 1.0\linewidth]{plots/muon_2b_le3j_logy.pdf}
\centering (b)  \mupjets sample, $2 \leq n_{jet} \leq 3$ and $n_{b}^{reco} = 2$  
\end{minipage} \\
\caption[Comparison of the observed yields and \ac{SM} expectations given by the simultaneous fit in bins of \theht for the (a) hadronic, (b) \mupjets, (c$)$ \dimupjets and (d) \gpjets samples when requiring $n_{b}^{reco}$ = 2 and $n_{jet} \leq 3$.]{Comparison of the observed yields and \ac{SM} expectations given by the simultaneous fit in bins of \theht for the (a) hadronic, (b) \mupjets, (c$)$ \dimupjets and (d) \gpjets samples when requiring $n_{b}^{reco}$ = 2 and $n_{jet} \leq 3$. The observed event yields in data (black dots) and the expectations and their uncertainties for all SM processes (blue line with light blue bands) are shown. An example signal expectation (red solid line) for the $D2$ \ac{SMS} signal point from Table \ref{tab:sms_model_table} is superimposed on the \ac{SM} background expectation.}
\label{fig:result2blow}
\end{figure}

\begin{figure}[ht]
\footnotesize
\centering
\begin{minipage}[b]{0.48 \linewidth}
\includegraphics[width = 1.0\linewidth]{plots/hadronic_0b_ge4j_logy.pdf}
\centering (a)  Hadronic sample, $n_{jet} \geq 4$ and $n_{b}^{reco} = 0$ 
\end{minipage}
\quad
\begin{minipage}[b]{0.48\linewidth}
\includegraphics[width = 1.0\linewidth]{plots/muon_0b_ge4j_logy.pdf}
\centering (b)  \mupjets sample, $n_{jet} \geq 4$ and $n_{b}^{reco} = 0$  
\end{minipage} \\
\vspace{0.4cm}
\begin{minipage}[b]{0.48 \linewidth}
\includegraphics[width = 1.0\linewidth]{plots/mumu_0b_ge4j_logy.pdf}
\centering (c$)$ \dimupjets sample, $n_{jet} \geq 4$ and $n_{b}^{reco} = 0$ 
\end{minipage}
\quad
\begin{minipage}[b]{0.48\linewidth}
\includegraphics[width = 1.0\linewidth]{plots/photon_0b_ge4j_logy.pdf}
\centering (d)  \gpjets sample, $n_{jet} \geq 4$ and $n_{b}^{reco} = 0$ 
\end{minipage}
\caption[Comparison of the observed yields and \ac{SM} expectations given by the simultaneous fit in bins of \theht for the (a) hadronic, (b) \mupjets, (c$)$ \dimupjets and (d) \gpjets samples when requiring $n_{b}^{reco}$ = 0 and $n_{jet} \geq 4$.]{Comparison of the observed yields and \ac{SM} expectations given by the simultaneous fit in bins of \theht for the (a) hadronic, (b) \mupjets, (c$)$ \dimupjets and (d) \gpjets samples when requiring $n_{b}^{reco}$ = 0 and $n_{jet} \geq 4$. The observed event yields in data (black dots) and the expectations and their uncertainties for all SM processes (blue line with light blue bands) are shown. An example signal expectation (red solid line) for the $D2$ \ac{SMS} signal point from Table \ref{tab:sms_model_table} is superimposed on the \ac{SM} background expectation.}
\label{fig:result0bhigh}
\end{figure}


\begin{figure}[ht]
\footnotesize
\centering
\begin{minipage}[b]{0.48 \linewidth}
\includegraphics[width = 1.0\linewidth]{plots/hadronic_1b_ge4j_logy.pdf}
\centering (a)  Hadronic sample, $n_{jet} \geq 4$ and $n_{b}^{reco} = 1$ 
\end{minipage}
\quad
\begin{minipage}[b]{0.48\linewidth}
\includegraphics[width = 1.0\linewidth]{plots/muon_1b_ge4j_logy.pdf}
\centering (b)  \mupjets sample, $n_{jet} \geq 4$ and $n_{b}^{reco} = 1$  
\end{minipage} \\
\vspace{0.4cm}
\begin{minipage}[b]{0.48 \linewidth}
\includegraphics[width = 1.0\linewidth]{plots/mumu_1b_ge4j_logy.pdf}
\centering (c$)$ \dimupjets sample, $n_{jet} \geq 4$ and $n_{b}^{reco} = 1$ 
\end{minipage}
\quad
\begin{minipage}[b]{0.48\linewidth}
\includegraphics[width = 1.0\linewidth]{plots/photon_1b_ge4j_logy.pdf}
\centering (d)  \gpjets sample, $n_{jet} \geq 4$ and $n_{b}^{reco} = 1$ 
\end{minipage}
\caption[Comparison of the observed yields and \ac{SM} expectations given by the simultaneous fit in bins of \theht for the (a) hadronic, (b) \mupjets, (c$)$ \dimupjets and (d) \gpjets samples when requiring $n_{b}^{reco}$ = 1 and $n_{jet} \geq 4$.]{Comparison of the observed yields and \ac{SM} expectations given by the simultaneous fit in bins of \theht for the (a) hadronic, (b) \mupjets, (c$)$ \dimupjets and (d) \gpjets samples when requiring $n_{b}^{reco}$ = 1 and $n_{jet} \geq 4$. The observed event yields in data (black dots) and the expectations and their uncertainties for all SM processes (blue line with light blue bands) are shown.}
\label{fig:result1bhigh}
\end{figure}

\begin{figure}[ht]
\footnotesize
\centering
\begin{minipage}[b]{0.48 \linewidth}
\includegraphics[width = 1.0\linewidth]{plots/hadronic_2b_ge4j_logy.pdf}
\centering (a)  Hadronic sample, $n_{jet} \geq 4$ and $n_{b}^{reco} = 2$ 
\end{minipage}
\quad
\begin{minipage}[b]{0.48\linewidth}
\includegraphics[width = 1.0\linewidth]{plots/muon_2b_ge4j_logy.pdf}
\centering (b)  \mupjets sample, $n_{jet} \geq 4$ and $n_{b}^{reco} = 2$  
\end{minipage} \\
\caption[Comparison of the observed yields and \ac{SM} expectations given by the simultaneous fit in bins of \theht for the (a) hadronic, (b) \mupjets, (c$)$ \dimupjets and (d) \gpjets samples when requiring $n_{b}^{reco}$ = 2 and $n_{jet} \geq 4$.]{Comparison of the observed yields and \ac{SM} expectations given by the simultaneous fit in bins of \theht for the (a) hadronic, (b) \mupjets, (c$)$ \dimupjets and (d) \gpjets samples when requiring $n_{b}^{reco}$ = 2 and $n_{jet} \geq 4$. The observed event yields in data (black dots) and the expectations and their uncertainties for all SM processes (blue line with light blue bands) are shown. An example signal expectation (red solid line) for the $D3$ \ac{SMS} signal point from Table \ref{tab:sms_model_table} is superimposed on the \ac{SM} background expectation.}
\label{fig:result2bhigh}
\end{figure}

\begin{figure}[ht]
\footnotesize
\centering
\begin{minipage}[b]{0.48 \linewidth}
\includegraphics[width = 1.0\linewidth]{plots/hadronic_3b_ge4j_logy.pdf}
\centering (a)  Hadronic sample, $n_{jet} \geq 4$ and $n_{b}^{reco} = 3$ 
\end{minipage}
\quad
\begin{minipage}[b]{0.48\linewidth}
\includegraphics[width = 1.0\linewidth]{plots/muon_3b_ge4j_logy.pdf}
\centering (b)  \mupjets sample, $n_{jet} \geq 4$ and $n_{b}^{reco} = 3$  
\end{minipage} \\
\caption[Comparison of the observed yields and \ac{SM} expectations given by the simultaneous fit in bins of \theht for the (a) hadronic, (b) \mupjets, (c$)$ \dimupjets and (d) \gpjets samples when requiring $n_{b}^{reco}$ = 3 and $n_{jet} \geq 4$.]{Comparison of the observed yields and \ac{SM} expectations given by the simultaneous fit in bins of \theht for the (a) hadronic, (b) \mupjets, (c$)$ \dimupjets and (d) \gpjets samples when requiring $n_{b}^{reco}$ = 3 and $n_{jet} \geq 4$. The observed event yields in data (black dots) and the expectations and their uncertainties for all SM processes (blue line with light blue bands) are shown. An example signal expectation (red solid line) for the $G2$ \ac{SMS} signal point from Table \ref{tab:sms_model_table} is superimposed on the \ac{SM} background expectation.}
\label{fig:result3bhigh}
\end{figure}

\begin{figure}[ht]
\footnotesize
\centering
\begin{minipage}[b]{0.48 \linewidth}
\includegraphics[width = 1.0\linewidth]{plots/hadronic_ge4b_ge4j_logy.pdf}
\centering (a)  Hadronic sample, $n_{jet} \geq 4$ and $n_{b}^{reco} \geq 4$ 
\end{minipage}
\quad
\begin{minipage}[b]{0.48\linewidth}
\includegraphics[width = 1.0\linewidth]{plots/muon_ge4b_ge4j_logy.pdf}
\centering (b)  \mupjets sample, $n_{jet} \geq 4$ and $n_{b}^{reco} \geq 4$  
\end{minipage} \\
\caption[Comparison of the observed yields and \ac{SM} expectations given by the simultaneous fit in bins of \theht for the (a) hadronic, (b) \mupjets, (c$)$ \dimupjets and (d) \gpjets samples when requiring $n_{b}^{reco} \geq$  4 and $n_{jet} \geq 4$.]{Comparison of the observed yields and \ac{SM} expectations given by the simultaneous fit in bins of \theht for the (a) hadronic, (b) \mupjets, (c$)$ \dimupjets and (d) \gpjets samples when requiring $n_{b}^{reco}$ $\geq$ 4 and $n_{jet} \geq 4$. The observed event yields in data (black dots) and the expectations and their uncertainties for all SM processes (blue line with light blue bands) are shown. An example signal expectation (red solid line) for the $G3$ \ac{SMS} signal point from Table \ref{tab:sms_model_table} is superimposed on the \ac{SM} background expectation.}
\label{fig:result4bhigh}
\end{figure}


\section{SUSY}
\label{sec:resultsms}

Limits are set in the parameter space of a set of \ac{SMS} models that characterise both natural \ac{SUSY} third generation squark production, and compressed spectra where the mass splitting between the particle and \ac{LSP} is small, leading to soft final state jets. However as detailed in Section (\ref{subsec:sms}), the individual models are not representative of a real physical \ac{SUSY} model as only one decay process is considered. Instead these models represent a way to test for signs of specific signatures indicating new physics. 

\subsection{The CL$_{\text{s}}$ method}

The CLs method \cite{0954-3899-28-10-313}\cite{Junk1999435}\cite{Read:451614} is used to compute the limits for signal models, with the one-sided profile likelihood ratio as the test statistic \cite{asymptotictest}.

The test statistic is defined as
\begin{equation}
  q(\mu)=\begin{cases}
    -2\text{log}\lambda(\mu) & \text{ when $\mu \geq \hat{\mu}$},\\
      \ \ 0 & \text{otherwise}.
  \end{cases}
\end{equation}

where 

\begin{equation}
\lambda(\mu) = \frac{L(\mu,\theta_{\mu})}{L(\hat{\mu},\hat{\theta})}
\end{equation}

represents the profile likelihood ratio, in which $\mu \equiv f$ from Section (\ref{subsec:signalcontribution}), is the parameter characterising the signal strength.  $\hat{\mu}$ is defined at the maximum likelihood value, $\hat{\theta}$ the set of maximum likelihood values of the nuisance parameters and $\theta_{\mu}$ the set of maximum values of the nuisance parameters for a given value of $\mu$.

When $\mu \equiv f = 1$, the signal model is considered at its nominal production cross section. The distribution of $q_{\mu}$ is built up via the generation of pseudo experiments in order to obtain two distributions for the background (B) and signal plus background (S+B) cases.

The compatibility of a signal model with observations in data is determined by the parameter CL$_{s}$,

\begin{equation}
\text{CL$_{S}$} = \frac{\text{CL$_{S+B}$}}{\text{CL$_{B}$}},
\end{equation}

with CL$_{B}$ and CL$_{S+B}$ defined as one minus the quantiles of the observed value in the data of the two distributions. A model is considered to be excluded at 95\% confidence level when CL$_{s} \leq 0.05$ \cite{2011EPJClimits}.

\subsection{Interpretation in simplified signal models}

Different \njet and \nbreco bins are used in the interpretation of different \ac{SMS} models. The choice of the categories used within each interpretation, are made to maximise the signal to background ratio, increasing sensitivity to that particular type of final state signature. The production and decay modes of the \ac{SMS} models under consideration are summarised in Table \ref{tab:susyresults}, with limit plots of the experimental reach in these models shown in Figure \ref{fig:smslimitplots}.

The models \texttt{T1} and \texttt{T2} are used to characterise the pair production of gluinos and first or second generation squarks, respectively, with parameters for the sparticle mass as well as on the \ac{LSP} mass. The low number of third generation quarks produced from this decay topology makes choosing to interpret within the \nbreco = 0 category beneficial to improving sensitivity to these models

Conversely the \texttt{T2bb}, \texttt{T1tttt}, and \texttt{T1bbbb} \ac{SMS} model describe various production and decay mechanisms in the context of third-generation squarks. In this situation considering only higher \nbreco categories, bring significant improvements to the sensitivity to these types of final state signature. 

Finally the choice of jet category is made dependant upon the production mechanism, where gluino induced and direct squark production results in a large or small number of final state jets respectively.

 \begin{table}[h!]
 \footnotesize
\begin{center}
\begin{tabular*}{1.0\textwidth}{@{\extracolsep{\fill}}llcccccc}
\hline
Model & Production/decay & $n_{jet}$ & $n_{b}^{reco}$ & Process & Limit & m$_{\tilde{q}(\tilde{g})}^{\text{best}}$ (\GeV)  & m$_{\text{LSP}}^{\text{best}}$ (\GeV) \\
\hline\hline
\texttt{T1} &  $pp \rightarrow \widetilde{g}\widetilde{g}^{*} \rightarrow q\bar{q}\widetilde{\chi}^{0}_{1}q\bar{q}\widetilde{\chi}^{0}_{1}$ & $\geq 4$ & 0 & \ref{fig:smsprocesses}(a) & \ref{fig:smslimitplots}(a) & $\sim$950 & $\sim$450 \\
\texttt{T2}  & $ pp \rightarrow \widetilde{q}\widetilde{q}^{*} \rightarrow q\widetilde{\chi}^{0}_{1}\bar{q}\widetilde{\chi}^{0}_{1}$ & $\leq 3$ & 0 & \ref{fig:smsprocesses}(b) & \ref{fig:smslimitplots}(b) & $\sim$775 &  $\sim$325 \\
\texttt{T2bb} & $ pp \rightarrow \widetilde{b}\widetilde{b}^{*} \rightarrow b\widetilde{\chi}^{0}_{1}\bar{b}\widetilde{\chi}^{0}_{1}$ & $\leq 3$ & 1,2 & \ref{fig:smsprocesses}(c$)$ & \ref{fig:smslimitplots}(c$)$ & $\sim$600 & $\sim$200\\
\texttt{T1tttt} & $ pp \rightarrow \widetilde{g}\widetilde{g}^{*} \rightarrow t\bar{t}\widetilde{\chi}^{0}_{1}t\bar{t}\widetilde{\chi}^{0}_{1}$ & $\geq 4$ & 2,3,$\geq4$ & \ref{fig:smsprocesses}(d) & \ref{fig:smslimitplots}(d) & $\sim$975 & $\sim$325 \\
\texttt{T1bbbb} & $ pp \rightarrow \widetilde{g}\widetilde{g}^{*} \rightarrow b\bar{b}\widetilde{\chi}^{0}_{1}b\bar{b}\widetilde{\chi}^{0}_{1}$ & $\geq 4$ & 2,3,$\geq4$ & \ref{fig:smsprocesses}(e) & \ref{fig:smslimitplots}(e) & $\sim$1125 & $\sim$650 \\
\hline
\end{tabular*}
\end{center}
\caption[A table representing the \ac{SMS} models interpreted within the analysis.]{A table representing the \ac{SMS} models interpreted within the analysis. The model name and production and decay chain is specified in the first two columns. Each \ac{SMS} model is interpreted in specific $n_{jet}$ and $n_{b}^{reco}$ categories which are detailed in the third and fourth columns. The last two columns indicate the search sensitivity for each model, representing the largest $m_{\widetilde{q}/\widetilde{g}}$ mass beyond which no limit can be set for this particular decay topology. The quotes values are conservatively determined from the observed exclusion based on the theoretical production cross section minus 1$\sigma$ uncertainty.}\label{tab:susyresults}
\end{table}


Experimental uncertainties on the \ac{SM} background predictions (10 $-$ 30\%, described in Section (\ref{subsec:determinesystematics})), the luminosity measurement (4.4\%), and the total acceptance times efficiency of the selection for the considered signal model (12 $-$18\%, from Section (\ref{sec:smsmodels})) are included in the calculation of the limit. 

Signal efficiency in the kinematic region defined by 0 $< m_{\widetilde{g}(\widetilde{q})} <$ 175 \GeV or $m_{\widetilde{g}(\widetilde{q})} <$ 300 \GeV is strongly affected by the presence of \acf{ISR}. This region in which direct (i.e., non-\ac{ISR} induced) production is kinematically forbidden due to the \theht $>$ 275 \GeV requirement, therefore a large percentage of signal acceptance is due to the effect of \ac{ISR} jets. Given the large associated uncertainties, no interpretation is provided for this kinematic region.

The estimates on mass limits shown in Table  \ref{tab:susyresults}, are determined conservatively from the observed exclusion based on the theoretical production cross section, minus 1$\sigma$ uncertainty. The most stringent mass limits on pair-produced sparticles are obtained at low \ac{LSP} masses and larger squark and gluino masses due to the high \pt jets and consequently high \theht of such signal topologies. The limits are seen to weaken for compressed spectra points closer to the diagonal, where the signal is populates the lower \theht bins in which more background resides. For all of the considered models, there is an \ac{LSP} mass beyond which no limit can be set, which can observed from the figures referenced in the table. 

Two small upwards fluctuations are observed within the data, and are seen at high \theht within the \nbreco = 0 category and at mid-\theht in the \nbreco = 1, 2 categories, see Table \ref{tab:fitsdata}. As each of these fluctuations occur within at least one of the analysis categories that each \ac{SMS} model interpretation is made, the observed exclusions within all \ac{SMS} models are generally found to be weaker than the expected limits in the region of 1-2 standard deviations. In isolation these fluctuations are not significant and additional data would be necessary to make any further conclusions.

Despite these fluctuations, the range of parameter space that can excluded has been extended with respect to analysis based upon the $\sqrt{s} = 7$ \TeV dataset \cite{Chatrchyan:2012wa}, by up to 225 and 150 \GeV for m$_{\tilde{q}(\tilde{g})}^{\text{best}}$ and m$_{LSP}^{\text{best}}$ respectively. The parameter space for light third generation squarks, the main tenet of natural \ac{SUSY} models, is increasingly squeezed for larger mass splitting, with exclusions in the region of 1 \TeV in these topologies. 



\begin{figure}[ht]
\footnotesize
\centering
\begin{minipage}[b]{0.48 \linewidth}
\includegraphics[width = 1.0\linewidth]{plots/t1susydecay.pdf}
\centering \caption*{(a) $\widetilde{g}\widetilde{g}^{*} \rightarrow q\bar{q}\widetilde{\chi}^{0}_{1}q\bar{q}\widetilde{\chi}^{0}_{1}$ (\texttt{T1})}\label{f`ig:t1}
\end{minipage}
\quad
\begin{minipage}[b]{0.48\linewidth}
\includegraphics[width = 1.0\linewidth]{plots/t2susydecay.pdf}
\centering \caption*{(b) $\widetilde{q}\widetilde{q}^{*} \rightarrow q\widetilde{\chi}^{0}_{1}\bar{q}\widetilde{\chi}^{0}_{1}$ (\texttt{T2})} \label{fig:t2}
\end{minipage} \\
\vspace{0.4cm}
\begin{minipage}[b]{0.48 \linewidth}
\includegraphics[width = 1.0\linewidth]{plots/t2bbsusydecay.pdf}
\centering \caption*{(c$)$ $\widetilde{b}\widetilde{b}^{*} \rightarrow b\widetilde{\chi}^{0}_{1}\bar{b}\widetilde{\chi}^{0}_{1}$ (\texttt{T2bb})} \label{fig:t2bb}
\end{minipage}
\quad
\begin{minipage}[b]{0.48\linewidth}
\includegraphics[width = 1.0\linewidth]{plots/t1ttttsusydecay.pdf}
\centering \caption*{(d) $\widetilde{g}\widetilde{g}^{*} \rightarrow t\bar{t}\widetilde{\chi}^{0}_{1}t\bar{t}\widetilde{\chi}^{0}_{1}$ (\texttt{T1tttt})} \label{fig:t2tttt}
\end{minipage}
\quad
\begin{minipage}[b]{0.48\linewidth}
\centering
\includegraphics[width = 1.0\linewidth]{plots/t1bbbbsusydecay.pdf}
\centering \caption*{(e) $\widetilde{g}\widetilde{g}^{*} \rightarrow b\bar{b}\widetilde{\chi}^{0}_{1}b\bar{b}\widetilde{\chi}^{0}_{1}$ (\texttt{T1bbbb})} \label{fig:t1bbbb}
\end{minipage}
\caption[Production and decay modes for the various \ac{SMS} models interpreted within the analysis.]{Production and decay modes for the various \ac{SMS} models interpreted within the analysis.}
\label{fig:smsprocesses}
\end{figure}

\begin{figure}[ht]
\footnotesize
\centering
\begin{minipage}[b]{0.48 \linewidth}
\includegraphics[width = 1.0\linewidth]{plots/t1.pdf}
\centering \caption*{(a) $\widetilde{g}\widetilde{g}^{*} \rightarrow q\bar{q}\widetilde{\chi}^{0}_{1}q\bar{q}\widetilde{\chi}^{0}_{1}$ (\texttt{T1})}\label{fig:t1}
\end{minipage}
\quad
\begin{minipage}[b]{0.48\linewidth}
\includegraphics[width = 1.0\linewidth]{plots/t2.pdf}
\centering \caption*{(b) $\widetilde{q}\widetilde{q}^{*} \rightarrow q\widetilde{\chi}^{0}_{1}\bar{q}\widetilde{\chi}^{0}_{1}$ (\texttt{T2})} \label{fig:t2}
\end{minipage} \\
\vspace{0.4cm}
\begin{minipage}[b]{0.48 \linewidth}
\includegraphics[width = 1.0\linewidth]{plots/t2bb.pdf}
\centering \caption*{(c$)$ $\widetilde{b}\widetilde{b}^{*} \rightarrow b\widetilde{\chi}^{0}_{1}\bar{b}\widetilde{\chi}^{0}_{1}$ (\texttt{T2bb})} \label{fig:t2bb}
\end{minipage}
\quad
\begin{minipage}[b]{0.48\linewidth}
\includegraphics[width = 1.0\linewidth]{plots/t1tttt.pdf}
\centering \caption*{(d) $\widetilde{g}\widetilde{g}^{*} \rightarrow t\bar{t}\widetilde{\chi}^{0}_{1}t\bar{t}\widetilde{\chi}^{0}_{1}$ (\texttt{T1tttt})} \label{fig:t2tttt}
\end{minipage}
\quad
\begin{minipage}[b]{0.48\linewidth}
\centering
\includegraphics[width = 1.0\linewidth]{plots/t1bbbb.pdf}
\centering \caption*{(e) $\widetilde{g}\widetilde{g}^{*} \rightarrow b\bar{b}\widetilde{\chi}^{0}_{1}b\bar{b}\widetilde{\chi}^{0}_{1}$ (\texttt{T1bbbb})} \label{fig:t1bbbb}
\end{minipage}
\caption[Upper limit of cross section at 95\% CL as a function of $m_{\widetilde{q}/\widetilde{g}}$ and $m_{LSP}$ for various \ac{SMS} models.]{Upper limit of cross section at 95\% CL as a function of $m_{\widetilde{q}/\widetilde{g}}$ and $m_{LSP}$ for various \ac{SMS} models. The solid thick black line indicates the observed exclusion region assuming \ac{NLO} and \ac{NLL}  SUSY production cross section. The analysis selection efficiency is measured for each interpreted model, with the signal yield per point given by $\epsilon \times \sigma$ 
. The thin black lines represent the observed excluded region when varying the cross section by its theoretical uncertainty. The dashed purple lines indicate the median (thick line) �1$\sigma$ (thin lines) expected exclusion regions.}
\label{fig:smslimitplots}
\end{figure}




  %% To ignore a specific chapter while working on another,
  %% making the build faster, comment it out like this:
\end{mainmatter}

%% Produce the appendices
\begin{appendices}
  %% The "\appendix" call has already been made in the declaration
%% of the "appendices" environment (see thesis.tex).

\chapter{Additional Material  for L1 Jet Performance Studies}

\section{Jet Matching Efficiencies}
\label{app:jetmatching}

The single jet turn-on curves are derived from events independent of whether the leading jet in an event is matched to a Level 1 jet using $\Delta R$ matching detailed in Section  (\ref{subsec:l1jettrigeff}).  These turn-ons are produced from events which are not triggered on jet quantities and therefore it is not guaranteed that the lead jet of an event will be seeded by a Level 1 jet. Figure \ref{fig:leadjetmatcheff} shows the particular matching efficiency of a lead jet to a L1 jet before (2012B) and after (2012C) the introduction of the \L1 jet seed requirement of 5 \GeV. 

\begin{figure}[htp]
\centering
\resizebox{.58\linewidth}{0.42\height}{\includegraphics{plots/leadjet_matchingeff.pdf}}
\caption[Leading jet matching efficiency as a function of the offline CaloJet $\et$.]{Leading jet matching efficiency as a function of the offline CaloJet $\et$, measured in an isolated muon triggered dataset in the 2012B and 2012C run periods.}
  \label{fig:leadjetmatcheff}
\end{figure}

It can be seen that the turn-on occurs at a lower $E_{T}$ during the 2012B run period before the jet seed requirement was introduced. The seed threshold requirement of a 5 \GeV jet seed introduced for run 2012C result in more events in which the lead offline jet does not have an associated L1 jet. This behaviour is expected and can be attributed to events with soft non-collimated jets in which the energy deposits are not centralised in a calorimeter region. This in turn leads to events in which the lead jets energy is spread across the 3$\times$3 calorimeter region and thus below the threshold required by the central seed region.

However, for larger jet $\et$ thresholds typical of those used by physics analyses (e.g. 100 \GeV lead jet threshold in the \alphat search), 100$\%$ efficiency is observed, and therefore this effect has no impact to overall physics performance.

A fit of an \ac{EMG} function to the matching efficiencies find mean, $\mu$, values of 6.62 GeV and 19.51 GeV for Run 2012B and 2012C respectively. This result highlights the difference at low jet \pt between the jet matching efficiencies due to the change in the jet clustering algorithm, and is shown in Table~\ref{tab:matcheff}. 

\begin{table}
\begin{center}
\begin{tabular*}{0.5\textwidth}{ccc}
\cline{1-3}
\multicolumn{1}{c}{Run Period} & $\mu$ & $\sigma$ \\ \hline\hline
2012B & 6.62 $\pm$ 0.01 & 0.79 $\pm$ 0.03  \\ 
2012C & 19.51 $\pm$ 0.03 & 7.14 $\pm$ 0.02 \\ 
\end{tabular*}
\caption[Results of a cumulative EMG function fit to the turn-on
curves for the matching efficiency of the leading jet in an event to
a Level-1 jet in run 2012C and 2012B data.]{Results of a cumulative EMG function (defined in Section \ref{subsec:l1jettrigeff}) fit to the turn-on
curves for the matching efficiency of the leading jet in an event to a Level-1 jet in run 2012C and 2012B data, measured in an isolated
muon triggered sample. The turn-on point, $\mu$, and resolution, $\sigma$, are measured with respect to offline CaloJet $\et$. Only statistical errors are quoted for each measured value.} \label{tab:matcheff}
\end{center}
\end{table}

\section{Leading Jet Energy Resolution}
\label{app:jetpuresolution}

Fits to an \acf{EMG} function (Equation \ref{eq:emg}) applied to distributions of the variable $\Delta  E_{\text{L1-Offline}}$ (Equation \ref{eq:jetresolution}), for lead jets matched to \L1 jets measured in an isolated $\mu$ triggered event sample. Each of the six plots are binned according to increasing offline lead jet energy for both Calo and PF jets. The best fit values for $\mu$ and $\sigma$ and shown in Figures \ref{fig:jetptresultspu} and \ref{fig:pfjetptresultspu} for Calo and PF jets respectively.

\begin{figure}[ht]
    \centering
    \begin{minipage}[b]{0.80\linewidth}
    \centering
   \includegraphics[width = 1.0\linewidth]{plots/calo_ptresolution_low.pdf}
   \centering (a)
     \end{minipage}
    \centering
    \begin{minipage}[b]{0.80\linewidth}
    \includegraphics[width=1.0\textwidth]{plots/calo_ptresolution_medium.pdf}
    \centering (b)
    \end{minipage}
    \quad
    \begin{minipage}[b]{0.80\linewidth}
    \includegraphics[width=1.0\textwidth]{plots/calo_ptresolution_high.pdf}
    \centering (c$)$ 
    \end{minipage}
     \caption[ Resolution plots of the leading offline \Calo $\et$ measured as a function of $\Delta E_{\text{L1-Offline}}  $ for (a) low, (b) medium, and (c) high pile-up conditions.] { Resolution plots of the leading offline jet \Calo $\et$ measured as a function of $\Delta  E_{\text{L1-Offline}}  $ for (a) low, (b) medium, and (c) high pile-up conditions as defined in Section (\ref{subsec:l1jetseedpu}). }
\end{figure}

\begin{figure}[htpb]

    \centering
    \begin{minipage}[b]{0.8\linewidth}
    \includegraphics[width=1.0\textwidth]{plots/pf_ptresolution_low.pdf}
    \centering (a)
    \end{minipage}
    \quad
    \begin{minipage}[b]{0.8\linewidth}
    \includegraphics[width=1.0\textwidth]{plots/pf_ptresolution_medium.pdf}
    \centering (b) 
    \end{minipage}
    \begin{minipage}[b]{0.8\linewidth}
    \includegraphics[width=1.0\textwidth]{plots/pf_ptresolution_high.pdf}
    \centering (c$)$ 
    \end{minipage}
      \caption[ Resolution plots of the leading off-line PF $\et$ measured as a function of $\Delta E_{\text{L1-Offline}} $ for (a) low, (b) medium, and (c) high pile-up conditions.] { Resolution plots of the leading offline jet \PF $\et$ measured as a function of  $\Delta  E_{\text{L1-Offline}}  $ for (a) low, (b) medium, and (c) high pile-up conditions as defined in Section (\ref{subsec:l1jetseedpu}). }
\end{figure}


\newpage
\section{Resolution for Energy Sum Quantities}
\label{app:jetenergysums}

The following plots show the resolution parameters for energy sum quantities as a function of the quantity (q) itself. In this case, the $\mu$, $\sigma$ and $\lambda$ fit values to an \ac{EMG} function defined by Equation (\ref{eq:emg}) for each of the individual $\Delta Q_{\text{L1-Offline}} = \frac{(\text{L1 Q } -  \text{Offline Q})}{\text{Offline Q}}$ distributions, in bins of the quantity (q) is displayed. 


\begin{figure}[h!]
  \vspace{20pt}
        \centering
        \includegraphics[width=0.74\textwidth]{plots/res_CaloHT_summary_v2.pdf}
        \caption[$\theht$~resolution parameters in bins of Calo $\theht$~measured in the defined low, medium and high pile up conditions.]{$\theht$~resolution parameters in bins of Calo $\theht$~measured for the defined low, medium and high pile-up conditions. Shown are the mean $\mu$ (left) and resolution $\sigma$ (right) fit values to an \ac{EMG} function for the $\Delta \text{HT}_{\text{L1-Offline}}$ distributions.}
        \label{fig:calohtresultspu}
\end{figure}
\begin{figure}[h!]
  \vspace{20pt}
        \centering
        \includegraphics[width=0.74\textwidth]{plots/res_pfHT_summary_v2.pdf}
        \caption[$\theht$~resolution parameters in bins of PF $\theht$~measured in the defined low, medium and high pile up conditions.]{$\theht$~resolution parameters in bins of PF $\theht$~measured for the defined low, medium and high pile-up conditions. Shown are the mean $\mu$ (left) and resolution $\sigma$ (right) fit values to an \ac{EMG} function for the $\Delta \text{HT}_{\text{L1-Offline}}$ distributions.}
        \label{fig:pfhtresultspu}
\end{figure}

\begin{figure}[h!]
  \vspace{20pt}
        \centering
        \includegraphics[width=0.74\textwidth]{plots/res_CaloMHT_summary_v2.pdf}
        \caption[$\mht$~resolution parameters in bins of Calo $\mht$~measured in the defined low, medium and high pile up conditions.]{$\mht$~resolution parameters in bins of Calo $\mht$~measured for the defined low, medium and high pile-up conditions. Shown are the mean $\mu$ (left) and resolution $\sigma$ (right) fit values to an \ac{EMG} function for the $\Delta \text{MHT}_{\text{L1-Offline}}$ distributions.}
        \label{fig:calomhtresultspu}
\end{figure}
\begin{figure}[h!]
  \vspace{20pt}
        \centering
        \includegraphics[width=0.74\textwidth]{plots/res_pfMHT_summary_v2.pdf}
        \caption[$\mht$~resolution parameters in bins of PF $\mht$~measured in the defined low, medium and high pile up conditions.]{$\mht$~resolution parameters in bins of PF $\mht$~measured for the defined low, medium and high pile-up conditions. Shown are the mean $\mu$ (left) and resolution $\sigma$ (right) fit values to an \ac{EMG} function for the $\Delta \text{MHT}_{\text{L1-Offline}}$ distributions.} \label{fig:pfmhtresultspu}
\end{figure}

\chapter{Datasets and Monte Carlo Samples}
\label{app:datasets}
The following datasets are used to populate the hadronic signal and control samples. They correspond to the full data run of 2012 and an integrated luminosity of 11.7$\pm$0.5 \fb. The official JSON from the 21st September 2012 is used to filter only certified runs and luminosity sections.

\footnotesize
\texttt{/HT/Run2012A-13Jul2012-v1/AOD} \\
\texttt{/HT/Run2012A-recover-06Aug2012-v1/AOD} \\
\texttt{/HTMHT/Run2012B-13Jul2012-v1/AOD} \\
\texttt{/HTMHT/Run2012C-24Aug2012\_v1/AOD} \\
\texttt{/HTMHT/Run2012C-PromptReco-v2/AOD} \\
\texttt{/JetHT/Run2012B-13Jul2012-v1/AOD} \\
\texttt{/JetHT/Run2012C-24Aug2012-v1/AOD} \\
\texttt{/JetHT/Run2012C-PromptReco-v2/AOD}  \\
\texttt{/SingleMu/Run2012A-13Jul2012-v1/AOD} \\
\texttt{/SingleMu/Run2012A-recover-06Aug2012-v1/AOD} \\
\texttt{/SingleMu/Run2012B-13Jul2012-v1/AOD}\\ 
\texttt{/SingleMu/Run2012C-24Aug2012\_v1/AOD}  \\
\texttt{/SingleMu/Run2012C-PromptReco-v2/AOD} \\
\texttt{/Photon/Run2012A-13Jul2012-v1/AOD} \\
\texttt{/Photon/Run2012A-recover-06Aug2012-v1/AOD} \\
\texttt{/SinglePhoton/Run2012B-13Jul2012-v1/AOD} \\
\texttt{/SinglePhoton/Run2012C-24Aug2012\_v1/AOD} \\
\texttt{/SinglePhoton/Run2012C-PromptReco-v2/AOD} \\
\normalsize
\section{Monte Carlo Samples for \ac{SM} Processes and Simplified Signal Modles}

The SM background Monte Carlo samples for physics at \com = 8 \TeV are taken from the Summer12 simulation production run with \texttt{CMSSW\_5\_3\_X} with the \texttt{PU\_S10} scenario.

\footnotesize
\texttt{/WJetsToLNu\_TuneZ2Star\_8TeV-madgraph-tarball/Summer12\_DR53X-PU\_S10\_START53\_V7A-v1/AODSIM} \\
\texttt{/WJetsToLNu\_HT-250To300\_8TeV-madgraph/Summer12\_DR53X-PU\_S10\_START53\_V7A-v1/AODSIM} \\
\texttt{/WJetsToLNu\_HT-300To400\_8TeV-madgraph/Summer12\_DR53X-PU\_S10\_START53\_V7A-v1/AODSIM} \\
\texttt{/WJetsToLNu\_HT-400ToInf\_8TeV-madgraph/Summer12\_DR53X-PU\_S10\_START53\_V7A-v1/AODSIM} \\
\texttt{/ZJetsToNuNu\_50\_HT\_100\_TuneZ2Star\_8TeV\_madgraph/Summer12\_DR53X-PU\_S10\_START53\_V7A-v1/AODSIM} \\
\texttt{/ZJetsToNuNu\_100\_HT\_200\_TuneZ2Star\_8TeV\_madgraph/Summer12\_DR53X-PU\_S10\_START53\_V7A-v1/AODSIM} \\
\texttt{/ZJetsToNuNu\_200\_HT\_400\_TuneZ2Star\_8TeV\_madgraph/Summer12\_DR53X-PU\_S10\_START53\_V7A-v1/AODSIM} \\
\texttt{/ZJetsToNuNu\_400\_HT\_inf\_TuneZ2Star\_8TeV\_madgraph/Summer12\_DR53X-PU\_S10\_START53\_V7A-v1/AODSIM} \\
\texttt{/TT\_CT10\_TuneZ2star\_8TeV-powheg-tauola/Summer12\_DR53X-PU\_S10\_START53\_V7A-v1/AODSIM} \\
\texttt{/TT\_CT10\_TuneZ2star\_8TeV-powheg-tauola/Summer12\_DR53X-PU\_S10\_START53\_V7A-v2/AODSIM} \\
\texttt{/TTZJets\_8TeV-madgraph\_v2/Summer12\_DR53X-PU\_S10\_START53\_V7A-v1/AODSIM} \\
\texttt{/T\_s-channel\_TuneZ2star\_8TeV-powheg-tauola/Summer12\_DR53X-PU\_S10\_START53\_V7A-v1/AODSIM} \\
\texttt{/T\_tW-channel-DR\_TuneZ2star\_8TeV-powheg-tauola/Summer12\_DR53X-PU\_S10\_START53\_V7A-v1/AODSIM} \\
\texttt{/T\_t-channel\_TuneZ2star\_8TeV-powheg-tauola/Summer12\_DR53X-PU\_S10\_START53\_V7A-v1/AODSIM} \\
\texttt{/Tbar\_t-channel\_TuneZ2star\_8TeV-powheg-tauola/Summer12\_DR53X-PU\_S10\_START53\_V7A-v1/AODSIM} \\
\texttt{/Tbar\_s-channel\_TuneZ2star\_8TeV-powheg-tauola/Summer12\_DR53X-PU\_S10\_START53\_V7A-v1/AODSIM} \\
\texttt{/Tbar\_tW-channel-DR\_TuneZ2star\_8TeV-powheg-tauola/Summer12\_DR53X-PU\_S10\_START53\_V7A-v1/AODSIM} \\
\texttt{/DYJetsToLL\_M-50\_TuneZ2Star\_8TeV-madgraph-tarball/Summer12\_DR53X-PU\_S10\_START53\_V7A-v1/AODSIM} \\
\texttt{/DYJetsToLL\_HT-200To400\_TuneZ2Star\_8TeV-madgraph/Summer12\_DR53X-PU\_S10\_START53\_V7A-v1/AODSIM} \\
\texttt{/DYJetsToLL\_HT-400ToInf\_TuneZ2Star\_8TeV-madgraph/Summer12\_DR53X-PU\_S10\_START53\_V7A-v1/AODSIM} \\
\texttt{/DYJetsToLL\_M-10To50filter\_8TeV-madgraph/Summer12\_DR53X-PU\_S10\_START53\_V7A-v1/AODSIM} \\
\texttt{/GJets\_HT-200To400\_8TeV-madgraph/Summer12\_DR53X-PU\_S10\_START53\_V7A-v1/AODSIM} \\
\texttt{/GJets\_HT-400ToInf\_8TeV-madgraph/Summer12\_DR53X-PU\_S10\_START53\_V7A-v1/AODSIM} \\
\texttt{/WZ\_TuneZ2star\_8TeV\_pythia6\_tauola/Summer12\_DR53X-PU\_S10\_START53\_V7A-v1/AODSIM} \\
\texttt{/WW\_TuneZ2star\_8TeV\_pythia6\_tauola/Summer12\_DR53X-PU\_S10\_START53\_V7A-v1/AODSIM} \\
\texttt{/ZZ\_TuneZ2star\_8TeV\_pythia6\_tauola/Summer12\_DR53X-PU\_S10\_START53\_V7A-v1/AODSIM} \\
\normalsize


The signal Monte Carlo samples for physics at \com = 8 \TeV are taken from a \texttt{FastSim} simulation production run with \texttt{CMSSW\_5\_2\_6}. \\
\footnotesize
\texttt{/SMS-T1\_Mgluino-100to2000\_mLSP-0to2000\_8TeV-Pythia6Z/Summer12-START52\_V9\_FSIM-v1/AODSIM} \\
\texttt{/SMS-T1tttt\_Mgluino-350to2000\_mLSP-0to1650\_8TeV-Pythia6Z/Summer12-START52\_V9\_FSIM-v3/AODSIM} \\
\texttt{/SMS-T1bbbb\_Mgluino-100to2000\_mLSP-0to2000\_8TeV-Pythia6Z/Summer12-START52\_V9\_FSIM-v1/AODSIM} \\
\texttt{/SMS-T2\_Msquark-225to1200\_mLSP-0to1200\_8TeV-Pythia6Z/Summer12-START52\_V9\_FSIM-v1/AODSIM} \\
\texttt{/SMS-T2bb\_Msbottom-225to1200\_mLSP-0to1175\_8TeV-Pythia6Z/Summer12-START52\_V9\_FSIM-v2/AODSIM} \\
\texttt{/SMS-T2bw\_FineBin\_Mstop-100to600\_mLSP-0to500\_8TeV-Pythia6Z/Summer12-START52\_V9\_FSIM-v2/AODSIM} \\
\texttt{/SMS-T2tt\_FineBin\_Mstop-225to1200\_mLSP-0to1000\_8TeV-Pythia6Z/Summer12-START52\_V9\_FSIM-v1/AODSIM} \\
\normalsize

\chapter{Additional Material on Background Estimation Methods}
\label{app:backgroundestimation}
\section{Determination of $k_{QCD}$}
\label{app:kqcd}


\begin{minipage}{\linewidth}
\centering
\includegraphics[width = 4.9in]{plots/qcd_sideband_fits.pdf}
\captionof{figure}[$R_{\alphat}$(\theht) and exponential fits for each of the data sideband regions. Fit is conducted between the \theht region 275 $<$ \theht $<$ 575.]{$R_{\alphat}$(\theht) and exponential fits for each of the data sideband regions. Fit is conducted between the \theht region 275 $<$ \theht $<$ 575.}
\label{fig:qcd_sideband_fits}
\end{minipage}

\section{Effect of Varying Background Cross-sections on Closure Tests }
\label{app:xsecvariation}

Closure tests with cross section variations of +20\% and -20\% applied to W + jets and \ttbar processes respectively.

\begin{figure}[ht]
\centering
\begin{minipage}[b]{0.48 \linewidth}
\includegraphics[width = 1.0\linewidth]{plots/syst-le3j_nominal.pdf}
\centering
(a)  
\end{minipage}
\quad
\begin{minipage}[b]{0.48\linewidth}
\includegraphics[width = 1.0\linewidth]{plots/syst-le3j_varied.pdf}
\centering
(b) 
\end{minipage}
\caption[Sets of closure tests overlaid on top of the systematic uncertainty used for each of the five \theht regions in the $2 \leq n_{jet} \leq 3$ jet multiplicity category for nominal and varied cross-sections.]{Sets of closure tests (open symbols) overlaid on top of the systematic uncertainty used for each of the five \theht regions (shaded bands) in the $2 \leq n_{jet} \leq 3$  jet multiplicity category for nominal and varied cross-sections; (a) Nominal and (b) Varied $\pm$20\%.}
\label{fig:xsecvariedle3j}
\end{figure}


\begin{figure}[ht]
\centering
\begin{minipage}[b]{0.48 \linewidth}
\includegraphics[width = 1.0\linewidth]{plots/syst-ge4j_nominal.pdf}
\centering
(a)  
\end{minipage}
\quad
\begin{minipage}[b]{0.48\linewidth}
\includegraphics[width = 1.0\linewidth]{plots/syst-ge4j_varied.pdf}
\centering
(b) 
\end{minipage}
\caption[Sets of closure tests overlaid on top of the systematic uncertainty used for each of the five \theht regions in the $n_{jet} \geq 4$ jet multiplicity category for nominal and varied cross-sections.]{Sets of closure tests (open symbols) overlaid on top of the systematic uncertainty used for each of the five \theht regions (shaded bands) in the $n_{jet} \geq 4$ jet multiplicity category for nominal and varied cross-sections; (a) Nominal (b) Varied $\pm$20\%.}
\label{fig:xsecvariedge4j}
\end{figure}

\def\arraystretch{1.3}
\begin{table}[ht!]
\begin{center}
\footnotesize
\begin{tabular}{ cccccc}
\hline\hline
\multicolumn{2}{c}{}               & \multicolumn{4}{c}{$H_{T}$ (GeV)} \\
$n_{b}^{reco}$ & Cross Section     & 275--325                  & 325--375                  & 375--475                  & 475--575                 \\ 
\hline
 0 & Nominal                       & 0.303  $\pm$  0.010       & 0.258  $\pm$  0.007       & 0.192  $\pm$  0.003       & 0.148  $\pm$  0.004      \\
 0 & Varied                        & 0.300  $\pm$  0.010       & 0.256  $\pm$  0.007       & 0.191  $\pm$  0.003       & 0.147  $\pm$  0.004      \\ 
 1 & Nominal                       & 0.294  $\pm$  0.005       & 0.246  $\pm$  0.004       & 0.189  $\pm$  0.003       & 0.139  $\pm$  0.003      \\
 1 & Varied                        & 0.295  $\pm$  0.006       & 0.248  $\pm$  0.004       & 0.191  $\pm$  0.003       & 0.140  $\pm$  0.003      \\ 
 2 & Nominal                       & 0.208  $\pm$  0.003       & 0.183  $\pm$  0.004       & 0.145  $\pm$  0.003       & 0.123  $\pm$  0.004      \\
 2 & Varied                        & 0.211  $\pm$  0.004       & 0.185  $\pm$  0.004       & 0.147  $\pm$  0.003       & 0.124  $\pm$  0.004      \\ 
 3 & Nominal                       & 0.214  $\pm$  0.005       & 0.202  $\pm$  0.007       & 0.159  $\pm$  0.006       & 0.140  $\pm$  0.007      \\
 3 & Varied                        & 0.215  $\pm$  0.005       & 0.203  $\pm$  0.007       & 0.159  $\pm$  0.006       & 0.140  $\pm$  0.007      \\ 
 $\geq$4 & Nominal                 & 0.220  $\pm$  0.015       & 0.245  $\pm$  0.035       & 0.119  $\pm$  0.009       & -                         \\
 $\geq$4 & Varied                  & 0.220  $\pm$  0.015       & 0.245  $\pm$  0.035       & 0.119  $\pm$  0.009       & -                          \\  
\hline\hline
$n_{b}^{reco}$ & Cross Section   & 575--675                  & 675--775                  & 775--875                  & 875--$\infty$            \\ 
\hline
0 & Nominal                        & 0.119  $\pm$  0.004       & 0.098  $\pm$  0.005       & 0.077  $\pm$  0.006       & 0.049  $\pm$  0.005      \\
0 & Varied                         & 0.120  $\pm$  0.005       & 0.098  $\pm$  0.006       & 0.077  $\pm$  0.007       & 0.049  $\pm$  0.005      \\ 
1 & Nominal                        & 0.115  $\pm$  0.004       & 0.093  $\pm$  0.005       & 0.075  $\pm$  0.007       & 0.063  $\pm$  0.006      \\
1 & Varied                         & 0.116  $\pm$  0.004       & 0.098  $\pm$  0.005       & 0.081  $\pm$  0.007       & 0.065  $\pm$  0.006      \\ 
2 & Nominal                        & 0.096  $\pm$  0.005       & 0.070  $\pm$  0.006       & 0.051  $\pm$  0.007       & 0.063  $\pm$  0.008      \\
2 & Varied                         & 0.098  $\pm$  0.005       & 0.073  $\pm$  0.006       & 0.053  $\pm$  0.007       & 0.064  $\pm$  0.008      \\ 
3 & Nominal                        & 0.114  $\pm$  0.009       & 0.065  $\pm$  0.007       & 0.070  $\pm$  0.017       & 0.092  $\pm$  0.020      \\
3 & Varied                         & 0.114  $\pm$  0.009       & 0.066  $\pm$  0.007       & 0.070  $\pm$  0.016       & 0.093  $\pm$  0.020      \\ 
\end{tabular}
\end{center}
\caption[Translation factors constructed from the \mupjets control sample and signal selection MC, to predict yields for the W + jets and \ttbar backgrounds in the signal region.]{Translation factors constructed from the \mupjets control sample and signal selection MC, to predict yields for the W + jets and \ttbar backgrounds in the signal region with (a) NNLO cross sections corrected by k-factors determined from a data sideband see Section (\ref{subsec:mckfactors}), marked as �Nominal�, and (b) the same cross sections but with those for W + jets and \ttbar varied up and down by 20\%, respectively, marked as �Varied�. No requirement is placed on the jet multiplicity of events within this table.}\label{tab:xsecvaried}
\end{table}
\def\arraystretch{1.0}


\chapter{Additional Material  for B-tag Template Method}
\label{app:templatematerial}
\section{Templates Fits in Simulation}
\label{app:templatemc}

The result of template fits for the three \ac{CSV} working points in the $n_{jet} = 3$, \theht $>$ 375 category:

\begin{figure}[ht]
\centering
\begin{minipage}[b]{0.51 \linewidth}
\vspace{30mm}
\includegraphics[width = 1.0\linewidth]{plots/ThesisPlots/Final_Fit_To_MC_Normal_Loose_HTBin_OneMuon_Template_375_jet_mult_3.pdf}
\centering (a) Loose working point $n_{jet}$ = 3 
\end{minipage}
\begin{minipage}[b]{0.51\linewidth}
\includegraphics[width = 1.0\linewidth]{plots/ThesisPlots/Final_Fit_To_MC_Normal_Medium_HTBin_OneMuon_Template_375_jet_mult_3.pdf}
\centering (b) Medium working point $n_{jet}$ = 3 
\end{minipage}
\quad
\begin{minipage}[b]{0.51\linewidth}
\centering
\includegraphics[width = 1.0\linewidth]{plots/ThesisPlots/Final_Fit_To_MC_Normal_Tight_HTBin_OneMuon_Template_375_jet_mult_3.pdf}
\centering (c) Tight working point $n_{jet} =$ 3 
\end{minipage}
\caption[Results of fitting the Z = 0 and Z = 2 templates in the $n_{b}^{reco}$ = 0-2 control region to yields from simulation in the \mupjets control sample for the \theht $>$ 375 \GeV, $n_{jet} = 3$ category.]{Results of fitting the Z = 0 and Z = 2 templates in the $n_{b}^{reco}$ = 0-2 control region to yields from simulation in the \mupjets control sample for the \theht $>$ 375 \GeV, $n_{jet} = 3$ category. Data is represented by the black circles with the blue, red and black lines representing the Z=0, Z=2 and combination of both templates respectively. Grey bands represent the uncertainty of the fit. The $\chi^{2}$ parameter represent the goodness of fit to the control and signal region.}
\label{app:template_closure_njet3}
\end{figure}

\FloatBarrier

Template fits for the three \ac{CSV} working points in the $n_{jet} = 4$, \theht $>$ 375 category:

\begin{figure}[ht]
\centering
\begin{minipage}[b]{0.51\linewidth}
\includegraphics[width = 1.0\linewidth]{plots/ThesisPlots/Final_Fit_To_MC_Normal_Loose_HTBin_OneMuon_Template_375_jet_mult_4.pdf}
\centering (a) Loose working point $n_{jet}$ = 4 
\end{minipage}
\quad
\begin{minipage}[b]{0.51\linewidth}
\includegraphics[width = 1.0\linewidth]{plots/ThesisPlots/Final_Fit_To_MC_Normal_Medium_HTBin_OneMuon_Template_375_jet_mult_4.pdf}
\centering (b) Medium working point $n_{jet}$ = 4 
\end{minipage}
\quad
\begin{minipage}[b]{0.51\linewidth}
\centering
\includegraphics[width = 1.0\linewidth]{plots/ThesisPlots/Final_Fit_To_MC_Normal_Tight_HTBin_OneMuon_Template_375_jet_mult_4.pdf}
\centering (c) Tight working point $n_{jet} =$ 4 
\end{minipage}
\caption[Results of fitting the Z = 0 and Z = 2 templates in the $n_{b}^{reco}$ = 0-2 control region to yields from simulation in the \mupjets control sample for the \theht $>$ 375 \GeV, $n_{jet} = 4$ category.]{Results of fitting the Z = 0 and Z = 2 templates in the $n_{b}^{reco}$ = 0-2 control region to yields from simulation in the \mupjets control sample for the \theht $>$ 375 \GeV, $n_{jet} = 4$ category. Data is represented by the black circles with the blue, red and black lines representing the Z=0, Z=2 and combination of both templates respectively. Grey bands represent the statistical uncertainty of the fit. The $\chi^{2}$ parameters represent the goodness of fit to the control and signal region.}
\label{fig:template_closure_high}
\end{figure}

\FloatBarrier

\section{Pull Distributions for Template Fits}
\label{app:templatepulldistributions}

\begin{figure}[ht]
\centering
\begin{minipage}[b]{0.48 \linewidth}
\includegraphics[width = 1.0\linewidth]{plots/ThesisPlots/Pull_Plot_Z0_HTbin_Template_375_jet_mult_3_num_param_3.pdf}
\centering (a) Z0 Template, \theht $> 375$ ,  $n_{jet} = 3$
\end{minipage}
\quad
\begin{minipage}[b]{0.48\linewidth}
\includegraphics[width = 1.0\linewidth]{plots/ThesisPlots/Pull_Plot_Z2_HTbin_Template_375_jet_mult_3_num_param_3.pdf}
\centering (b) Z2 Template, \theht $> 375$ ,  $n_{jet} = 3$
\end{minipage}
\quad
\begin{minipage}[b]{0.48 \linewidth}
\includegraphics[width = 1.0\linewidth]{plots/ThesisPlots/Pull_Plot_Z0_HTbin_Template_375_jet_mult_4_num_param_3.pdf}
\centering (a) Z0 Template, \theht $> 375$ ,  $n_{jet} = 4$
\end{minipage}
\quad
\begin{minipage}[b]{0.48\linewidth}
\includegraphics[width = 1.0\linewidth]{plots/ThesisPlots/Pull_Plot_Z2_HTbin_Template_375_jet_mult_4_num_param_3.pdf}
\centering (b) Z2 Template, \theht $> 375$ ,  $n_{jet} = 4$
\end{minipage}
\quad
\begin{minipage}[b]{0.48 \linewidth}
\includegraphics[width = 1.0\linewidth]{plots/ThesisPlots/Pull_Plot_Z0_HTbin_Template_375_jet_mult_5_num_param_3.pdf}
\centering (a) Z0 Template, \theht $> 375$ ,  $n_{jet} \geq 5$
\end{minipage}
\quad
\begin{minipage}[b]{0.48\linewidth}
\includegraphics[width = 1.0\linewidth]{plots/ThesisPlots/Pull_Plot_Z2_HTbin_Template_375_jet_mult_5_num_param_3.pdf}
\centering (b) Z2 Template, \theht $> 375$ ,  $n_{jet} \geq 5$
\end{minipage}
\caption[Pull distributions of the normalisation parameter of each template, $\frac{(\theta - \hat{\theta})}{\sigma}$. Distributions are constructed from $10^{4}$ pseudo-experiments generated by a gaussian distribution with width $\sigma$, centred on the nominal template value of each point within the low \nbreco control region.]{Pull distributions of the normalisation parameter of each template, $\frac{(\theta - \hat{\theta})}{\sigma}$. Distributions are constructed from $10^{4}$ pseudo-experiments generated by a gaussian distribution with width $\sigma$, centred on the nominal template value of each point within the low \nbreco control region. Distributions are shown for both Z0 and Z2 templates for the medium \ac{CSV} working point.}
\label{app:templatepulldistributionsmedium}
\end{figure}



\section{Templates Fits in Data Control Sample}
\label{app:templatedata}

Template fits for the three \theht bins, in the $n_{jet} = 3$,  medium \ac{CSV} working point:

\begin{figure}[ht]
\centering
\begin{minipage}[b]{0.50 \linewidth}
\includegraphics[width = 1.0\linewidth]{plots/ThesisPlots/Final_Fit_To_Data_Normal_Medium_HTBin_OneMuon_275_325_jet_mult_3.pdf}
\centering (a) $n_{jet} =$  3 , 275 $<$ \theht $<$ 325
\end{minipage}
\quad
\begin{minipage}[b]{0.50\linewidth}
\includegraphics[width = 1.0\linewidth]{plots/ThesisPlots/Final_Fit_To_Data_Normal_Medium_HTBin_OneMuon_325_375_jet_mult_3.pdf}
\centering (b) $n_{jet} =$  3 , 325 $<$ \theht $<$ 375 
\end{minipage}
\quad
\begin{minipage}[b]{0.50\linewidth}
\centering
\includegraphics[width = 1.0\linewidth]{plots/ThesisPlots/Final_Fit_To_Data_Normal_Medium_HTBin_OneMuon_Template_375_jet_mult_3.pdf}
\centering (c) $n_{jet} =$ 3 , \theht $>$ 375 
\end{minipage}
\caption[Results of fitting the Z = 0 and Z = 2 templates in the $n_{b}^{reco}$ = 0-2 control region to data from the \mupjets control sample, for the \ac{CSVM} working point, with $n_{jet} = 3$ in each \theht category.]{Results of fitting the Z = 0 and Z = 2 templates in the $n_{b}^{reco}$ = 0-2 control region to data from the \mupjets control sample, for the \ac{CSVM} working point, with $n_{jet} = 3$ in each \theht category. Data is represented by the black circles with the blue, red and black lines representing the Z=0, Z=2 and combination of both templates respectively. Grey bands represent the uncertainty of the fit. The $\chi^{2}$ parameters represent the goodness of fit to the control and signal region.}
\label{app:template_data_loose_njet5}
\end{figure}

Template fits for the three \theht bins, in the $n_{jet} = 4$,  medium \ac{CSV} working point:

\begin{figure}[ht]
\centering
\begin{minipage}[b]{0.50 \linewidth}
\includegraphics[width = 1.0\linewidth]{plots/ThesisPlots/Final_Fit_To_Data_Normal_Medium_HTBin_OneMuon_275_325_jet_mult_4.pdf}
\centering (a) $n_{jet} =$  4 , 275 $<$ \theht $<$ 325
\end{minipage}
\quad
\begin{minipage}[b]{0.50\linewidth}
\includegraphics[width = 1.0\linewidth]{plots/ThesisPlots/Final_Fit_To_Data_Normal_Medium_HTBin_OneMuon_325_375_jet_mult_4.pdf}
\centering (b) $n_{jet} =$ 4 , 325 $<$ \theht $<$ 375 
\end{minipage}
\quad
\begin{minipage}[b]{0.50\linewidth}
\centering
\includegraphics[width = 1.0\linewidth]{plots/ThesisPlots/Final_Fit_To_Data_Normal_Medium_HTBin_OneMuon_Template_375_jet_mult_4.pdf}
\centering (c) $n_{jet} =$ 4 , \theht $>$ 375 
\end{minipage}
\caption[Results of fitting the Z = 0 and Z = 2 templates in the $n_{b}^{reco}$ = 0-2 control region to data from the \mupjets control sample, for the \ac{CSV} medium working point, with $n_{jet} = 4$ in each \theht category.]{Results of fitting the Z = 0 and Z = 2 templates in the $n_{b}^{reco}$ = 0-2 control region to data from the \mupjets control sample, for the \ac{CSVM} working point, with $n_{jet} = 4$ in each \theht category. Data is represented by the black circles with the blue, red and black lines representing the Z=0, Z=2 and combination of both templates respectively. Grey bands represent the uncertainty of the fit. The $\chi^{2}$ parameters represents the goodness of fit to the control and signal region.}
\label{app:template_data_tight_njet5}
\end{figure}

\section{Templates Fits in Data Signal Region}
\label{app:templatedata_signal}

Template fits for the three \ac{CSV} working points, in the $n_{jet}$ = 3, \theht $>$ 375 category :

\begin{figure}[ht]
\footnotesize
\centering
\begin{minipage}[b]{0.50 \linewidth}
\includegraphics[width = 1.0\linewidth]{plots/TemplatesSignal/Final_Fit_To_Data_Normal_Loose_HTBin_Template_375_jet_mult_3.pdf}
\centering (a) Loose working point : $n_{jet}=$  3 , \theht $>$ 375
\end{minipage}
\quad
\begin{minipage}[b]{0.50\linewidth}
\includegraphics[width = 1.0\linewidth]{plots/TemplatesSignal/Final_Fit_To_Data_Normal_Medium_HTBin_Template_375_jet_mult_3.pdf}
\centering (b) Medium working point : $n_{jet}$ = 3 , \theht $>$ 375 
\end{minipage}
\quad
\begin{minipage}[b]{0.50\linewidth}
\centering
\includegraphics[width = 1.0\linewidth]{plots/TemplatesSignal/Final_Fit_To_Data_Normal_Tight_HTBin_Template_375_jet_mult_3.pdf}
\centering (c) Tight working point :  $n_{jet} =$ 3 , \theht $>$ 375 
\end{minipage}
\caption[Results of fitting the Z = 0 and Z = 2 templates in the $n_{b}^{reco}$ = 0-2 control region to data from the hadronic signal selection, in the $n_{jet} = 3$ and \theht $>$ 375 category for all \ac{CSV} working points.]{Results of fitting the Z = 0 and Z = 2 templates in the $n_{b}^{reco}$ = 0-2 control region to data from the hadronic signal selection, in the $n_{jet} = 3$ and \theht $>$ 375 category for all \ac{CSV} working points. Data is represented by the black circles with the blue, red and black lines representing the Z=0, Z=2 and combination of both templates respectively. Grey bands represent the uncertainty of the fit. The $\chi^{2}$ parameters represent the goodness of fit to the control and signal region.}
\label{app:template_data_signal_njet3}
\end{figure}

Template fits for the three \ac{CSV} working points, in the $n_{jet}$ = 4, \theht $>$ 375 category :

\begin{figure}[ht]
\footnotesize
\centering
\begin{minipage}[b]{0.50 \linewidth}
\includegraphics[width = 1.0\linewidth]{plots/TemplatesSignal/Final_Fit_To_Data_Normal_Loose_HTBin_Template_375_jet_mult_4.pdf}
\centering (a) Loose working point : $n_{jet}=$  4 , \theht $>$ 375
\end{minipage}
\quad
\begin{minipage}[b]{0.50\linewidth}
\includegraphics[width = 1.0\linewidth]{plots/TemplatesSignal/Final_Fit_To_Data_Normal_Medium_HTBin_Template_375_jet_mult_4.pdf}
\centering (b) Medium working point : $n_{jet}$ = 4 , \theht $>$ 375 
\end{minipage}
\quad
\begin{minipage}[b]{0.50\linewidth}
\centering
\includegraphics[width = 1.0\linewidth]{plots/TemplatesSignal/Final_Fit_To_Data_Normal_Tight_HTBin_Template_375_jet_mult_4.pdf}
\centering (c) Tight working point :  $n_{jet} =$ 4 , \theht $>$ 375 
\end{minipage}
\caption[Results of fitting the Z = 0 and Z = 2 templates in the $n_{b}^{reco}$ = 0-2 control region to data from the hadronic signal selection, in the $n_{jet} = 4$ and \theht $>$ 375 category for all \ac{CSV} working points.]{Results of fitting the Z = 0 and Z = 2 templates in the $n_{b}^{reco}$ = 0-2 control region to data from the hadronic signal selection, in the $n_{jet} = 4$ and \theht $>$ 375 category for all \ac{CSV} working points. Data is represented by the black circles with the blue, red and black lines representing the Z=0, Z=2 and combination of both templates respectively. Grey bands represent the uncertainty of the fit. The $\chi^{2}$ parameters represent the goodness of fit to the control and signal region.}
\label{app:template_data_signal_njet3}
\end{figure}

\end{appendices}

%% Produce the un-numbered back matter (e.g. colophon,
%% bibliography, tables of figures etc., index...)
\begin{backmatter}
  %\begin{colophon}
%  This thesis was made in \LaTeXe{} using the ``hepthesis'' class~\cite{hepthesis}.
%\end{colophon}

%% You're recommended to use the eprint-aware biblio styles which
%% can be obtained from e.g. www.arxiv.org. The file mythesis.bib
%% is derived from the source using the SPIRES Bibtex service.
%\bibliographystyle{h-physrev}



\bibliographystyle{ieeetr}
\bibliography{mythesis}


\section*{Acronyms}

\begin{acronym}[AAAAAAA]
\acro{ALICE}[ALICE]{A Large Ion Collider Experiment}
\acro {ATLAS} [ATLAS] {A Toroidal LHC ApparatuS}
\acro{APD}[APD]{Avalanche Photo-Diodes}
\acro{BSM}[BSM]{Beyond Standard Model}
\acro {CERN} [CERN] {European Organization for Nuclear Research}
\acro {CMS} [CMS] {Compact Muon Solenoid}
\acro {CMSSM} [CMSSM] {Compressed Minimal SuperSymmetric Model}
\acro {CSC} [CSC] {Cathode Stripe Chamber}
\acro {CSV} [CSV] {Combined Secondary Vertex}
\acro {CSVM} [CSVM] {Combined Secondary Vertex Medium Working Point}
\acro {DT} [DT] {Drift Tube}
\acro {ECAL} [ECAL] {Electromagnetic CALorimeter}
\acro {EB} [EB] {Electromagnetic CALorimeter Barrel}
\acro {EE} [EE] {Electromagnetic CALorimeter Endcap}
\acro {ES} [ES] {Electromagnetic CALorimeter pre-Shower}
\acro {EMG} [EMG] {Exponentially Modified Gaussian}
\acro {EPJC} [EPJC] {European Physical Journal C}
\acro {EWK} [EWK] {Electroweak Sector}
\acro {GCT} [GCT] {Global Calorimeter Trigger}
\acro {GMT} [GMT] {Global MuonTrigger}
\acro {GT} [GT] {Global Trigger}
\acro {HB} [HB] {Hadron Barrel}
\acro {HCAL} [HCAL] {Hadronic CALorimeter}
\acro {HE} [HE] {Hadron Endcaps}
\acro {HF} [HF] {Hadron Forward}
\acro {HLT}[HLT] {Higher Level Trigger}
\acro {HO} [HO] {Hadron Outer}
\acro {HPD} [HPD] {Hybrid Photo Diode}
\acro {ISR} [ISR] {Initial State Radiation}
\acro {LUT} [LUT] {Look Up Table}
\acro {L1} [L1] {Level 1 Trigger}
\acro{lhc}[LHC]{Large Hadron Collider}
\acro{LHCb}[LHCb]{Large Hadron Collider Beauty}
\acro{LSP}[LSP]{Lightest Supersymmetric Partner}
\acro{NLL}[NLL]{Next to Leading Logorithmic Order}
\acro{NLO}[NLO]{Next to Leading Order}
\acro{NNLO}[NNLO]{Next to Next Leading Order}
\acro{POGs}[POGs]{Physics Object Groups}
\acro{PS}[PS]{Proton Synchrotron}
\acro{QED}[QED]{Quantum Electro-Dynamics}
\acro{QCD}[QCD]{Quantum Chromo-Dynamics}
\acro{QFT}[QFT]{Quantum Field Theory}
\acro {RBXs} [RBXs] {Readout Boxes}
\acro {RPC} [RPC] {Resistive Plate Chamber}
\acro {RCT} [RCT] {Regional Calorimeter Trigger}
\acro {RMT} [RMT] {Regional Muon Trigger}
\acro{SUSY}[SUSY]{SUperSYmmetry}
\acro{SM}[SM]{Standard Model}
\acro{SMS}[SMS]{Simplified Model Spectra}
\acro{SPS}[SPS]{Super Proton Synchrotron}
\acro{TF}[TF]{Transfer Factor}
\acro{TP}[TP]{Trigger Primative}
\acro{VEV}[VEV]{Vacuum Expectation Value}
\acro{VPT}[VPT]{Vacuum Photo-Triodes}
\acro{WIMP}[WIMP]{Weakly Interacting Massive Particle}

\end{acronym}


%% I prefer to put these tables here rather than making the
%% front matter seemingly interminable. No-one cares, anyway!


%%\listoffigures
%%\listoftables
%% If you have time and interest to generate a (decent) index,
%% then you've clearly spent more time on the write-up than the 
%% research ;-)
%\printindex

\end{backmatter}

%% Close
\end{document}
