\chapter{A Theoretical Overview}
\label{chap:theorysection}

Within this chapter, a brief introduction and background to the \ac{SM} is given. Its success as a rigorously tested and widely accepted theory is discussed as well as the deficiencies with this theory that hint there this theory is not a complete description of our universe. The motivations for new physics at the \TeV scale and in particular Supersymmetric theories are outlined within Section \ref{sec:susytheory}, with the chapter concluding with how an experimental signature of such theories can be produced and observed at the \ac{lhc}, Section \ref{sec:susysearches}.

\section{The Standard Model}

\label{sec:thesm}

The \ac{SM} is the name given to the quantum field theory which describes the interactions and properties of all the known elementary particles .

\section{Motivation for Beyond the Standard Model Physics}

\label{sec:bsmmotivation}

Dark Matter etc

\section{Supersymmmetry}

\label{sec:susytheory}

What is this theory that doesn't exist all about?

\subsection{R-Parity}

\label{subsec:rparity}

R-Parity stuff here innit.

\subsection{ Supersymmetry Breaking}

\label{subsec:susybreaking}

Why are supersymmetric particles not the same mass.


\section{Searching for SUSY at the LHC}
\label{sec:susysearches}

\subsection{Simplified Models}

\label{subsec:sms}

With such a variety of different way for a \ac{SUSY} signal to manifest itself, it is necessary to be able to interpret experimental reach through the masses of gluinos and squarks which can excluded by experimental searches rather than on a model specific basis. This is accomplished through \acf{SMS} models, which are \ac{SUSY} decays which contain only one process. For example the production of a pair of gluinos which then decay via \ac{SM} processes, to a set decay topology with a 100$\%$ branching ratio, shown in �.

Searching and interpreting \ac{SUSY} searches in this way�.

The convention for the naming of these \ac{SMS} models is via the prefix...
