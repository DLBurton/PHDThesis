\chapter{A Theoretical Overview}
\label{chap:theorysection}

Within this chapter, a brief introduction and background to the \ac{SM} is given. Its success as a rigorously tested and widely accepted theory is discussed as well as the deficiencies with this theory that hint there this theory is not a complete description of our universe. The motivations for new physics at the \TeV scale and in particular Supersymmetric theories are outlined within Section \ref{sec:susytheory}, with the chapter concluding with how an experimental signature of such theories can be produced and observed at the \ac{lhc}, Section \ref{sec:susysearches}.

\section{The Standard Model}

\label{sec:thesm}

The \ac{SM} is the name given to the relativistic  \acf{QFT}, where particles are represented as excitations of fields, which describes the interactions and properties of all the known elementary particles \cite{PhysRevLett.19.1264}\cite{Glashow:1961tr}\cite{Salam:1968rm}\cite{Hooft1971167}. It is a renormalisable field theory which contains three symmetries: $SU(3)$ for colour charge, SU(2) for weak isospin and U(1) relating to weak hyper charge, which require its Lagrangian \lsm to be invariant under local gauge transformation. 

Within the \ac{SM} theory, matter is composed of spin \half fermions, which interact with each other via the exchange of spin-1 gauge bosons. A summary of the known fundamental fermions and bosons is given in Table \ref{tab:sm_particles}.

\begin{table}[h!]
\begin{center}
\begin{tabular*}{0.75\textwidth}{@{\extracolsep{\fill}}|c|c|c|c|c|}
\cline{1-5}
Particle                   & Symbol      & Spin & Charge         & Mass (\GeV) \\ \cline {1-5}
\multicolumn{5}{|c|}{\textbf{First Generation Fermions}}  					 \\ \cline{1-5}
Electron Neutrino &$\nu_{e}$   & \half & 0                     &   $< 2.2 \times 10^{-6}$ \\ \cline{1-5}
Electron                  & e                 & \half & -1                    &   $0.51 \times 10^{-3}$ \\ \cline{1-5}
Up Quark                & u                 & \half & $\frac{2}{3}$ &   $2.3 ^{+0.7}_{-0.5} \times 10^{-3}$ \\ \cline{1-5}
Down Quark          & d                 & \half & $-\frac{1}{3}$&   $4.8 ^{+0.7}_{-0.3}\times 10^{-3}$ \\ \cline{1-5}
\multicolumn{5}{|c|}{\textbf{Second Generation Fermions}}  					 \\ \cline{1-5}
Muon Neutrino       &$\nu_{\mu}$& \half & 0                     &   -                           \\ \cline{1-5}
Muon                       & $\mu$       & \half & -1                    &   $1.05 \times 10^{-3}$ \\ \cline{1-5}
Charm Quark        & c                 & \half & $\frac{2}{3}$ &   $1.275 \pm 0.025$ \\ \cline{1-5}
Strange Quark      & s                 & \half & $-\frac{1}{3}$&   $95 \pm 5 \times 10^{-3}$ \\ \cline{1-5}
\multicolumn{5}{|c|}{\textbf{Third Generation Fermions}}  					 \\ \cline{1-5}
Tau Neutrino         &$\nu_{\tau}$& \half & 0                     &   -                           \\ \cline{1-5}
Tau                          & $\tau$       & \half & -1                    &   1.77		 	\\ \cline{1-5}
Top Quark              & t                 & \half & $\frac{2}{3}$ &   $173.5  \pm 0.8$ \\ \cline{1-5}
Bottom Quark        & b                 & \half & $-\frac{1}{3}$&   $4.65  \pm 0.03$ \\ \cline{1-5}
\multicolumn{5}{|c|}{\textbf{Gauge Bosons}}  					 \\ \cline{1-5}
Photon                    &$\gamma$&1   & 0                         &   0                           \\ \cline{1-5}
W Boson                 & $\Wboson$&1& $\pm$1              &   80.385 $\pm 0.015$ \\ \cline{1-5}
Z Boson                 & $\Zboson$& 1 & 0		          &   $91.187  \pm 0.002$ \\ \cline{1-5}
Gluons                   & g                 & 1 & 0			 &   0 \\ \cline{1-5}
Higgs Boson         & H                & 0 & 0			 &   125.3 $\pm 0.5$ \cite{Chatrchyan:2012ufa} \\ \cline{1-5}
\end{tabular*}
\end{center}
\caption[The fundamental particles of the \ac{SM}, with spin, charge and mass displayed.]{The fundamental particles of the \ac{SM}, with spin, charge and mass displayed. Latest mass measurements taken from \cite{pdg2012}. }
\label{tab:sm_particles}

\end{table}

Fermions are separated into quarks and leptons of which only quarks interact with the strong nuclear force. Quarks unlike leptons are not seen as free particles in nature, but rather exist only within baryons, composed of three quarks with an overall integer charge, and quark-anti-quark pairs called mesons. Both leptons and quarks are grouped into three generations which have the same properties, but with ascending mass in each subsequent generation. 

The gauge bosons mediate the interactions between fermions. The field theories of \acf{QED} and \acf{QCD}, yield massless mediator bosons, the photon and eight coloured gluons which are consequences of the gauge invariance of those theories.  

The unification of the electromagnetic and weak-nuclear forces into the current Electroweak theory yield the weak gauge bosons, \Wboson and \Zboson through the mixing of the associated gauge fields. The force carriers of this theory were experimentally detected by the observation of weak neutral current, discovered in 1973 in the Gargamelle bubble chamber located at \ac{CERN} \cite{Hasert:1973ff}, with the masses of and the weak gauge bosons measured by the UA1 and U2 experiments at the \acf{SPS} collider in 1983 \cite{Arnison:1983mk}\cite{Banner:1983jy}.

\subsection{Gauge Symmetries of the SM}

\label{subsec:gaugetheories}


Noether's theorem states that for. Talk about nothers theorem. That a lagrangian invariant under a set of transformations has conserved symmetries. Which are observed nature, charge conversation, lepton num etc etc

The Lagrangian density \lsm of the \ac{SM} is described by 

\begin{equation}
equation here,
\end{equation}

In order to represent the symmetries observed in nature, \lsm must be invariant under a set of gauge transformations. 



\subsection{Electroweak Symmetry Breaking}

\label{subsec:ewsb}
The Electroweak sector is described by the $SU(2)_{L} \times SU(1)_{hyper charge}$ symmetries, contains bosons which have a mass, contrary to the massless gauge bosons inferred by the theories invariance under local gauge transformations. 


\section{Motivation for Beyond the Standard Model Physics}

\label{sec:bsmmotivation}

As previously described, the \ac{SM} is a successful theory, predicting the existence of the \Wboson and \Zboson bosons and the top quark long before they were experimentally observed. However the theory does not fully explain�.


\section{Supersymmmetry}

\label{sec:susytheory}

What is this theory that doesn't exist all about?

\subsection{R-Parity}

\label{subsec:rparity}

R-Parity stuff here innit.

\subsection{ Supersymmetry Breaking}

\label{subsec:susybreaking}

Why are supersymmetric particles not the same mass.


\section{Searching for SUSY at the LHC}
\label{sec:susysearches}

\subsection{Simplified Models}

\label{subsec:sms}

With such a variety of different way for a \ac{SUSY} signal to manifest itself, it is necessary to be able to interpret experimental reach through the masses of gluinos and squarks which can excluded by experimental searches rather than on a model specific basis. This is accomplished through \acf{SMS} models, which are \ac{SUSY} decays which contain only one process. For example the production of a pair of gluinos which then decay via \ac{SM} processes, to a set decay topology with a 100$\%$ branching ratio, shown in �.

Searching and interpreting \ac{SUSY} searches in this way�.

The convention for the naming of these \ac{SMS} models is via the prefix...
