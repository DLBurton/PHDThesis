\chapter{A Theoretical Overview}
\label{chap:theorysection}

Within this chapter, a brief introduction and background to the \ac{SM} is given. Its success as a rigorously tested and widely accepted theory is discussed as well as the deficiencies with this theory that hint there this theory is not a complete description of our universe. The motivations for new physics at the \TeV scale and in particular Supersymmetric theories are outlined within Section \ref{sec:susytheory}, with the chapter concluding with how an experimental signature of such theories can be produced and observed at the \ac{lhc}, Section \ref{sec:susysearches}.

\section{The Standard Model}

\label{sec:thesm}

The \ac{SM} is the name given to the relativistic  \acf{QFT}, where particles are represented as excitations of fields, which describes the interactions and properties of all the known elementary particles \cite{PhysRevLett.19.1264}\cite{Glashow:1961tr}\cite{Salam:1968rm}\cite{Hooft1971167}. It is a renormalisable field theory which contains three symmetries: $SU(3)$ for colour charge, SU(2) for weak isospin and U(1) relating to weak hyper charge, which require its Lagrangian \lsm to be invariant under local gauge transformation. 

Within the \ac{SM} theory, matter is composed of spin \half fermions, which interact with each other via the exchange of spin-1 gauge bosons. A summary of the known fundamental fermions and bosons is given in Table \ref{tab:sm_particles}.

\begin{table}[h!]
\begin{center}
\begin{tabular*}{0.75\textwidth}{@{\extracolsep{\fill}}|c|c|c|c|c|}
\cline{1-5}
Particle                   & Symbol      & Spin & Charge         & Mass (\GeV) \\ \cline {1-5}
\multicolumn{5}{|c|}{\textbf{First Generation Fermions}}  					 \\ \cline{1-5}
Electron Neutrino &$\nu_{e}$   & \half & 0                     &   $< 2.2 \times 10^{-6}$ \\ \cline{1-5}
Electron                  & e                 & \half & -1                    &   $0.51 \times 10^{-3}$ \\ \cline{1-5}
Up Quark                & u                 & \half & $\frac{2}{3}$ &   $2.3 ^{+0.7}_{-0.5} \times 10^{-3}$ \\ \cline{1-5}
Down Quark          & d                 & \half & $-\frac{1}{3}$&   $4.8 ^{+0.7}_{-0.3}\times 10^{-3}$ \\ \cline{1-5}
\multicolumn{5}{|c|}{\textbf{Second Generation Fermions}}  					 \\ \cline{1-5}
Muon Neutrino       &$\nu_{\mu}$& \half & 0                     &   -                           \\ \cline{1-5}
Muon                       & $\mu$       & \half & -1                    &   $1.05 \times 10^{-3}$ \\ \cline{1-5}
Charm Quark        & c                 & \half & $\frac{2}{3}$ &   $1.275 \pm 0.025$ \\ \cline{1-5}
Strange Quark      & s                 & \half & $-\frac{1}{3}$&   $95 \pm 5 \times 10^{-3}$ \\ \cline{1-5}
\multicolumn{5}{|c|}{\textbf{Third Generation Fermions}}  					 \\ \cline{1-5}
Tau Neutrino         &$\nu_{\tau}$& \half & 0                     &   -                           \\ \cline{1-5}
Tau                          & $\tau$       & \half & -1                    &   1.77		 	\\ \cline{1-5}
Top Quark              & t                 & \half & $\frac{2}{3}$ &   $173.5  \pm 0.8$ \\ \cline{1-5}
Bottom Quark        & b                 & \half & $-\frac{1}{3}$&   $4.65  \pm 0.03$ \\ \cline{1-5}
\multicolumn{5}{|c|}{\textbf{Gauge Bosons}}  					 \\ \cline{1-5}
Photon                    &$\gamma$&1   & 0                         &   0                           \\ \cline{1-5}
W Boson                 & $\Wboson$&1& $\pm$1              &   80.385 $\pm 0.015$ \\ \cline{1-5}
Z Boson                 & $\Zboson$& 1 & 0		          &   $91.187  \pm 0.002$ \\ \cline{1-5}
Gluons                   & g                 & 1 & 0			 &   0 \\ \cline{1-5}
Higgs Boson         & H                & 0 & 0			 &   125.3 $\pm 0.5$ \cite{Chatrchyan:2012ufa} \\ \cline{1-5}
\end{tabular*}
\end{center}
\caption[The fundamental particles of the \ac{SM}, with spin, charge and mass displayed.]{The fundamental particles of the \ac{SM}, with spin, charge and mass displayed. Latest mass measurements taken from \cite{pdg2012}. }
\label{tab:sm_particles}

\end{table}

Fermions are separated into quarks and leptons of which only quarks interact with the strong nuclear force. Quarks unlike leptons are not seen as free particles in nature, but rather exist only within baryons, composed of three quarks with an overall integer charge, and quark-anti-quark pairs called mesons. Both leptons and quarks are grouped into three generations which have the same properties, but with ascending mass in each subsequent generation. 

The gauge bosons mediate the interactions between fermions. The field theories of \acf{QED} and \acf{QCD}, yield massless mediator bosons, the photon and eight coloured gluons which are consequences of the gauge invariance of those theories, detailed in Section \ref{subsec:gaugetheories}.  

The unification of the electromagnetic and weak-nuclear forces into the current Electroweak theory yield the weak gauge bosons, \Wboson and \Zboson through the mixing of the associated gauge fields. The force carriers of this theory were experimentally detected by the observation of weak neutral current, discovered in 1973 in the Gargamelle bubble chamber located at \ac{CERN} \cite{Hasert:1973ff}, with the masses of and the weak gauge bosons measured by the UA1 and U2 experiments at the \acf{SPS} collider in 1983 \cite{Arnison:1983mk}\cite{Banner:1983jy}.

\subsection{Gauge Symmetries of the SM}

\label{subsec:gaugetheories}

Symmetries are of fundamental importance in the description of physical phenomena. Noether's theorem states that for a dynamical system, the consequence of any symmetry is an associated conserved quantity \cite{Noether1918}. Invariance under translations, rotations, and Lorentz transformations in physical systems lead to conservation of momentum, energy and angular momentum. 

In the \ac{SM}, a quantum theory described by Lagrangian formalism, the weak,strong and electromagnetic interactions are described in terms of ``gauge theories''.  A gauge theory possesses invariance under a set of ``local transformations", which are transformations whose parameters are space-time dependent. The requirement of gauge invariance within the \ac{SM} necessitates the introduction of force-mediating gauge bosons and interactions between fermions and the bosons themselves. Given the nature of the topics covered by this thesis, the formulation of \ac{EWK}  within the \ac{SM} Lagrangian is reviewed within this section.

The simplest example of the application of the principle of local gauge invariance within the \ac{SM} is in \acf{QED}, the consequences of which require a massless photon field \cite{quarksandleptons}\cite{introtoparticles}. 

Starting from the free Dirac Lagrangian written as 

\begin{equation}
\label{eq:diracequation}
\mathcal{L} = \bar{\psi}(i\gamma^{\mu}\partial_{\mu} - m)\psi,
\end{equation}

where \fermfield represents a free non interacting fermionic field, with the matrices $\gamma^{\mu}$,$\mu \in  0,1,2,3$ defined by the anti commutator relationship $\gamma^{\mu}\gamma^{\nu}  + \gamma^{\mu}\gamma^{\nu} = 2\eta^{\mu\nu}I_{4}$ where $\eta^{\mu\nu}$ is the flat space-time metric $(+,-,-,-)$ and $I_{4}$ is the 4 $\times$ 4 identity matrix. 

Under a local U(1) abelian gauge transformation in which \fermfield transforms as:

\begin{equation}
\fermfield(x) \rightarrow \fermfield^{'}(x) = e^{i\theta(x)}\fermfield(x) \qquad      \bar{\fermfield}(x) \rightarrow \bar{\fermfield}^{'}(x) = e^{i\theta(x)}\bar{\fermfield}(x)
\end{equation}

the kinetic term of the Lagrangian does not remain invariant, due to the partial derivative interposed between the $\bar{\fermfield}$ and \fermfield yielding,

\begin{equation}
\label{eq:remainderterm}
\partial_{\mu}\fermfield \rightarrow  e^{i\theta(x)}\partial_{\mu}\fermfield + ie^{i\theta(x)}\fermfield\partial_{\mu}\theta.
\end{equation} 

To ensure that $\mathcal{L}$ remains invariant, a modified derivative, $D_{\mu}$, that transforms covariantly under phase transformations is introduced. In doing this a vector field $A_{\mu}$ with transformation properties that cancel out the unwanted term in (\ref{eq:remainderterm}) must also be included, 

\begin{equation}
\label{eq:afieldtrans}
D_{\mu} \equiv \partial_{\mu} - ieA_{\mu},   \qquad A_{\mu} \rightarrow A_{\mu} + \frac{1}{e}\partial_{\mu}\theta .
\end{equation}

Invariance of the Lagrangian is then achieved by replacing $\partial_{\mu}$ by $D_{\mu}$:

\begin{align}
\label{eq:covlag}
\mathcal{L} &= i\bar{\fermfield}\gamma^{\mu}D_{\mu}\fermfield - m\bar{\fermfield}\fermfield \nonumber    \\
 &= \bar{\fermfield}(i\gamma^{\mu}\partial_{\mu} - m)\fermfield +  e\bar{\fermfield}\gamma^{\mu}\fermfield A_{\mu}
\end{align}


An additional interaction term is now present in the Lagrangian, coupling the Dirac particle to this vector field, which is interpreted as the photon in \ac{QED}. To regard this new field as the physical photon field, a term corresponding to its kinetic energy must be added to the Lagrangian in (\ref{eq:covlag}). Since this term must also be invariant under (\ref{eq:afieldtrans}), it is defined in the form $F_{\mu\nu} = \partial^{\mu}A^{\nu} - \partial_{\nu}A{\mu}$. 

This then leads to the Lagrangian of \ac{QED}: 

\begin{equation}
\label{qedlagrangian}
\mathcal{L}_{QED} =.\overbrace{i\bar{\fermfield}\gamma^{\mu}\partial_{\mu}\fermfield - \frac{1}{4}F_{\mu\nu}F^{\mu\nu} }^\text{kinetic term} + \overbrace{m\bar{\fermfield}\fermfield}^\text{mass term} + \overbrace{e\bar{\fermfield}\gamma^{\mu}\fermfield A_{\mu} }^\text{interaction term}
\end{equation}

Within the Lagrangian there remains no mass term of the form $m^{2}A_{\mu}A^{\mu}$, which is prohibited by gauge invariance. This implies that the gauge particle, the photon, must be massless.


\subsection{The Electroweak Sector and Electroweak Symmetry Breaking}

\label{subsec:ewsb}
The same application of gauge symmetry and the requirement of local gauge invariance can be used to unify \ac{QED} and the Weak force in the \acf{EWK}. The nature of \ac{EWK} interactions is encompassed within a Lagrangian invariant under transformations of the group $SU(2)_{L} \times U(1)_{Y}$. 

The weak interactions from experimental observation \cite{wu-parity}, are known to violate parity and are therefore not symmetric under interchange of left and right helicity fermions. Thus within the \ac{SM} the left and right handed parts of these fermion fields are treated separately. A fermion field is then split into two left and right handed chiral components, $\fermfield = \fermfield_{L} + \fermfield_{R}$, where $\fermfield_{L\slash R} = (1 \pm \gamma^{5})\fermfield$. 

The $SU(2)_{L}$ group is the special unitary group of $2 \times 2$ matrices $U$ satisfying $UU^{\dagger} = I$ and $det(U) = 1$. It may be written in the form $U = e^{-i\omega_{i}T_{i}}$, with the generators of the group $T_{i} = \frac{1}{2}\tau_{i}$ where $\tau_{i}$, $i \in$ 1,2,3 being the $2 \times 2$ Pauli matrices

\begin{equation}
\tau_{1}= \left( \begin{array}{ccc}
0 & 1\\
1 & 0\end{array} \right)\qquad
\tau_{2}= \left( \begin{array}{ccc}
0 & -i\\
i & 0\end{array} \right)\qquad
\tau_{3}= \left( \begin{array}{ccc}
1 & 0\\
0 & -1\end{array} \right),
\end{equation}


which form a non Abelian group obeying the commutation relation $[T^{a},T^{b}] \equiv  if^{abc}T^{c} \neq 0$. The gauge fields that accompany this group are represented by $\hat{W}_{\mu}  = (\hat{W}^{1}_{\mu},\hat{W}^{2}_{\mu},\hat{W}^{3}_{\mu}$  ) and act only on the left handed component of the fermion field $\fermfield_{L}$.

One additional generator $Y$ which represents the hypercharge of the particle under consideration is introduced through the $U(1)_{Y}$ group acting on both components of the fermion field, with an associated vector boson field $\hat{B}_{\mu}$.  

The $SU(2)_{L} \times U(1)_{Y}$ transformations of the left and right handed components of $\fermfield$ are summarised by,

\begin{align}
\label{eq:su2xu1transform}
 & \chi_{L} \rightarrow \chi^{'}_{L} = e^{i\theta(x) \cdot T + i\theta(x)Y}\chi_{L}, \nonumber \\
 & \fermfield_{R} \rightarrow \fermfield^{'}_{R} = e^{i\theta(x)Y}\fermfield_{R}, 
\end{align}

where the left handed fermions form isospin doubles $\chi_{L}$ and the right handed fermions are isosinglets $\fermfield_{R}$. For the first generation of leptons and quarks this represents

\begin{align}
\label{firstdoublet}
 \chi_{L} &= \left( \begin{array}{c} 
\nu_{e} \\
e \end{array} \right)_{L}, \qquad
 \left( \begin{array}{c} 
u  \\
d \end{array} \right)_{L} \nonumber \\
 \fermfield &= e_{R}, \qquad \qquad u_{R}, d_{R}
\end{align}

Imposing local gauge invariance within $\mathcal{L}_{EWK}$ is once again achieved by modifying the covariant derivative

\begin{equation}
D_{\mu} = \partial_{\mu} - \frac{ig}{2}\tau^{i}W^{i}_{\mu} - \frac{ig^{'}}{2}Y B_{\mu},
\end{equation}

where $g$ and $g^{'}$ are the coupling constant of the $SU(2)_{L}$ and $U(1)_{Y}$ groups respectively. Taking the example of the first generation of fermions defined in Eq.\ref{firstdoublet}, this would lead to a lagrangian $\mathcal{L}_{1}$ of the form,

\begin{equation}
\mathcal{L}_{1} =  . 
\end{equation}

As in \ac{QED}, these additional gauge fields introduce field strength tensors $B_{\mu\nu}$ and $W^{\theta}_{\mu\nu}$,

\begin{align}
\hat{B}_{\mu\nu} &=  \partial_{\mu}\hat{B}_{\nu} - \partial_{\nu}\hat{B}_{\mu} \\
\hat{W}^{\theta}_{\mu\nu} &=  \partial_{\mu}\hat{W}_{\nu} - \partial_{\nu}\hat{W}_{\mu} - g\hat{W}_{\mu}\times\hat{W}_{\mu}
\end{align}

corresponding to the kinetic energy and self coupling of the $W_{\mu}$ fields and the kinetic energy term of the $B_{\mu}$ field.

None of these gauge bosons are physical particles, and instead linear combinations of these gauge bosons make up $\gamma$ and the W and Z bosons, defined as 

\begin{equation}
W^{\pm} = \frac{1}{\sqrt{2}} \left(W^{1}_{\mu} \mp iW^{2}_{\mu}\right) \qquad   
\left( \begin{array}{c} 
Z_{\mu} \\
A_{\mu} \end{array} \right)   = 
\left( \begin{array}{cc} 
cos\theta_{W} & -sin\theta_{W} \\
sin\theta_{W} & cos\theta_{W} \end{array} \right) 
\left( \begin{array}{c} 
W^{3}_{\mu} \\
B_{\mu} \end{array} \right)
\end{equation}

where the mixing angle, $\theta_{w} = \tan ^{-1} \frac{g^{'}}{g}$ , relates the coupling of the neutral weak and electromagnetic interactions. 

As in the case of the formulation of the \ac{QED} Lagrangian there remains no mass term for the photon. However this is also the case for the W, Z and fermions in the, contrary to experimental measurement. Any explicit introduction of mass terms would break the symmetry of the Lagrangian and instead mass terms can be introduced through spontaneous breaking of the \ac{EWK} symmetry via the Higgs mechanism.

Talk about higgs sector


\section{Motivation for Beyond the Standard Model Physics}

\label{sec:bsmmotivation}

As previously described, the \ac{SM} is a successful theory, predicting the existence of the \Wboson and \Zboson bosons and the top quark long before they were experimentally observed. However the theory does not fully explain�.


\section{Supersymmmetry}

\label{sec:susytheory}

What is this theory that doesn't exist all about?

\subsection{R-Parity}

\label{subsec:rparity}

R-Parity stuff here innit.

\subsection{ Supersymmetry Breaking}

\label{subsec:susybreaking}

Why are supersymmetric particles not the same mass.


\section{Searching for SUSY at the LHC}
\label{sec:susysearches}

\subsection{Simplified Models}

\label{subsec:sms}

With such a variety of different way for a \ac{SUSY} signal to manifest itself, it is necessary to be able to interpret experimental reach through the masses of gluinos and squarks which can excluded by experimental searches rather than on a model specific basis. This is accomplished through \acf{SMS} models, which are \ac{SUSY} decays which contain only one process. For example the production of a pair of gluinos which then decay via \ac{SM} processes, to a set decay topology with a 100$\%$ branching ratio, shown in �.

Searching and interpreting \ac{SUSY} searches in this way�.

The convention for the naming of these \ac{SMS} models is via the prefix...
