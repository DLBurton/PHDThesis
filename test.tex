A search for supersymmetric particles in events with high transverse momentum jets and a large missing transverse energy signature, is conducted using 11.7 fb$^{-1}$ of data, collected with a center-of-mass collision energy of 8 \TeV by the CMS detector. The dimensionless kinematic variable \alphat is used to select events with genuine missing transverse energy signatures. Standard Model backgrounds are estimated through the use of data driven control samples. 
No excess over Standard Model expectations is found. Exclusion limits on squark and gluino masses are set at the 95\% confidence level in the parameter space of a range of supersymmetric simplified model topologies.  \\ 

Results of benchmarking the Level-1 (the first line of the CMS trigger system) single jet and hadronic transverse energy trigger efficiencies, before and after the implementation of a change to the Level-1 jet clustering algorithm are presented. Similar performance is observed for all L1 quantities. This change was introduced to negate an increase in trigger cross-section, which can be attributed to soft jets from secondary interactions. \\

Furthermore, a templated fit method to estimate the Standard Model background distribution of the number of jets originating from a b-quark within a supersymmetric search, is validated in data and simulation. Applicable to searches sensitive to gluino induced third-generation signatures, this technique is utilised as a crosscheck to the results of the \alphat analysis. Standard Model background predictions from the template fits are compared to those from the \alphat search in the hadronic signal region, where good agreement between the two methods is observed. 