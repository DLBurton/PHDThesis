\chapter{Introduction}
\label{chap:introduction}

During the 20th century, great advances were made in the human understanding of the universe, its origins, its future and its composition. The \acf{SM} first formulated in the 1960s is one of the crowning achievements in science's quest to explain the most fundamental processes and interactions that make up our universe. It has provided a highly successful explanation for a wide range of phenomena in Particle Physics and has stood up to extensive experimental scrutiny \cite{pdg2012}.

Despite its success it is not  a complete theory, with significant questions remaining unanswered. It describes only three of the four known forces with gravity not incorporated within the framework of the \ac{SM}. Cosmological experiments infer that just $\sim$5$\%$ of the observable universe exists as matter, with elusive ``Dark Matter'' accounting for a further $\sim$27$\%$ \cite{Ade:2013zuv}. However no particle predicted by the \ac{SM} is able to account for it.  At higher energy scales, the (non-)unification of the fundamental forces point to problems with the \ac{SM} at least at higher energies not yet probed experimentally. 

Many theories exist as extensions to the \ac{SM}, predicting a range of observables that can be detected at the \acf{lhc}, of which \acf{SUSY} is one such example. It predicts a new symmetry of nature in which all current particles in the \ac{SM} would have a corresponding supersymmetric partner. Common to most Supersymmetric theories is a stable, weakly interacting \acf{LSP}, which has the properties of a possible dark matter candidate. The \ac{SM} and the main principles of Supersymmetric theories are outlined in Chapter \ref{chap:theorysection}, with emphasis placed on how experimental signatures of \ac{SUSY} may reveal themselves in proton collisions at the \ac{lhc}.

The experimental goal of the \ac{lhc} is to further test the framework of the \ac{SM}, exploring the \TeV mass scale for the first time, and to seek a connection between the particles produced in proton-proton collisions and dark matter. The first new discovery by this extraordinary machine was announced on the 4th of July 2012. The long-awaited discovery was the culmination of decades of experimental endeavours in the search for the Higgs boson, which provided an answer to the mechanism of electroweak symmetry breaking within the \ac{SM} \cite{Aad:2012tfa}\cite{Chatrchyan:2012ufa}. 

This discovery was made possible through the combination of data taken by the \acf{CMS} \cite{cmstdr} and \ac{ATLAS} \cite{atlastdr}, two multipurpose detectors located on the \ac{lhc} ring. An experimental description of the \ac{CMS} detector and the \ac{lhc} is described in Chapter \ref{chap:cmsoverview}, including object reconstruction and identification used by \ac{CMS} in searches for \ac{SUSY} signatures. 

The performance of the \ac{CMS} Level-1 single jet and energy sum triggers is benchmarked within Chapter \ref{subsec:l1trigger}. The Level-1 trigger is the first line of the \ac{CMS} trigger system and is of paramount importance to the collection of physics events. A change in the jet clustering algorithm, via the introduction of a jet seed threshold, was introduced approximately half way through the data taking period. The aim of this change, was to reduce the rate at which collisions not of interest to physics analysis were recorded, whilst avoiding impact to the overall performance of the triggers.

Chapter \ref{chap:SUSYsearches}, contains a description of the search for direct evidence of the production of supersymmetric particles at the \ac{lhc}. The main basis of the search centres around the kinematic dimensionless \alphat variable, which provides a strong rejection of backgrounds with fake missing transverse energy signatures, whilst maintaining good sensitivity to a variety of \ac{SUSY} topologies. The author's work (as an integral part of the analysis group) is documented in detail, and has culminated in numerous publications over the past two years, the latest results having been published in the \acf{EPJC} \cite{ra1_epjc}. 

The author in particular has played a major role in the extension of the \alphat analysis into additional b-tagged jet (jets identified as originating from a b-quark) and jet multiplicity dimensions, increasing the sensitivity of the analysis to a range of \ac{SUSY} topologies. Additionally, the author has worked extensively on increasing the statistical precision of the data driven electroweak predictions through analytical techniques. This included work on developing the derivation of data driven systematic uncertainties through the establishment of closure tests within the control samples of the analysis. 

The compatibility of the data collected for the \alphat search with a \ac{SM} only hypothesis is documented in Chapter \ref{chap:SUSYresults}. In the absence of an observed excess, interpretations of the data within the framework of a variety of \acf{SMS}, describing an array of possible \ac{SUSY} event topologies are made.

Finally, a method to search for gluino mediated \ac{SUSY} signatures rich in top and bottom flavoured jet final states is introduced in Chapter \ref{chap:templatemethod}. These particular \ac{SUSY} topologies are increasingly of interest to physicists in light of the discovery of the Higgs boson. A parametrisation of the b-tagged jet distribution for different electroweak processes is used to establish template shapes, which are then fitted at low b-tagged jet multiplicity,  to extrapolate an expected \ac{SM} background of 3 and 4 b-tagged jet events within an event sample. The \alphat control and hadronic signal event selections are used to validate the functionality of this template method in both data and simulation. Background predictions within the hadronic signal region are compared to those presented in Chapter \ref{chap:SUSYresults}, with the intention of serving as an independent crosscheck of the estimated \ac{SM} backgrounds from the \alphat search. 

Natural units are used throughout this thesis in which \hbarred = c = 1.
