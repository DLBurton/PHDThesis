\chapter{Introduction}
\label{chap:introduction}

During the 20th century great advances have been made in our understanding of the universe, where it comes from, where it is going and what it is made of. The Standard Model (\SM) first formulated in the 1960's is one of the crowning achievements in science's quest to explain the most fundamental processes and interactions that make up our universe. It has provided a highly successful explanation of a wide range of phenomena in Particle Physics and has stood up to extensive experimental scrutiny \cite{pdg2012}.

Despite it's successes it is not  a complete theory, with significant questions remaining unanswered. It describes only three of the four known forces with gravity not incoorporated within the framework of the \SM. Cosmological experiments infer that just $\sim$ 4$\%$ of the observable universe exists as matter, with elusive ``Dark Matter'' accounting for a further $\sim$ 23$\%$. However no particle predicted by the \SM is able to account for it.  At higher energy scales and small distances the (non-)unification of the fundamental forces point to problems with the \SM at least at higher energies not yet probed experimentally. 

Many theories exist as extensions to the \SM and predict a range of observables that can be observed at the Large Hadron Collider (\LHC) of which Supersymmetry (\SUSY) is one such example. A possible extension to the \SM, it predicts a new symmetry of nature in which all current particles in the \SM would have a corresponding supersymmetric partner. Common to most Supersymmetric theories is a stable, weakly interacting Lightest Supersymmetric Partner (\LSP), which has the properties of a possible dark matter candidate. The \SM and the main principles of Supersymmetric theories are outlined in Chapter \ref{chap:theorysection}, with emphasis on placed on how experimental signatures of \SUSY may reveal themselves at the \LHC.

The experimental goal of the \LHC is to further test the framework of the \SM, exploring the \TeV mass scale for the first time, and to seek a connection between the particles produced in proton collisions and dark matter. The first new discovery by this extraordinary machine was announced on the 4th of July 2012. The long-awaited discovery was the culmination decades of experimental endeavours in the search for the Higgs boson, providing an answer to the mechanism of electroweak symmetry breaking within the \SM. 

This discovery was made possible through data taken by the two multi purpose detectors (\CMS and \ATLAS) located on the \LHC ring. An experimental description of the \CMS detector and the \LHC is described in Chapter \ref{chap:cmsoverview}, including some of the object reconstruction used by \CMS in searches for \SUSY signatures. The performance of the \CMS Level-1 calorimeter trigger, benchmarked by the author is also included within this chapter.

The analysis conducted by the author is detailed within Chapter \ref{chap:SUSYsearches}. This chapter contains a description of the search for evidence of the production of Supersymmetric particles at the \LHC. The main basis of the search centres around the kinematic dimensionless \alphat variable, which provides strong rejection of backgrounds with fake missing energy signatures whilst maintaining good sensitivity to a variety of \SUSY topologies. The author's work as an integral part of the analysis group is documented in detail, which has culminated in numerous publications over the past two years. The latest of which was published in JHEP (cite this) and contains the results which are discussed within this and the following Chapters.

Also included within this Chapter is a method to search for \SUSY signatures which are rich in top and bottom flavoured jet final states. A parametrisation of the b-tagging distribution for different Electroweak processes is used to establish templates, which are then used to predict the expected number of 3 or 4 b-tagged jet events from \SM processes. The \alphat search is used as a cross check for the template method to establish it's functionality. 

Finally the interpretation of such results within the framework of a variety of simplified models (\SMS), which describe an array of possible \SUSY event topologies is documented in Chapter \ref{chap:SUSYresults}. A description of the statistical model used to derive these interpretations and the possible implications of results presented in this thesis is discussed within this Chapter. Natural units are used throughout this thesis in which \hbarred = c = 1.