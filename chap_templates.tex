\chapter{SUSY Searches with B-tag Templates}
\label{chap:templatemethod}


Within this chapter a complementary technique is discussed as a means to predict the distribution of three and four reconstructed b-tagged (\nbreco = 3, 4), jets in an event sample. The recent discovery of the Higgs boson has made ``Natural \ac{SUSY}'' models attractive, given that light top and bottom squarks are a candidate to stabilise divergent loop corrections to the Higgs boson mass. A light gluino which subsequently decays to third generation sparticle pairs, will give rise to many events with a large number of final state b-tagged jets.

The method described within this chapter is used to estimate the \ac{SM} background at high b-tagged jet multiplicities (3-4), from a templated fit conducted in a low b-tagged jet (0-2) control region of an event sample. This approach can hypothetically be applied to generic supersymmetric searches, to gain sensitivity to signals which contain a higher number of b-tagged jets than the search's dominant \ac{SM} backgrounds. 

As a proof-of-concept, the procedure is applied to the \ac{SM} enriched \mupjets control sample of the \alphat search detailed in Chapter \ref{chap:SUSYsearches}, and validated in both data and simulation. This method is then further utilised to provide an independent crosscheck of the \ac{SM} background estimations determined by the \alphat search within its hadronic signal region at high b-tagged jet multiplicities.

To highlight the relative insensitivity of this method to the choice of b-tagging algorithm working point, results are presented using the \ac{CSV} tagger (introduced in Section (\ref{subsec:cmsobjects-btagging})) for the ``Loose'', ``Medium'' and ``Tight'' working points.

\section{Concept}
\label{sec:templateconcept}

The dominant \ac{SM} backgrounds of most \ac{SUSY} searches are typically \ttbar + jets, W + jets, \zinv + jets or other rare processes (e.g. Diboson, $\ttbar W +$ jets production in the case of hadronic searches) with neutrinos in the final state. These processes are characterised by typically having zero or two underlying b-quarks per event as shown in Table \ref{tab:bquarkcontent}. This ultimately means that the resultant shape of the \nbreco distribution for these two types of event topologies will differ significantly due to the varying tagging probabilities of the different jet flavours present in the final state of these processes.  

Similarly, \ac{SMS} models comprising the gluino-mediated production of third generation squarks, such at the \texttt{T1tttt} and \texttt{T1bbbb} models described in the previous chapter, will contain four underlying b-quarks in its decay. Therefore the resultant shape of the \nbreco distribution from such a signal will be further skewed towards a higher number of b-tagged jets. As \ac{SM} processes with a similarly large number of underlying b-quarks are rare, a signal indicative of natural \ac{SUSY} can potentially be easily identified, via an observed excess of \nbreco = 3, $\geq$ 4 events with respect to the expected yields from \ac{SM} processes.
 
 \begin{table}[h!]
\begin{center}
\footnotesize
\begin{tabular*}{0.65\textwidth}{@{\extracolsep{\fill}}cl}
\hline
Typical underlying b-quark content & Process \\
\hline\hline
 = 0 & $W \rightarrow l\nu$  + jets \\
   & \zinv  + jets  \\
   & $Z/\gamma^{*} \rightarrow \mu\mu$ + jets \\
 \\
 = 1 & $t$ + jets  \\
 \\
= 2 & \ttbar + jets
\end{tabular*}
\end{center}
\caption[Typical underlying b-quark content of different \ac{SM} processes which are common to many \ac{SUSY} searches.]{Typical underlying b-quark content of different \ac{SM} processes which are common to many \ac{SUSY} searches.}
\label{tab:bquarkcontent}
\end{table}

Within a supersymmetric or indeed any search for new physics, the compatibility of the \nbreco distribution in data with \ac{SM} expectations can be tested, via the shape parameterisation of the \ac{SM} background \nbreco distribution, grouped in terms of these two most common underlying b-quark topologies. 

Two templates, representing processes which have an underlying b-quark content of zero or two are defined as Z0 and Z2 respectively (single top processes are a negligible background, $< 1\%$, within the \alphat search to which this method is applied in the following section, and are thus incorporated within the Z2 template). \ac{SM} background estimates at high \nbreco multiplicities can then be extrapolated from the fitting of these two template shapes in a low \nbreco control region (0-2) under the assumption of negligible signal contamination.

The simplest way to determine the shapes of the \nbreco distributions for both templates would be, after the application of the relevant event selection, to take the \nbreco distribution as given directly from simulation. However as discussed within Section (\ref{subsec:backgroundestimation}), there are large statistical uncertainties in simulation at high $n_{b}^{reco}$ multiplicities (which is the region in which we wish to use the templates to estimate the \ac{SM} backgrounds). This statistical uncertainty is particularly pronounced for processes incorporated within the Z0 templates, where events with a large number reconstructed b-tagged jets stem largely from the mis-tagging of all the light-flavoured jets in the final state. Therefore to improve the statistical precision of the final background prediction at high b-tagged jet multiplicities, the formula method first introduced in Section (\ref{subsec:formulamethod}) is utilised to generate the template shapes. 

The template shapes of each analysis category (\theht and \njet in the case of the \alphat analysis) are dependant upon the jet-flavour content and b-tagging rate within the phase space of interest, with the tagging probabilities of a jet being a function of the jet \pt, the pseudo-rapidity $\rvert\eta\lvert$, and jet-flavour. This can be seen in Figure \ref{fig:templatetaggingefficiencies}, where the tagging efficiency of jets identified from truth information as stemming from the hadronisation of a b, c or light quark is shown for the three working points of the \ac{CSV} tagger as a function of jet \pt. 

\begin{minipage}{\textwidth}
\centering
\begin{minipage}[b]{0.48\linewidth}
\includegraphics[width = 1.0\linewidth]{plots/bjet_PtDistribution_Htbin_Template_375.pdf}
\centering (a)  b-jets
\end{minipage}
\quad
\begin{minipage}[b]{0.48\linewidth}
\includegraphics[width = 1.0\linewidth]{plots/cjet_PtDistribution_Htbin_Template_375.pdf}
\centering (b) c-jets
\end{minipage}
\end{minipage}
\quad
\begin{figure}[ht]
\begin{minipage}[b]{0.48\linewidth}
\centering
\includegraphics[width = 1.0\linewidth]{plots/lighjet_PtDistribution_Htbin_Template_375.pdf}
\centering (c) light-jets
\end{minipage}
\caption[The b-quark (a), c-quark  (b), and light-quark (c$)$ tagging efficiency as a function of jet \pt, measured in simulation after the application of the \alphat analysis \mupjets control sample selection, in the region \theht $>$ 375.]{The b-quark (a), c-quark  (b), and light-quark (c$)$ tagging efficiency as a function of jet \pt, measured in simulation after the application of \alphat analysis \mupjets control sample selection, in the region \theht $>$ 375. Efficiencies are measured for the three \ac{CSV} working points.}
\label{fig:templatetaggingefficiencies}
\end{figure}
\FloatBarrier

Therefore, before the template shapes are generated by the formula method, the jet \pt and \eta averaged tagging efficiencies of each jet flavour are calculated for the phase space of interest. Additionally, the relevant jet \pt and \eta corrections are applied to correct the measured b-tagging rate in simulation to that of data, as specified in Section (\ref{subsec:formulamethodsf}). These corrections propagate through to the average determined tagging efficiency for each jet flavour, consequently affecting the final Z0 and Z2 template shape of the \nbreco distribution, determined within each analysis category (\theht and \njet in the case of the \alphat search). 

Using the truth-level flavour information of each of the defined Z0 and Z2 templates and the measured tagging efficiencies of each jet flavour, the template shapes are constructed from simulation via the formula method. These two shapes are then fitted to data in a low $n_{b}^{reco}$ control region (0-2), by allowing the normalisation constants $\theta_{Z0}$ and $\theta_{Z2}$ of the two templates to float. The fits are performed independently within each of the defined analysis category to remove any dependence on the modelling of jet multiplicity between simulation and data. Best fit values of $\theta_{Z0}$ and $\theta_{Z2}$ are used, along with the fixed shape of each template, to extrapolate a \ac{SM} background estimation within the high $n_{b}^{reco}$ signal region (3,4) as shown in Figure \ref{fig:templateexample}. 

 \begin{figure}[!h]
 \centering
\includegraphics[width=0.60\textwidth]{plots/Template_Example.pdf}
\caption[An example of a template fit with the defined Z0 (blue) and Z2 (red) templates to data within the low \nbreco control region (left).]{An example of a template fit with the defined Z0 (blue) and Z2 (red) templates to data within the low \nbreco control region (left). The shape of the two templates are fixed but the normalisations $\theta_{Z0}$ and $\theta_{Z2}$ are allowed to vary. The best fit values are then applied to extrapolate a combined background prediction from the shaded signal region (right), represented by the dashed black line. Statistical and systemic uncertainties are not shown within this figure.}  
\label{fig:templateexample}
\end{figure}

Any large excess in data is an indication that the \nbreco distribution is not adequately described by the \ac{SM} backgrounds encapsulated by the templates. This could mean there are additional \ac{SM} backgrounds that fall within the selection of the analysis that need to be considered, or that there is signal present within the data. This method relies solely on fitting to the shape of the $n_{b}^{reco}$ distribution, and can in principle, be applied to any analysis where the signal hypothesis has a larger underlying b-quark spectra than the \ac{SM} backgrounds. 

However, in the scenario where a \ac{SUSY} signal sits at a low number of underlying b-quarks, the template would be unable to discriminate between this signal and background during the fit in the control region. This will be the case unless the jet \pt distribution of the signal and background were drastically different, in which case there would anyway be many more sensitive and practical ways to establish the presence of a signal in the data than this method. Indeed the template method is only really applicable to the hypothesis that any signal resides at high \nbreco and that the control region 0 $\leq \nbreco \leq 2$ has negligible signal contamination.  

\FloatBarrier
\section{ Application to the \alphat Search}
\label{sec:templateapplication}

As detailed in the previous chapter, the \alphat analysis is a search for \ac{SUSY} particles in all-hadronic final states, utilising the kinematic variable \alphat to suppress QCD to a negligible level. \ac{SM} enriched control samples are used to estimate the background within an all-hadronic signal region. 

The selection for the \mupjets control samples defined in Section (\ref{subsec:controlsampledefinition}) is used to demonstrate the template fitting procedure both conceptually in simulation, and also when applied in data. This is chosen, as such a selection is dominated by events stemming from the \ac{SM} processes with little or no signal contamination from potential new physics. Contributions from rare \ac{SM} processes with a higher underlying b-quark content (e.g. $t\bar{t}b\bar{b}$) are also found to be negligible from studies in simulation. For these reasons, there is a degree of confidence that the procedure should adequately describe the observations in data when extrapolated to the signal region.

As a departure from the \alphat search strategy described in the previous section, events are categorised according to jet multiplicity categories of 3, 4 and $\geq$ 5 reconstructed jets per event (di-jet events are not included as there is no contribution to the high $n_{b}^{reco}$ region (3,4)), in order to reduce the kinematic range of the jet \pt's within each category. Furthermore the analysis is split into just three \theht regions, for the purpose of increasing statistics within the control region, 

\begin{itemize}
\item 275-325 \GeV
\item 325-375 \GeV
\item $>$ 375 \GeV
\end{itemize}

contrary to the eight used within the \alphat analysis. Templates for both underlying b-quark content hypotheses are then generated for the nine defined event categories.

\subsection{Proof of Principle in Simulation}
\label{subsec:templateclosuretest}

This template procedure must be first demonstrated to work within simulated events free from any potential signal contamination before it can be applied to data. By combining the relevant ingredients necessary to employ the formula method, \nbreco shape templates are generated individually for each \njet and \theht category using one half of the available simulated events for each \ac{SM} process. In this case, as the template shapes are being fitted to simulation, it is \emph{not} necessary to apply the relevant corrections of the b-tagging rates between data and simulation. 

The other half of simulated events is utilised to provide a statistically independent sample from which the \nbreco distribution is taken directly. The two generated templates are then fit within the low \nbreco (0-2) control region to this pseudo-data, from which a signal region prediction is then extrapolated from the template best fit values.  The simulated event samples in the analysis are split in this way to allow for statistically independent fits to be performed. 

The aim of this procedure is to ensure that the template fit can accurately extrapolate the \nbreco distribution within the defined signal region from two independent but kinematically identical samples. Furthermore, as the pseudo-data of the \nbreco distribution is taken directly from simulation, observation of good closure for both the initial fit of the two templates within the control region and after extrapolation to the signal region will serve as a validation of the formula method in recovering the original $n_{b}^{reco}$ distribution itself. 

Results are presented in Figure \ref{fig:template_closure_njet5} for each \ac{CSV} working point in the $\njet \geq 5$ category, using the \mupjets control sample selection and the inclusive \theht$>$ 375 \GeV analysis bin. Additional fit results for other \njet categories which show a similar level of closure can be found within Appendix \ref{app:templatemc}.  The grey bands represent the statistical uncertainty on the template shapes as derived from adding in quadrature the statistical uncertainty on the $\theta_{Z0}$ and $\theta_{Z2}$ values from the fit performed, the statistical uncertainty of the measured tagging efficiencies for each jet flavour, and the statistical uncertainty of the template shapes from the formula method.

\begin{figure}[ht]
\footnotesize
\vspace{5mm}
\centering
\begin{minipage}{.51\textwidth}
\centering
\includegraphics[width = 1.0\linewidth]{plots/ThesisPlots/Final_Fit_To_MC_Normal_Loose_HTBin_OneMuon_Template_375_jet_mult_5.pdf}
\centering (a) Loose working point : $n_{jet} \geq$  5 
\end{minipage}
\quad
\begin{minipage}[b]{0.51\linewidth}
\includegraphics[width = 1.0\linewidth]{plots/ThesisPlots/Final_Fit_To_MC_Normal_Medium_HTBin_OneMuon_Template_375_jet_mult_5.pdf}
\centering (b) Medium working point : $n_{jet} \geq$ = 5 
\end{minipage}
\footnotesize
\centering
\begin{minipage}[b]{0.51\linewidth}
\centering
\includegraphics[width = 1.0\linewidth]{plots/ThesisPlots/Final_Fit_To_MC_Normal_Tight_HTBin_OneMuon_Template_375_jet_mult_5.pdf}
\centering (c) Tight working point : $n_{jet} \geq$ 5 
\end{minipage}
\caption[Results of fitting the Z = 0 and Z = 2 templates in the $n_{b}^{reco}$ = 0-2 control region to yields from simulation in the \mupjets control sample for the \theht $>$ 375 \GeV, $n_{jet} \geq 5$ category for all \ac{CSV} working points.]{Results of fitting the Z = 0 and Z = 2 templates in the $n_{b}^{reco}$ = 0-2 control region to yields from simulation in the \mupjets control sample for the \theht $>$ 375 \GeV, $n_{jet} \geq 5$ category  for all \ac{CSV} working points. Data is represented by the black circles with the blue, red and black lines representing the Z=0, Z=2 and combination of both templates respectively. Grey bands represent the uncertainty of the fit. The $\chi^{2}$ parameters represent the goodness of fit to the control and signal region.}
\label{fig:template_closure_njet5}
\end{figure}
\FloatBarrier

The extrapolated fit predictions summed over all \njet multiplicities within the high $n_{b}^{reco}$ signal region, are summarised for all \theht bins and working points in Table \ref{tab:template_mctable}. 

\begin{table}[h!]
\begin{center}
\footnotesize
\begin{tabular*}{0.95\textwidth}{@{\extracolsep{\fill}}llccc}
\cline{1-5}
\multicolumn{2}{c}{\theht} & 275-325 & 325-375 & $>$375 \\

\multicolumn{5}{c}{Loose working point} \\
\hline\hline
Simulation & \multirow{2}{*}{$n_{b} = 3$} & $793.0 \pm 14.8$ & $387.9 \pm 10.2$ & $794.1 \pm 14.34$ \\
Template & & $820.4 \pm 26.7$ & $376.3 \pm 11.9$ & $780.1 \pm 15.1$ \\
Simulation & \multirow{2}{*}{$n_{b} = 4$} & $68.2 \pm 3.9$ & $27.6 \pm 2.7$ & $91.3 \pm 4.9$ \\
Template & & $72.5 \pm 4.7$ & $28.3 \pm 2.34$ & $84.4 \pm 3.8$ \\
\hline
\multicolumn{5}{c}{Medium working point} \\
\hline\hline
Simulation & \multirow{2}{*}{$n_{b} = 3$} & $133.7 \pm 5.7$ & $74.5 \pm 4.5$ & $164.2 \pm 6.4$ \\
Template &  & $132.8 \pm 4.8$ & $74.5 \pm 3.9$ & $159.9 \pm 5.7$ \\
Simulation & \multirow{2}{*}{$n_{b} = 4$} & $1.6 \pm 0.6$ & $0.6 \pm 0.4$ & $3.4 \pm 0.9$ \\
Template & & $1.8 \pm 0.2$ & $1.1 \pm 0.2$ & $4.1 \pm 0.4$ \\
\hline
\multicolumn{5}{c}{Tight working point} \\
\hline\hline
Simulation & \multirow{2}{*}{$n_{b} = 3$} & $26.9 \pm 2.6$ & $13.9 \pm 1.9$ & $31.8 \pm 2.9$ \\
Template & & $24.7 \pm 1.5$ & $13.8 \pm 1.2$ & $28.1 \pm 1.5$ \\
Simulation & \multirow{2}{*}{$n_{b} = 4$} & $0.5 \pm 0.4$ &  -  & - \\
Template & & $0.1 \pm 0.1$ & $0.1 \pm 0.1$ & $0.2 \pm 0.1$ \\
\end{tabular*}
\end{center}
\caption[Summary of the fit predictions in the $n_{b}^{reco}$ signal region after combination of the $n_{jet} = 3, = 4, \geq 5$ categories compared against yields taken directly from simulation. The predictions are extrapolated from a $n_{b}^{reco}$ = 0, 1, 2 control region and simulation yields are normalised to an integrated luminosity of 10 fb$^{-1}$. ]{Summary of the fit predictions in the $n_{b}^{reco}$ signal region after combination of the $n_{jet} = 3, = 4, \geq 5$ categories compared against yields taken directly from simulation. The fit predictions are extrapolated from a $n_{b}^{reco}$ = 0, 1, 2 control region and simulation yields are normalised to an integrated luminosity of 10 fb$^{-1}$. The uncertainties quoted on the template yields are purely statistical.}\label{tab:template_mctable}
\end{table}

The pull distributions for all the fits performed can be found in Appendix \ref{app:templatepulldistributions}, and are compatible with a mean of zero and standard deviation of one, showing no obvious bias to the fitting procedure. Each of the fits performed show good compatibility between the template shapes and data from simulation within the defined control region, with additional good overall agreement also observed for extrapolation to the signal region as shown in Table \ref{tab:template_mctable}. This validates both the formula method used in the generation of the template shapes as well as the method of predicting the \ac{SM} background in the high \nbreco signal region. 

The application of this method to the same selection in a data control sample is now used to demonstrate necessary control over the efficiency and mis-tagging rates when b-tagging scale factors are applied, and to test the assumption of no signal contamination with the \mupjets control sample.

\subsection{Results in a Data Control Sample}
\label{subsec:templatedataresults}

The procedure is now applied to the 2012 8 \TeV dataset in the \mupjets control sample, to establish the validity of this method in data. The relevant data to simulation b-tagging scale factors are applied to produce corrected values of the efficiency and mis-tagging rates within each analysis category \cite{btagscalefactor}. 

Figure \ref{fig:template_data_med_njet5} shows the results of the templates derived from simulation to each of the three defined \theht bins, in the $n_{jet} \geq 5$ category for the medium working point \ac{CSV} tagger (the same working point used within the \alphat analysis).  Grey bands represent the previously detailed statistical uncertainty of the fit combined in quadrature with the systematic uncertainties of varying the data to simulation scale factors by their b-tag scale factor systematic uncertainties. These b-tag scale factor uncertainties are to be taken as fully correlated across all jet flavours \cite{btagscalefactor}. Additional fit results for other jet multiplicities are found in Appendix \ref{app:templatedata}.

\begin{minipage}{\textwidth}
\footnotesize
\centering
\begin{minipage}[b]{0.50\linewidth}
\includegraphics[width = 1.0\linewidth]{plots/ThesisPlots/Final_Fit_To_Data_Normal_Medium_HTBin_OneMuon_275_325_jet_mult_5.pdf}
\centering (a) $n_{jet} \geq$  5 , 275 $<$ \theht $<$ 325
\end{minipage}
\begin{minipage}[b]{0.50\linewidth}
\includegraphics[width = 1.0\linewidth]{plots/ThesisPlots/Final_Fit_To_Data_Normal_Medium_HTBin_OneMuon_325_375_jet_mult_5.pdf}
\centering (b) $n_{jet} \geq$ = 5 , 325 $<$ \theht $<$ 375 
\end{minipage}
\end{minipage}
\begin{figure}[ht]
\footnotesize
\centering
\begin{minipage}[b]{0.51\linewidth}
\centering
\includegraphics[width = 1.0\linewidth]{plots/ThesisPlots/Final_Fit_To_Data_Normal_Medium_HTBin_OneMuon_Template_375_jet_mult_5.pdf}
\centering (c) $n_{jet} \geq$ 5 , \theht $>$ 375 
\end{minipage}
\caption[Results of fitting the Z = 0 and Z = 2 templates in the $n_{b}^{reco}$ = 0-2 control region to data from the \mupjets control sample, for the \ac{CSV} medium working point, with a jet multiplicity $n_{jet} \geq 5$, in all three \theht categories.]{Results of fitting the Z = 0 and Z = 2 templates in the $n_{b}^{reco}$ = 0-2 control region to data from the \mupjets control sample, for the \ac{CSV} medium working point, with a jet multiplicity $n_{jet} \geq 5$, in all three \theht categories. Data is represented by the black circles with the blue, red and black lines representing the Z=0, Z=2 and combination of both templates respectively. Grey bands represent the uncertainty of the fit. The $\chi^{2}$ parameters represent the goodness of fit to the control and signal region.}
\label{fig:template_data_med_njet5}
\end{figure}
\FloatBarrier
The numerical results and extrapolation to the $n_{b}^{reco} =$3, 4 bins for all \theht and working points, is shown in Table \ref{tab:template_datatable}.

\begin{table}[h!]
\begin{center}
\footnotesize
\begin{tabular*}{0.95\textwidth}{@{\extracolsep{\fill}}lllll}
\cline{1-5}
\multicolumn{2}{c}{\theht} & 275-325 & 325-375 & $>$375 \\
\multicolumn{5}{c}{Loose working point} \\
\hline\hline
Data & \multirow{2}{*}{$n_{b} = 3$} & 838 & 394 & 717\\
Template & & $861.8 \pm 38.1$ & $372.1 \pm 18.4$ & $673.2 \pm 34.5$ \\
Data & \multirow{2}{*}{$n_{b} = 4$} & 81 & 43 & 81 \\
Template & & $78.5 \pm 5.8$ & $27.6 \pm 2.6$ & $78.6 \pm 3.3$ \\
\hline
\multicolumn{5}{c}{Medium working point} \\
\hline\hline
Data & \multirow{2}{*}{$n_{b} = 3$} & 137 & 79 & 152 \\
Template & & $131.2 \pm 4.3$ & $75.1 \pm 2.9$ & $137.8 \pm 5.7$ \\
Data & \multirow{2}{*}{$n_{b} = 4$} & 1 & 1 & 3 \\
Template & & $1.8 \pm 0.1$ & $0.9 \pm 0.1$ & $3.1 \pm 0.2$ \\
\hline
\multicolumn{5}{c}{Tight working point} \\
\hline\hline
Data & \multirow{2}{*}{$n_{b} = 3$} & 24 & 15 & 25 \\
Template & & $23.0 \pm 0.9$ & $12.9 \pm 0.6$ & $20.3 \pm 1.1$ \\
Data & \multirow{2}{*}{$n_{b} = 4$} & 0 & 0 & 1 \\
Template & & $0.1 \pm 0.1$ & $0.1 \pm 0.1$ & $0.2 \pm 0.1$ \\
\end{tabular*}
\end{center}
\caption[Summary of the fit predictions in the $n_{b}^{reco}$ signal region of the \mupjets control sample, after combination of the $n_{jet} = 3, = 4, \geq 5$ categories.. The predictions are extrapolated from a $n_{b}^{reco}$ = 0, 1, 2 control region using 11.4 fb$^{-1}$ of $\sqrt{s} = 8$\TeV data.]{Summary of the fit predictions in the $n_{b}^{reco}$ signal region of the \mupjets control sample, after combination of the $n_{jet} = 3, = 4, \geq 5$ categories. The predictions are extrapolated from a $n_{b}^{reco}$ = 0, 1, 2 control region using 11.4 fb$^{-1}$ of $\sqrt{s} = 8$\TeV data. The uncertainties quoted on the template yields are a combination of statistical and systematic uncertainties.}\label{tab:template_datatable}
\end{table}

When this method is applied to the \mupjets control sample, it is expected that good agreement would be observed between the template predictions and observation in the absence of signal contamination. The good compatibility for all working points as shown in the table, demonstrate that this is the case and that the method is able to accurately predict the background yields. However the assumption of negligible signal contamination can no longer made when applied to the hadronic signal region of the \alphat search, where agreement between estimated backgrounds and observations in data is now not necessarily expected.
 
\subsection{Application to the \alphat Hadronic Search Region}
\label{subsec:templatedataresults}

As an accompaniment to the background estimation methods outlined in the \alphat search, the b-tag template method offers a complementary way of testing the \ac{SM} only background hypothesis within the hadronic signal region of the search. In the presence of a natural \ac{SUSY} signature mediated by a light gluino and containing four underlying $\widetilde{b}$ or $\widetilde{t}$ squarks, which subsequently decay to t or b quarks, the number of reconstructed \nbreco = 3, $\geq 4$ events will be enhanced.

Figure \ref{fig:template_data_signal_njet5} shows the  the results of the template shapes derived from simulation and fitted to data for each of the three \ac{CSV} working points, in the $n_{jet} \geq 5$, \theht $>$ 375 \GeV category.  Grey bands represent the statistical uncertainty of the fit combined in quadrature with the systematic uncertainties of varying the data to simulation scale factors up and down by their measured systematic uncertainties.  Additional fit results for other jet multiplicities are found in Appendix \ref{app:templatedata_signal}.

\begin{figure}[ht]
\footnotesize
\centering
\begin{minipage}[b]{0.51 \linewidth}
\includegraphics[width = 1.0\linewidth]{plots/TemplatesSignal/Final_Fit_To_Data_Normal_Loose_HTBin_Template_375_jet_mult_5.pdf}
\centering (a) Loose working point : $n_{jet} \geq$  5 , \theht $>$ 375
\end{minipage}
\footnotesize
\begin{minipage}[b]{0.51\linewidth}
\includegraphics[width = 1.0\linewidth]{plots/TemplatesSignal/Final_Fit_To_Data_Normal_Medium_HTBin_Template_375_jet_mult_5.pdf}
\centering (b) Medium working point : $n_{jet} \geq$ 5 , \theht $>$ 375 
\end{minipage}
\quad
\begin{minipage}[b]{0.51\linewidth}
\centering
\includegraphics[width = 1.0\linewidth]{plots/TemplatesSignal/Final_Fit_To_Data_Normal_Tight_HTBin_Template_375_jet_mult_5.pdf}
\centering (c) Tight working point :  $n_{jet} \geq$ 5 , \theht $>$ 375 
\end{minipage}
\caption[Results of fitting the Z = 0 and Z = 2 templates in the $n_{b}^{reco}$ = 0-2 control region to data from the hadronic signal selection, in the $n_{jet} \geq 5$ and \theht $>$ 375 category for all \ac{CSV} working points.]{Results of fitting the Z = 0 and Z = 2 templates in the $n_{b}^{reco}$ = 0-2 control region to data from the hadronic signal selection, in the $n_{jet} \geq 5$ and \theht $>$ 375 category for all \ac{CSV} working points. Data is represented by the black circles with the blue, red and black lines representing the Z=0, Z=2 and combination of both templates respectively. Grey bands represent the uncertainty of the fit. The $\chi^{2}$ parameters represent the goodness of fit to the control and signal region.}
\label{fig:template_data_signal_njet5}
\end{figure}
\FloatBarrier
The numerical results and extrapolation to the $n_{b}^{reco} =$3, 4 bins for all \theht and working points are shown in Table \ref{tab:template_signal_table}. Also included within the table are total \ac{SM} background predictions determined by the maximum likelihood fit in both jet multiplicity categories of the \alphat analysis for the \ac{CSVM} tagger, as introduced in Section (\ref{sec:statframework}). No excess of data is found for any of the three \ac{CSV} working points. Predictions within the signal region from this method are also found to be compatible with the background predictions determined by the \alphat simultaneous fit as already shown in Table \ref{tab:fitsdata}.

\begin{table}[h!]
\begin{center}
\footnotesize
\begin{tabular*}{0.95\textwidth}{@{\extracolsep{\fill}}lllll}
\cline{1-5}
\multicolumn{2}{c}{\theht} & 275-325 & 325-375 & $>$375 \\
\multicolumn{5}{c}{Loose working point} \\
\hline\hline
Data & \multirow{2}{*}{$n_{b} = 3$} & 198 & 85 & 126\\
Template &  & $207.1 \pm 33.3$ & $103.4 \pm 10.9$ & $124.98 \pm 16.2$ \\
Data & \multirow{2}{*}{$n_{b} = 4$} & 15 & 9 & 16 \\
Template &  & $15.9 \pm 3.7$ & $8.05 \pm 1.2$ & $13.1 \pm 2.2$ \\
\hline
\multicolumn{5}{c}{Medium working point} \\
\hline\hline
Data & \multirow{3}{*}{$n_{b} = 3$} & 32 & 16 & 15 \\
Template & & $26.4 \pm 1.7$ & $12.7 \pm 1.2$ & $19.9 \pm 2.8$ \\
\alphat ML Fit &  & $29.8 ^{+5.2}_{-4.4}$ & $14.0 ^{+1.8}_{-2.0}$ & $16.5 ^{+1.4}_{-1.4}$ \\ [1.1ex]
Data & \multirow{3}{*}{$n_{b} = 4$} & 1 & 0 & 2 \\
Template &  & $0.3 \pm 0.2$ & $0.3 \pm 0.1$ & $0.5 \pm 0.2$ \\
\alphat ML Fit &  & $0.9 ^{+0.4}_{-0.7}$ & $0.3 ^{+0.2}_{-0.2}$ & $0.6 ^{+0.3}_{-0.3}$ \\  [1.1ex]
\hline
\multicolumn{5}{c}{Tight working point} \\
\hline\hline
Data & \multirow{2}{*}{$n_{b} = 3$} & 5 & 2 & 0 \\
Template & & $4.03 \pm 0.3$ & $2.4 \pm 0.3$ & $3.1 \pm 0.3$ \\
Data & \multirow{2}{*}{$n_{b} = 4$} & 1 & 0 & 0 \\
Template & & $0.1 \pm 0.1$ & $0.1 \pm 0.1$ & $0.0 \pm 0.1$ \\
\end{tabular*}
\end{center}
\caption[Summary of the fit predictions in the $n_{b}^{reco}$ signal region of the \alphat hadronic signal selection, after combination of the $n_{jet} = 3, = 4, \geq 5$ categories. The predictions are extrapolated from a $n_{b}^{reco}$ = 0, 1, 2 control region using 11.7 fb$^{-1}$ of $\sqrt{s} = 8$\TeV data.]{Summary of the fit predictions in the $n_{b}^{reco}$ signal region of the \alphat hadronic signal selection, after combination of the $n_{jet} = 3, = 4, \geq 5$ categories. The predictions are extrapolated from a $n_{b}^{reco}$ = 0, 1, 2 control region using 11.7 fb$^{-1}$ of $\sqrt{s} = 8$\TeV data. The uncertainties quoted on the template yields are a combination of statistical and systematic uncertainties.}\label{tab:template_signal_table}
\end{table}
\FloatBarrier


\section{Summary}
\label{subsec:templateconclusions}

A \ac{SUSY} signature such as one from gluino-induced third-generation squark production, would result in a final state with an underlying b-quark content greater than two. In order to be able to discriminate such signatures from the \ac{SM} background, templates are generated based on a parameterisation of \ac{SM} processes, where the underlying b-quarks per event is typically zero or two. These templates are then fit to data in a low $n_{b}^{reco}$ (0-2) control region in order to extrapolate a prediction within a high $n_{b}^{reco}$ (3-4) signal region. This approach is built upon the assumptions that the defined control region is almost entirely free of any possible signal contamination from possible signal topologies with a small number of b quarks in the final state.

The method was demonstrated both in simulation and also in data, using the \ac{SM} enriched \mupjets selection from the \alphat search. This was conducted to prove conceptually and experimentally that the method is valid and that there is adequate control over the efficiency and mis-tagging rates in data for all working points of the \ac{CSV} tagger. Additionally this method was further applied to the \alphat analysis signal region, where good agreement is observed between the predictions from the template extrapolations, observations in data and the background estimation method of the \alphat analysis.

