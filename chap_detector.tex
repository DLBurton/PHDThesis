\chapter{The LHC and the CMS Detector}
\label{chap:cmsoverview}

Probing the \SM for signs of new physics would not be possible without the immensely complex� \\

This chapter will cover \CERN 's  Large Hadron Collider (\LHC) and the CMS detector, being the experiment the author is a member of. Section \ref{sec:cmsdetector} serves to introduce an overview of the different components of  the CMS detector, with more detail spent on those that are relevant in the search for Supersymmetric particles. Section \ref{sec:cmsobjects} will focus on event and object reconstruction again with more emphasis on jet level quantities which are most relevant to the author's analysis research. Finally Section \ref{sec:l1trigger} will cover work performed by the author, as service to the CMS Collaboration, in measuring the performance of the GCT component of the L1 trigger during the 2012-2013 run period.  


\section{The LHC}
\label{sec:thelhc} 

The \LHC is a storage ring, accelerator, and collider of circulating beams of protons or ions. Housed in the tunnel dug for the Large Electron-Positron collider (LEP), it is approximately 27 km in circumference, 100 m underground, and straddles the border between France and Switzerland outside of Geneva. It is currently the only collider in operation that is able to study physics at the TeV scale. 


\section{CMS detector}
\label{sec:cmsdetector}

Detector stuff


\section{Object Definition}
\label{sec:cmsobjects}

Object stuff

\subsection{Jets}
\label{subsec:cmsobjects-jets}

Jets

\subsection{B-tagging}
\label{subsec:cmsobjects-btagging}

B-tagging

\section{L1 Trigger}
\label{sec:l1trigger}


L1 Work
